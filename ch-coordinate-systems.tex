\ProvidesFile{ch-coordinate-systems.tex}[Coordinate Systems]
\graphicspath{{/Users/liamrobinson/Documents/PyLightCurves/docs/build/html/_images}}

\chapter{Background}

\section{Time Systems}

\subsection{Time Scales}

There are a variety of scales used to measure time. What follows is a minimal treatment of each. For a more comprehensive overview, see Section 3.5 of \cite{vallado4ed}. International Atomic Time (TAI) is based on measurements from atomic clocks and is independent of astronomical effects or observations. By definition, TAI proceeds at the rate of $1$ SI second per second. Universal Time (UT0) is derived directly from observations of the apparent position of the stars. UT1 is derived from UT0 by adjusting for polar motion. UT1 is offset from TAI by $\Delta UT1$, which is a dynamic quantity that must be continually observed. Universal Coordinated Time (UTC) is a truncation of UT1 that uses an integer number of leap seconds $\Delta AT$ to stay within $0.9$ seconds of TAI. Terrestrial Time (TT) is defined by a constant offset of $TT - TAI = 32.184$ seconds from TAI and preceeding at the same rate as TAI. These time scale relations are summarized in Eq \ref{eq:time_scale_conversions}.

\begin{align*}  \label{eq:time_scale_conversions} \numberthis
  UTC &= UT1 - \Delta UT1 \\
  TAI &= UTC + \Delta AT \\
  TT &= TAI + 32.184^s \\
\end{align*}

These time scales are relevant for this research as the precise coordinate frame transformation from ITRF to the J2000.0 realization of ICRF relies on quanities expressed in UT1. Date timestamps are usually standardized to UTC, requiring the transformations in Eq \ref{eq:time_scale_conversions} for full accuracy. Figure \ref{fig:time_scales} shows the evolution of UTC, UT1, and TT with respect to TAI. Notice that $\Delta UT1$ continually changes while $\Delta AT$ is always truncated to a nearby integer.

\begin{figure}[ht]
  \centering
  \includegraphics[width=\figmed]{sphx_glr_time_systems_001.png}
  \caption{Time scales relative to TAI}
  \label{fig:time_scales}
\end{figure}

\subsection{Julian Date}

Most tasks in astrodynamics are easier when using a continuous time system. For this reason, the Julian date is adopted. This quantity is defined is the number of days elapsed since January 1, 4713 B.C., at 12:00 \cite{vallado4ed}. Given a date timestamp of the form D/M/Y h:m:s between the years of 1900 and 2100, the Julian date is computed via Eq \ref{eq:date_to_jd}. Note that Eq \ref{eq:date_to_jd} is always a function of the time scale used in the input, i.e., a UTC timestamp yields $JD_{UTC}$ whereas a UT1 timestamp yields $JD_{UT1}$.

\begin{equation} \label{eq:date_to_jd}
  JD = 376Y - \fl{\left[ \frac{7Y + 7 \cdot \fl{\left(\frac{M + 9}{12} \right)}}{4} \right]}
      + \fl{\left(\frac{275M}{9}\right)} 
      + d
      + 1721013.5
      + \frac{\frac{\left(\frac{s}{60} + 60\right)}{60} + h}{24}
\end{equation}

Another useful quantity for later time and coordinate system calculations is the number of Julian centuries since a particular epoch. The J2000.0 epoch is used unless otherwise stated, resulting in Eq \ref{eq:jd_to_t} \cite{vallado4ed}.

\begin{equation} \label{eq:jd_to_t}
  T = \frac{JD - 2451545.0}{36535}
\end{equation}

Often, more specificity is needed with respect to the time scale used in Eq \ref{eq:jd_to_t}. For example, computing $T$ with an input date in UT1 yields $T_{UT1}$ using $JD_{UT1}$, which is in turn a function a date timestamp expressed in UT1. 

\subsection{Solar and Sidereal Time}

A solar day is defined as the time required for the Sun to pass and return to an observer's meridian --- a line of constant longitude extending from the geographic south pole to the geographic north pole \cite{vallado4ed}. By contrast, a sidereal day is the time required for the stars to complete a revolution around an observer's meridian. Due to the Earth's orbit around the Sun, the sidereal day is about 4 minutes shorter than the solar day \cite{vallado4ed}. The Greenwich mean sidereal time (GMST) is computed in seconds via Eq \ref{eq:date_to_gmst} \cite{frueh2019notes}.

\begin{equation} \label{eq:date_to_gmst}
  \theta_{GMST} = 67310.54841
        + \left(3.15576 \cdot 10^9 + 8640184.812866 \right) T_{UT1}
        + 0.093104 T_{UT1}^2
        - 6.2 \cdot 10^{-6} T_{UT1}^3
\end{equation}

Accounting for the variations in the inclination of the ecliptic $\epsilon$ and the the change in the equinox compared to the reference epoch $\Delta \Psi$ produces Greenwich apparent sidereal time (GAST) via Eq \ref{eq:date_to_gast} \cite{frueh2019notes}. 

\begin{equation} \label{eq:date_to_gast}
  \theta_{GAST} = \theta_{GMST} + \Delta \Psi \cos\epsilon
\end{equation}

Both the inclination of the ecliptic and the difference in the equinox are computed with series expansions following the IAU 1980 theory of nutation \cite{vallado4ed}.

\section{Coordinate Systems}

\subsection{Altitude References}

\subsubsection{Ellipsoid}

Due to Earth's equatorial bulge, it is common to model the rough shape of the Earth as an ellipsoid. In particular, the 1984 World Geodetic Survey (WGS-84) model is used throughout this work to define the shape of the Earth ellipsoid, with parameters listed in Table \ref{tb:wgs84}.

\begin{table}[ht]
  \centering
  \begin{tabular}{|l|l|}
  \hline
  \textbf{Parameter} & \textbf{Value}              \\ \hline
  Equatorial radius $R_E$             & $6378.137 \: [km]$ \\ \hline
  Flattening ratio $f$                & $1 / 298.257$      \\ \hline
  \end{tabular}
  \caption{WGS-84 ellipsoid model of the Earth \cite{vallado4ed}}
  \label{tb:wgs84}
\end{table}

\subsubsection{Geoid}

The geoid accounts for the gravitational potential differences across the Earth's surface \cite{vallado4ed}. It is a surface of equal gravitational potential; the surface the ocean relaxes to without the influence of the wind and tides \cite{vallado4ed}. For this reason, the geoid is alternatively known as the mean sea level (MSL). The ellipsoid is a good approximation of the geoid, which deviates from the ellipsoid by less than $\approx 100$ meters at all latitudes and longitudes. The height of the geoid above the ellipsoid can be computed from a high-fidelity gravity model, but it is often more convenient to interpolate a pre-computed grid of geoid heights. Figure \ref{fig:geoid_shape} displays global geoid heights derived from the 1996 Earth Gravitational Model (EGM-96) relative to the ellipsoid.

\begin{figure}[ht]
  \centering
  \includegraphics[width=\figmed]{sphx_glr_geoid_heights_001_2_00x.png}
  \caption{EGM-96 geoid heights above the WGS-84 ellipsoid}
  \label{fig:geoid_shape}
\end{figure}

\subsubsection{Terrain}

Terrain elevation is usually the final component needed to fully define the altitude of a ground station, which is often defined relative to MSL. This work uses $30$-meter terrain tiles from the Shuttle Radar Topography Mission (SRTM). Figure \ref{fig:pogs_terrain} shows the local elevation around the Purdue Optical Ground Station using SRTM data.

\begin{figure}[ht]
  \centering
  \includegraphics[width=\figsmall]{sphx_glr_pogs_local_terrain_001.png}
  \caption{MSL elevations surrounding the Purdue Optical Ground Station}
  \label{fig:pogs_terrain}
\end{figure}

\subsection{Latitude, Longitude and Altitude}

Latitude, longitude, and altitude (LLA) are a spherical coordinates representation of position on or above the surface of the Earth. For the purposes of precise station positioning, the difference between the two types of longitude --- geocentric and geodetic --- is important. Geocentric latitude is the angle between the line from the center of mass of the Earth to the position of interest and the equatorial plane. Geodetic latitude instead measures the angle between the local ellipsoid surface normal and the equatorial plane. Geodetic latitude $\phi_{geod}$ is converted to geocentric $\phi_{geoc}$ latitude with Eq \ref{eq:geod_to_geoc} \cite{frueh2019notes}.

\begin{equation} \label{eq:geod_to_geoc}
  \phi_{geoc} = \tan^{-1} \left((1 - f)^2 \tan\phi_{geod} \right)
\end{equation}

Additionally, the radius of the ellipsoid $r_E$ at a given geocentric latitude is necessary for later conversion, expressed by Eq \ref{eq:rad_at_geoc} \cite{frueh2019notes}.

\begin{equation} \label{eq:rad_at_geoc}
  r_E = R_E - f \sin^2 \left( \phi_{geoc} \right)
\end{equation}

\subsection{International Terrestrial Reference Frame (ITRF)}

The cartesian form of LLA is known as the Earth-centered Earth-fixed (ECEF) reference frame. Throughout this work, ECEF and ITRF will be used interchangeably. This frame has its origin at the center of mass of the Earth and its axes fixed in the crust. The
fundamental plane of the frame is defined to be the equator ---  orienting the $z$-axis through Earth's
instantaneous spin axis, and the reference direction through the intersection of the prime meridian
and the equator ---  defining the $x$-axis. Completing the right-handed system with $\hat{y} = \hat{z} \times \hat{x}$ yields a
reference frame that remains fixed, neglecting effects like continental drift. The transformation from LLA $\left( \lambda, \phi_{geod}, a \right)$ to ITRF is given by Eq \ref{eq:lla_to_itrf}.

\begin{align*} \numberthis \label{eq:lla_to_itrf}
  e^2 &= 2f - f^2 \\
  N &= \frac{R_E}{\sqrt(1 - e^2 \sin(\phi_{geod})^2)} \\
  \rho &= (N + a) \cos(\phi_{geod}) \\
  x_{itrf} &= \rho \cos(\lambda) \\
  y_{itrf} &= \rho \sin(\lambda) \\
  z_{itrf} &= \left(N (1 - e^2) + a \right) \sin(\phi_{geod}) \\
\end{align*}

In Eq \ref{eq:lla_to_itrf}, $e^2$ is the squared eccentricity of the ellipsoid, $N$ is the radius of curvature in the meridian, and $\rho$ is the $x-y$ plane magnitude of the station's position \cite{vallado4ed}.

Many later transformations require the body axis rotation matrices $R_1$, $R_2$, and $R_3$ which are expressed in Eq \ref{eq:body_rotms}.

\begin{align*} \numberthis \label{eq:body_rotms}
  R_1(\theta) &= \begin{bmatrix}  1 & 0 & 0 \\ 0 & \cos\theta & \sin\theta \\ 0 & -\sin\theta & \cos\theta \end{bmatrix} \\
  R_2(\theta) &= \begin{bmatrix}  \cos\theta & 0 & -\sin\theta \\ 0 & 1 & 0 \\ \sin\theta & 0 & \cos\theta \end{bmatrix} \\
  R_3(\theta) &= \begin{bmatrix}  \cos\theta & \sin\theta & 0 \\ -\sin\theta & \cos\theta & 0 \\ 0 & 0 & 1 \end{bmatrix} \\
\end{align*}

\subsection{Topocentric Reference Frame (ENU)}

The remaining transformations in this chapter will only be defined in terms of their rotation matrices. It is often useful to express observations in a local reference frame. The East North Up (ENU) coordinate system is used throughout this work. This system has an origin at the observing station, with the first two basis vectors pointing towards the local East and North and the third pointing towards zenith. The transformation from ITRF to ENU is given by Eq \ref{eq:itrf_to_enu}.

\begin{equation} \label{eq:itrf_to_enu}
  \vec{r}_{enu} = F_2 F_1 R_2(\phi_{geoc}) R_3(\lambda) \vec{r}_{itrf}
\end{equation}

In Eq \ref{eq:itrf_to_enu}, $R_3$ is a rotation about the third body axis, $F_1$ swaps the second and third unit vectors, and $F_2$ swaps the first and third unit vectors. The orientation of the ENU reference frame at the Purdue Optical Ground Station is depicted in Figure \ref{fig:pogs_enu}.

\begin{figure}[ht]
  \centering
  \includegraphics[width=\figmed]{sphx_glr_az_el_parallel_001.png}
  \caption{ENU reference frame orientation at Purdue Optical Ground Station}
  \label{fig:pogs_enu}
\end{figure}

\subsection{International Celestial Reference Frame (ICRF)}

Transforming from ITRF to the a standardized intertial reference frame is an involved process due to the variety of nonlinear effects impacting the Earth's rotational motion. In total, this transformation must account for polar motion, the nutation and precession of the Earth's pole, and the mean sidereal time. These transformations are treated much more thoroughly in Vallado \cite{vallado4ed}. 

Accounting for polar motion --- the motion of the Earth's pole that cannot be explained through nutation theory --- transforms from ITRF to Greenwich True of Date (GTOD) via Eq \ref{eq:itrf_to_gtod}, where $x_p$ and $y_p$ are the angular components of the polar motion at the time of interest \cite{frueh2019notes}.

\begin{equation} \label{eq:itrf_to_gtod}
  \vec{r}_{gtod} = R_1(y_p) R_2(x_p) \vec{r}_{itrf}
\end{equation}

Accounting for the sidereal rotation of the Earth about its pole transforms from GTOD to the True Equator, Mean Equinox (TEME) reference frame via Eq \ref{eq:gtod_to_teme} \cite{frueh2019notes}.

\begin{equation} \label{eq:gtod_to_teme}
  \vec{r}_{teme} = R_3(-\theta_{GMST}) \vec{r}_{gtod}
\end{equation}

Accounting for the difference between GMST and GAST at the date of interest transforms from TEME to the True of Date (TOD) reference frame via Eq \ref{eq:teme_to_tod} \cite{vallado4ed}.

\begin{equation} \label{eq:teme_to_tod}
  \vec{r}_{tod} = R_3(-\Delta \Psi \cos \epsilon) \vec{r}_{teme}
\end{equation}

Accounting for the nutation of Earth's pole transforms from TOD to the Mean of Date (MOD) reference frame via Eq \ref{eq:tod_to_mod}, where $\bar{\epsilon}$ is the mean inclination of the ecliptic at the time of interest, and $\epsilon$ is the true inclination of the ecliptic \cite{vallado4ed}.

\begin{equation} \label{eq:tod_to_mod}
  \vec{r}_{mod} = R_1(-\bar{\epsilon}) R_3(\Delta\Psi) R_1(\bar{\epsilon} + \Delta\epsilon) \vec{r}_{tod}
\end{equation}

Accounting for the secular precession of Earth's pole transforms from MOD to ICRF via Eq \ref{eq:mod_to_icrf} through the 3-2-3 Euler angle sequence $\left( z, \theta, \zeta \right)$, which are each a function of the date of the transformation \cite{frueh2019notes}.

\begin{equation} \label{eq:mod_to_icrf}
  \vec{r}_{mod} = R_3(\zeta) R_2(\theta) R_3(z) \vec{r}_{tod}
\end{equation}

\subsection{Right Ascension and Declination}

Right ascension and declination, often shortened to RA/Dec, are useful angles from describing the angular position of an object
on the celestial sphere from the perspective of an observer. Right ascension is defined as the angle
of the observation projected onto the inertial $x-y$ plane, measured counterclockwise from inertial
$\hat{x}$, represented by $\alpha$. Declination is the angle from the $x-y$ plane to the observation
with positive values above the $x-y$ plane (closer to inertial $z$) and negative values below.
Declination is represented by $\delta$. Given a unit vector direction $\hat{v} = \left[ x, y, z \right]^T$ in
inertial space, RA/Dec is computed via Eq \ref{eq:eci_to_ra_dec} \cite{frueh2019notes}.

\begin{equation} \label{eq:eci_to_ra_dec}
  \begin{bmatrix}
	\alpha \\
	\delta
  \end{bmatrix} = 
  \begin{bmatrix}
	\atantwo(y, x) \\
	\atantwo(z, \sqrt{x^2 + y^2})
  \end{bmatrix}
\end{equation}

\subsection{Azimuth and Elevation}

Azimuth and elevation, often shortened to Az/El, are similar angular quantities to right ascension and declination \cite{frueh2019notes}. Instead of being based on
the inertial sphere, they are referenced to an arbitrary reference frame. For a telescope making
observations of an object, the local East-North-Up (ENU) frame may be used. For a satellite star
tracker, star azimuth and elevation might be reported in the satellite body frame. In either case,
Eq \ref{eq:eci_to_ra_dec} can be repurposed in terms of Az/El, where $\hat{v} = \left[ x, y, z
\right]^T$ is expressed in the frame of interest \cite{frueh2019notes}.

\begin{equation} \label{eq:enu_to_az_el}
  \begin{bmatrix}
	Az \\
	El
  \end{bmatrix} = 
  \begin{bmatrix}
	\atantwo(y, x) \\
	\atantwo(z, \sqrt{x^2 + y^2})
  \end{bmatrix}
\end{equation}

Note that Eq \ref{eq:enu_to_az_el} references azimuth to the $x$-axis, proceeding in the
counterclockwise direction. Often, this reference axis and direction may be changed depending on the
reference frame being used. For example, ground station observations may be referenced to local
North ---  the second axis of the ENU system ---  proceeding clockwise. This would require the
substitution $Az' = \frac{\pi}{2} - Az$. Notice that this substitution leads to $Az'$ leaking
outside the domain of $[0, 2\pi)$. This is not an issue for later coordinate transformations, but
may be undesirable for plots. Wrapping the result back to the standard azimuth range via
$Az_{wrapped} = \textrm{mod}(Az, 2\pi)$ is a sufficient fix.
