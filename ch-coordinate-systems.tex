\ProvidesFile{ch-coordinate-systems.tex}[Coordinate Systems]
\graphicspath{{/Users/liamrobinson/Documents/PyLightCurves/docs/build/html/_images}}

\section{Coordinate Systems}

\subsection{International Terrestrial Reference Frame}

The most intuitive Earth-centered reference frame is Earth-centered Earth-fixed (ECEF). An ECEF
frame has its origin at the center of mass of the Earth and its axes fixed in the crust. The
fundamental plane of the frame is defined to be the equator ---  defining the $z$-axis through Earth's
instantaneous spin axis, and the reference direction through the intersection of the prime meridian
and the equator ---  defining the $x$-axis. Completing the right-handed system the $y$-axis yields a
reference frame that remains fixed, neglecting continental drift and other pesky (but sufficiently
negligible) realities. 

\subsection{Right Ascension and Declination}

Right ascension and declination, often shortened to RA/Dec, are useful angles from describing the angular position of an object
on the celestial sphere from the perspective of an observer. Right ascension is defined as the angle
of the observation projected onto the inertial $x-y$ plane, measured counterclockwise from inertial
$\hat{x}$, represented by $\alpha$. Declination is the angle from the $x-y$ plane to the observation
with positive values above the $x-y$ plane (closer to inertial $z$) and negative values below.
Declination is represented by $\delta$. Given a unit vector direction $\hat{v} = \left[ x, y, z \right]^T$ in
inertial space, we can compute RA/Dec via Eq \ref{eq:eci_to_ra_dec}.

\begin{equation} \label{eq:eci_to_ra_dec}
  \begin{bmatrix}
	\alpha \\
	\delta
  \end{bmatrix} = 
  \begin{bmatrix}
	\atantwo(y, x) \\
	\atantwo(z, \sqrt(x^2 + y^2))
  \end{bmatrix}
\end{equation}

\subsection{Azimuth and Elevation}

Azimuth and elevation, often shortened to Az/El, are similar angular quantities to right ascension and declination, but instead of being based on
the inertial sphere, they are referenced to an arbitrary reference frame. For a telescope making
observations of an object, the local East-North-Up (ENU) frame may be used. For a satellite star
tracker, star azimuth and elevation might be reported in the satellite body frame. In either case,
Eq \ref{eq:eci_to_ra_dec} can be repurposed in terms of Az/El, where $\hat{v} = \left[ x, y, z
\right]^T$ is expressed in the frame of interest.

\begin{equation} \label{eq:enu_to_az_el}
  \begin{bmatrix}
	Az \\
	El
  \end{bmatrix} = 
  \begin{bmatrix}
	\atantwo(y, x) \\
	\atantwo(z, \sqrt(x^2 + y^2))
  \end{bmatrix}
\end{equation}

Note that Eq \ref{eq:enu_to_az_el} references azimuth to the $x$-axis, proceeding in the
counterclockwise direction. Often, this reference axis and direction may be changed depending on the
reference frame being used. For example, ground station observations may be referenced to local
North ---  the second axis of the ENU system ---  proceeding clockwise. This would require the
substitution $Az' = \frac{\pi}{2} - Az$. Notice that this substitution leads to $Az'$ leaking
outside the domain of $[0, 2\pi)$. This is not an issue for later coordinate transformations, but
may be undesirable for plots. Wrapping the result back to the standard azimuth range via
$Az_{wrapped} = \textrm{mod}(Az, 2\pi)$ is a sufficient fix.

\section{Time Systems}

\begin{figure}[ht]
  \centering
  \includegraphics[width=\figmed]{sphx_glr_time_systems_001.png}
  \caption{Time systems relative to TAI}
  \label{fig:time_systems}
\end{figure}