\ProvidesFile{ch-coordinate-systems.tex}[Coordinate Systems]
\graphicspath{{/Users/liamrobinson/Documents/PyLightCurves/docs/build/html/_images}}

\section{Coordinate Systems}

\subsection{International Terrestrial Reference Frame}

The most intuitive Earth-centered reference frame is Earth-centered Earth-fixed (ECEF). An ECEF
frame has its origin at the center of mass of the Earth and its axes fixed in the crust. The
fundamental plane of the frame is defined to be the equator ---  defining the $z$-axis through Earth's
instantaneous spin axis, and the reference direction through the intersection of the prime meridian
and the equator ---  defining the $x$-axis. Completing the right-handed system the $y$-axis yields a
reference frame that remains fixed, neglecting continental drift and other pesky (but sufficiently
negligible) realities. 

\subsection{Right Ascension and Declination}

Right ascension and declination, often shortened to RA/Dec, are useful angles from describing the angular position of an object
on the celestial sphere from the perspective of an observer. Right ascension is defined as the angle
of the observation projected onto the inertial $x-y$ plane, measured counterclockwise from inertial
$\hat{x}$, represented by $\alpha$. Declination is the angle from the $x-y$ plane to the observation
with positive values above the $x-y$ plane (closer to inertial $z$) and negative values below.
Declination is represented by $\delta$. Given a unit vector direction $\hat{v} = \left[ x, y, z \right]^T$ in
inertial space, we can compute RA/Dec via Eq \ref{eq:eci_to_ra_dec}.

\begin{equation} \label{eq:eci_to_ra_dec}
  \begin{bmatrix}
	\alpha \\
	\delta
  \end{bmatrix} = 
  \begin{bmatrix}
	\atantwo(y, x) \\
	\atantwo(z, \sqrt(x^2 + y^2))
  \end{bmatrix}
\end{equation}

\subsection{Azimuth and Elevation}

Azimuth and elevation, often shortened to Az/El, are similar angular quantities to right ascension and declination, but instead of being based on
the inertial sphere, they are referenced to an arbitrary reference frame. For a telescope making
observations of an object, the local East-North-Up (ENU) frame may be used. For a satellite star
tracker, star azimuth and elevation might be reported in the satellite body frame. In either case,
Eq \ref{eq:eci_to_ra_dec} can be repurposed in terms of Az/El, where $\hat{v} = \left[ x, y, z
\right]^T$ is expressed in the frame of interest.

\begin{equation} \label{eq:enu_to_az_el}
  \begin{bmatrix}
	Az \\
	El
  \end{bmatrix} = 
  \begin{bmatrix}
	\atantwo(y, x) \\
	\atantwo(z, \sqrt(x^2 + y^2))
  \end{bmatrix}
\end{equation}

Note that Eq \ref{eq:enu_to_az_el} references azimuth to the $x$-axis, proceeding in the
counterclockwise direction. Often, this reference axis and direction may be changed depending on the
reference frame being used. For example, ground station observations may be referenced to local
North ---  the second axis of the ENU system ---  proceeding clockwise. This would require the
substitution $Az' = \frac{\pi}{2} - Az$. Notice that this substitution leads to $Az'$ leaking
outside the domain of $[0, 2\pi)$. This is not an issue for later coordinate transformations, but
may be undesirable for plots. Wrapping the result back to the standard azimuth range via
$Az_{wrapped} = \textrm{mod}(Az, 2\pi)$ is a sufficient fix.

\section{Time Systems}

\subsection{Julian Date}

Most tasks in space domain awareness and astrodynamics more generally are easier when using a continuous time system. For that reason, we adopt the Julian date. This quantity is defined is the number of days elapsed since January 1, 4713 B.C., at 12:00 \cite{vallado4ed}. Given a date timestamp of the form D/M/Y h:m:s between the years of 1900 and 2100, we can compute the Julian date via Eq \ref{eq:date_to_jd}.

\begin{equation} \label{eq:date_to_jd}
  JD = 376Y - \fl{\left[ \frac{7Y + 7 \cdot \fl{\left(\frac{M + 9}{12} \right)}}{4} \right]}
      + \fl{\left(\frac{275M}{9}\right)} 
      + d
      + 1721013.5
      + \frac{\frac{\left(\frac{s}{60} + 60\right)}{60} + h}{24}
\end{equation}

Another useful quantity for later time and coordinate system calculations is the number of Julian centuries since a particular epoch. In particular, we will often use the J2000.0 epoch, resulting in Eq \ref{eq:jd_to_t} \cite{vallado4ed}.

\begin{equation} \label{eq:jd_to_t}
  T = \frac{JD - 2451545.0}{36535}
\end{equation}

Often we need to be more specific about the time scale being used in Eq \ref{eq:jd_to_t}. For example, if we wanted the number of Julian centuries for an input date in UT1, we would compute $T_{UT1}$ using $JD_{UT1}$, which is in turn a function a date timestamp expressed in UT1. 

\subsection{Solar and Sidereal Time}

Due to Earth's orbit around the Sun, we need to differentiate between solar and sidereal time. A solar day is defined as the time required for the Sun to pass and return to an observer's meridian --- a line of constant longitude extending from the geographic south pole to the geographic north pole \cite{vallado4ed}. By contrast, a sidereal day is the time required for the stars to complete a revolution around an observer's meridian. 

We can compute the sidereal time in seconds via Eq \ref{eq:date_to_sidereal}.

\begin{equation} \label{eq:date_to_sidereal}
  \theta_{GMST} = 67310.54841
        + (3.15576e+09 + 8640184.812866) T_{UT1}
        + 0.093104 T_{UT1}^2
        - 6.2 \cdot 10^{-6} T_{UT1}^3
\end{equation}

\subsection{Time Scales}

There are many slightly different scales for measuring the passage of time. What follows is a minimal treatment of each. For a more comprehensive overview, see the descriptions in Vallado that this section draws from \cite{vallado4ed}. International Atomic Time (TAI) is based on measurements from atomic clocks. Universal Time (UT0) is derived directly from observations of the apparent position of the stars. UT1 is derived from UT0 by adjusting for polar motion. UT1 is offset from TAI by $\Delta UT1$, which is a dynamic quantity that must be continually observed. Universal Coordinated Time (UTC) is a truncation of UT1 that uses an integer number of leap seconds $\Delta AT$ to stay within $0.9$ seconds of TAI. Terrestrial Time (TT) is defined by a constant offset of $32.184$ seconds from TAI and preceeding at the same rate as TAI. Global Positioning System (GPS) time is also defined by a constant offset of $19.0$ seconds from TAI and preceeds at the same rate as TAI. These time scale relations are summarized in Eq \ref{eq:time_scale_conversions}.

\begin{align*}  \label{eq:time_scale_conversions} \numberthis
  UTC &= UT1 - \Delta UT1 \\
  TAI &= UTC + \Delta AT \\
  TT &= TAI + 32.184^s \\
  TAI &= GPS + 19.0^s
\end{align*}

Figure \ref{fig:time_scales} shows the evolution of UTC, UT1, and TT with respect to TAI. Notice that $\Delta UT1$ continually changes while $\Delta AT$ is always an integer.

\begin{figure}[ht]
  \centering
  \includegraphics[width=\figmed]{sphx_glr_time_systems_001.png}
  \caption{Time scales relative to TAI}
  \label{fig:time_scales}
\end{figure}