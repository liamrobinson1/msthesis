\ProvidesFile{ch-light-curve-simulation.tex}[Light Curve Simulation]

\chapter{Light Curve Simulation}

\subsection{Simulating Convex Objects}

Light curve simulation for convex geometry can be solved semi-analytically as each facet's contribution 
to the measured irradiance can be computed individually \cite{kaasalainen2001}. 
Determining whether a face is illuminated requires two horizon checks to determine visibility 
from the Sun and to the observer. For a facet $i$ at timestep $j$ these horizon checks are 
expressed by the shadowing condition $\mu_{ij}$. 

\begin{equation} \label{eq:cvx_shadow_cond}
  \mu_{ij} = \begin{cases}
    1 \text{ if } \left( \hat{O}_j \cdot \hat{n}_i \right) > 0 \text{ and } \left( \hat{S}_j \cdot \hat{n}_i \right) > 0 
	  \text{ and } \delta_{ij,\text{ss}} = 0 \text{ and } \delta_{ij,\text{os}} = 0\\
    0 \text{ otherwise } \\
  \end{cases}
\end{equation}

The unit vectors $\hat{O}$ and $\hat{S}$ point from the  center of mass of the object to the observer and Sun, respectively. 
We choose the outward-pointing facet normal unit vector $\hat{n}$ by convention for all mesh operations. 
The self-shadowing and observer-shadowing conditions, $\delta_{ij,\text{ss}}$ and $\delta_{ij,\text{os}}$, 
are always zero for convex polyhedra but are crucial for accurately simulating non-convex geometry. 
For objects with concavities, self-shadowing refers to shadows cast by an object onto itself and observer-shadowing 
refers to otherwise visible faces blocked by other portions of the geometry.

The irradiance $I$ received by the observer at timestep $j$ is the sum of the received irradiance from all facets, 
composed of specular and diffuse contributions. We express each contribution as the product of the
normalized irradiance $\hat{I}$. This can be scaled to adjust for the distance from the observer to
the object to yield the noiseless received irradiance.
