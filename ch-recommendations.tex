\ProvidesFile{ch-recommendations.tex}[Recommendations]
\chapter{Recommendations}

\chapter{Future Work}

The next step with this work would be to extensively validate the CCD performance model against real observations taken by the Purdue Optical Ground Station. This would involve quantifying the sensor noise sources through zero second exposures and closed shutter exposures to isolate the effects of the dark noise and readout noise. Next, the background model would be fit to an array of observations at different times of the night, zenith angles, and lunar phases. Once both the background and sensor effects are well calibrated, the overall CCD model can be tested against the observed signal counts and light curve SNR of known objects.

More work must also be done to perform light curve inversion on actively-controlled box-wing satellites. This would involve a simple adaptation of the EGI inversion method where some normal vectors are allowed to move in the body frame, enabling solar panels to track the Sun. Such a parametric approach could be used to check whether the light curve of an unknown object --- in an unknown attitude profile --- fits the expected light curve of a box-wing satellite with Sun tracking solar panels. 

Finally, investigating shape inversion with multispectral measurements may open new avenues for simultaneously estimating attitude or material properties in addition to a convex shape. This might be accomplished operationally by switching filters between images, or by using a dedicated multispectral sensor. In either case, adding a second axis to the available data would enable far more robust characterization of  space objects with fewer assumptions.