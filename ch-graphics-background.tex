\ProvidesFile{ch-graphics-background.tex}[Computer Graphics]
\graphicspath{{/Users/liamrobinson/Documents/PyLightCurves/docs/build/html/_images}}

\section{Computer Graphics}

The light curve simulation framework developed in this work relies on a pixel-by-pixel image rendering on the GPU to compute the total reflected irradiance at each timestep of the simulation. As a result, a background in common computer graphics terminology and reference frames is necessary to understand the transformation from coordinates in the object body frame to a pixel in the final image.

\subsubsection{Camera Projections and Terminology}

In computer graphics, a camera is defined by its position $R_{cam}$, target $T_{cam}$, reference up direction $U_{cam}$, and a field of view $FOV$. In order to render a 3D scene to a 2D image using this camera, a transformation is needed. An orthographic camera accomplishes this transformation by orthogonally projecting points onto a plane perpendicular to camera look direction $T_{cam} - R_{cam}$. The volume of space that falls into view of the camera is known as the frustum \cite{shirley2009}. An important consequence of orthogonal projection is that objects further away from the camera do not shrink in the image, which is sometimes advantageous. A perspective projection forms a more complex frustum that expands outwards from a near plane to a proportionally larger far plane. Because of the expansion of the frustum, geometry processed by the perspective transformation shrinks as it moves further from the near plane. Figure \ref{fig:ortho_perspective_cameras} displays the geometry of these camera frustums, with additional reference frames detailed in Section \ref{sec:graphics_trans}.

\begin{figure}[!htb]
  \centering
  \includegraphics[width=\figbig]{sphx_glr_computer_graphics_background_001.png}
  \caption{Perspective and orthographic projections with relevant camera frustum attributes labeled}
  \label{fig:ortho_perspective_cameras}
\end{figure}


\subsubsection{Transformations} \label{sec:graphics_trans}

Transformations in computer graphics are often represented by $4 \times 4$ matrices. These matrices operate on so-called homogeneous coordinates, operating on vectors in $\mathbb{R}^4$ of the form $\left[ x, y, z, w \right]^T$ \cite{shirley2009}. The inclusion of the fourth component enables simultaneous rotation, translation, and scaling of the input points by the matrix, provided that the outputs are normalized to set $w = 1$ \cite{shirley2009}.

There are a few fundamental matrices used in computer graphics that are necessary for later hardware-accelerated light curve simulation algorithms. The first is the model matrix $M \in \mathbb{R}^{4 \times 4}$ which transforms from the world frame to the model frame given the origin of the model body frame $R_{m} \in \mathbb{R}^{3 \times 3}$ and the orientation of the model body frame relative to the world frame $\mathbf{q}_m \in \mathbb{R}^4$ as a quaternion \cite{shirley2009}:

\begin{equation} \label{eq:model_matrix}
  M = \begin{bmatrix}
    -\ q_2^2-\ q_3^2+q_1^2+q_4^2\ &\ 2\ q_1q_2+2\ q_3q_4&\ 2\ q_1q_3-2\ q_2q_4 & -R_{m,x}\\
    \ 2\ q_1q_2-2\ q_3q_4&\ -\ q_1^2-\ q_3^2+q_2^2+q_4^2\ &\ 2\ q_1q_4+2\ q_2q_3 & -R_{m,y}\\
    \ 2\ q_1q_3+2\ q_2q_4&\ 2\ q_2q_3-2\ q_1q_4&\ -q_1^2-\ q_2^2+q_3^2+q_4^2 & -R_{m,z} \\
    0 & 0 & 0 & 1
  \end{bmatrix}.
\end{equation}

Eq \ref{eq:model_matrix} uses Eq \ref{eq:quat2dcm} to transform the quaternion into a DCM. The model matrix basis vectors are illustrated in Figure \ref{fig:ortho_perspective_cameras} as $M_i$, with the world frame basis vectors notated with $W_i$. Given the location of the camera origin $R_{cam} \in \mathbb{R}^3$, its target position $T_{cam} \in \mathbb{R}^3$, and the camera up direction $U_{cam} \in \mathbb{R}^3$, the view matrix $V \in \mathbb{R}^{4 \times 4}$ that transforms from the world frame to the camera frame is given by \cite{shirley2009}:

\begin{align*} \numberthis \label{eq:view_matrix}
  v_3 &= \frac{R_{cam} - T_{cam}}{\| R_{cam} - T_{cam} \|} \\
  v_1 &= \frac{U_{cam} \times v_z}{\| U_{cam} \times v_3 \|} \\
  v_2 &= v_1 \times v_3 \\
  V &= \begin{bmatrix}
    v_{1,x} & v_{1,y} & v_{1,z} & -v_1 \cdot R_{cam} \\
    v_{2,x} & v_{2,y} & v_{2,z} & -v_2 \cdot R_{cam} \\
    v_{3,x} & v_{3,y} & v_{3,z} & -v_3 \cdot R_{cam} \\
    0 & 0 & 0 & 1
  \end{bmatrix}.
\end{align*}

The view matrix basis vectors are illustrated in Figure \ref{fig:ortho_perspective_cameras} as $V_i$. Given the field of view of the camera $FOV$ in radians, the distance from the camera origin to the near $n$ and far $f$ clipping planes, and the camera aspect ratio $a$, the orthographic projection matrix $P \in \mathbb{R}^{4 \times 4}$ that transforms from the camera frame to the image plane is given by \cite{shirley2009}:

\begin{align*} \numberthis \label{eq:projection_matrix}
  t &= n \cdot \tan\left(\frac{FOV}{2}\right) \\
  r &= t \cdot a \\
  P &= \begin{bmatrix}
    \frac{2n}{2r} & 0 & 0 & 0 \\
    0 & \frac{2n}{2t} & 0 & 0 \\
    0 & 0 & - \frac{f+n}{f-n} & \frac{2fn}{f-n} \\
    0 & 0 & -1 & 0 \\
  \end{bmatrix}
\end{align*}

Together, these matrices form the so-called Model-View-Projection matrix which transforms directly from coordinates in the object body frame $r_{obj}$ to the image plane:

\begin{equation} \label{eq:mvp}
  \begin{bmatrix} x_h \\ y_h \\ z_h \\w_h \end{bmatrix} = M V P \begin{bmatrix} r_{obj,x} \\ r_{obj,y} \\ r_{obj,z} \\ 1 \end{bmatrix}.
\end{equation}

In Eq \ref{eq:mvp}, $x_h/w_h$ and $y_h/w_h$ are homogeneous coordinates in the image plane, running linearly from $[-1, -1]$ at the top left corner of the image to $[1, 1]$ at the bottom right. Given the width of the image in $w_{pix}$ pixels, the image coordinates $\left(x,y\right)$ of $r_{obj}$ in pixels are:

\begin{equation} \label{eq:homo_to_pix}
  \begin{bmatrix} x \\ y \end{bmatrix} = \begin{bmatrix} w_{pix} \left(\frac{1}{2} + \frac{x_h}{2w_h}\right) \\ a \cdot w_{pix}\left(\frac{1}{2} + \frac{y_h}{2w_h}\right) \end{bmatrix}.
\end{equation}

The transformation summarized in Eqs \ref{eq:mvp} and \ref{eq:homo_to_pix} are crucial in Section \ref{sec:shadow_mapping} for the shadow mapping algorithm used to compute pixel-wise self-shadowing effects.