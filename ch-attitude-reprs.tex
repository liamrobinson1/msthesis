\ProvidesFile{ch-attitude-reprs.tex}[Attitude]

\chapter{Attitude}

\section{Attitude Representations}

When discussing about the orientation of a rigid body in three dimensions, otherwise known as its attitude, that orientation is implicitly understood to be relative to some other reference frame. The direction of a unit vector can be expressed with two numbers ---  the azimuth and elevation of that vector. Naïvely, this could be extrapolated to conclude that six numbers are needed to express an orientation. Because the basis vectors form an orthonormal set $\left\{ \hat{b}_1, \hat{b}_2, \hat{b}_3\right\}$, it follows for a right-handed system that $\hat{b}_3 = \hat{b}_1 \times \hat{b}_2$, $\hat{b}_2 = \hat{b}_3 \times \hat{b}_1$, and $\hat{b}_1 = \hat{b}_2 \times \hat{b}_3$. Each of these equations constrains one further degree of freedom, revealing that a minimum of three quantities are necessary to express the relative orientation of two reference frames. This minimum bound does not make any statements about the usefulness of three element sets; at least four dimensions are needed to remove singularities. 

\subsection{The Direction Cosine Matrix}

The direction cosine matrix (DCM) is a $3\times3$ symmetric, orthogonal matrix, expressing the three basis vectors of one frame in another. This amounts to projecting each basis vector in the initial frame onto each basis vector of the final frame; the cosine of the angle between the compared vectors. It is notated with two capital letters, the rightmost indicating the reference frame of the input vectors and the leftmost indicating the transformed frame. Alternatively, the DCM is sometimes expressed as $C$ when the frames involved are arbitrary or do not need to be denoted. For example, the DCM $\dcm{bn}$ takes vectors in the $\rf{n}$ frame to the $\rf{b}$ frame:

\begin{equation}
    \vrf{b}{r} = \dcm{bn} \vrf{n}{r}
\end{equation}

The orthogonal property of the DCM implies $\dcm{bn}^{-1} = \dcm{bn}^T$ such that $\dcm{bn}^T = \dcm{nb}$. 

\subsection{Principal Rotation Parameters}

Another common attitude representation is the Euler angle-axis set, otherwise known as principal rotation parameters \cite{crassidis1ed}. Euler's rotation theorem guarantees that any relative orientation can be expressed as a single rotation about an axis $\hat{\lambda} \in \mathbb{S}^2$ by an angle $\theta \in [0, 2\pi]$ \cite{crassidis1ed}. The set $\left\{\hat{\lambda},\theta\right\}$ is known as a principal rotation parameter, abbreviated PRP hereafter. The DCM is mapped to the PRP representation via \ref{eq:dcm_to_prp} \cite{shuster1993}.

\begin{align*} \numberthis \labelAndRemember{eq:dcm_to_prp}
    {
    \theta &= \cos^{-1}\left(\frac{1}{2} \left[C_{1,1} + C_{2,2} + C_{3,3} - 1 \right] \right) \\
    \hat{\lambda} &= \frac{1}{2\sin{\theta}} 
    \begin{bmatrix} C_{2,3} - C_{3,2} \\ C_{3,1}-C_{1,3} \\ C_{1,2} - C_{2,1}\end{bmatrix}
    }
\end{align*}

Where $C_{i,j}$ refers to the $i$th row and $j$th column of $C$. The mapping from PRP to DCM is also relatively straightforward.

\begin{equation} \labelAndRemember{eq:prp2dcm}
    {C = I_3 + \sin\theta\matcp{\hat{\lambda}} + (1-\cos\theta)\matcp{\hat{\lambda}}^2}
\end{equation}

Where $\matcp{v}$ is the matrix cross product operator, defined on $\vctr{v} \in \mathbb{R}^3$ as:

\begin{equation}
    \matcp{\vctr{v}} = \begin{bmatrix}
        0 & -v_3 & v_2 \\
        v_3 & 0 & -v_1 \\
        -v_2 & v_1 & 0
    \end{bmatrix}.
\end{equation}

This operator is useful as it rephrases the cross product as matrix multiplication, i.e. $\vctr{v} \times \vctr{u} = \matcp{\vctr{v}}\vctr{u}$. While the PRP $\{\theta, \hat{\lambda}\}$ is a four element set, there are only three degrees of freedom due to the unit norm constraint on $\hat{\lambda}$. 

\subsection{Quaternions}

The quaternion represents attitude with a point on the surface of the hypersphere \sthree. In terms of the PRP, the quaternion is given by Eq \ref{eq:prp2quat} \cite{crassidis1ed}.

\begin{equation} \labelAndRemember{eq:prp2quat}
    {
    \vctr{q} = \begin{bmatrix} \hat{\lambda} \sin\left( \theta \right) \\ \cos(\theta) \end{bmatrix}
    }
\end{equation}

The first three entries of the quaternion are often called the vector component, with the final entry being the scalar component. Some authors reorder the quaternion, placing the scalar term first. Often the entries of a single quaternion are referenced by index such that $\vctr{q} = \left[ q_1, q_2, q_3, q_4 \right]$. Similarly, the vector portion of the quaterion is referenced with $\vctr{q}_{1:3}$. The quaternion can be mapped back to the PRP \cite{crassidis1ed} via

\begin{align*} \numberthis \labelAndRemember{eq:quat2prp_theta} 
    {
    \theta &= \cos^{-1}\left(q_4 \right) \\
    \lambda &= \frac{\vctr{q}_{1:3}}{\sin{\theta}}.
    }
\end{align*}

The quaternion maps to the DCM \cite{crassidis1ed} via 

\begin{equation} \labelAndRemember{eq:quat2dcm}
    {
        C = \left[\begin{matrix}\ -\ q_2^2-\ q_3^2+q_1^2+q_4^2\ &\ 2\ q_1q_2+2\ q_3q_4&\ 2\ q_1q_3-2\ q_2q_4\\\ 2\ q_1q_2-2\ q_3q_4&\ -\ q_1^2-\ q_3^2+q_2^2+q_4^2\ &\ 2\ q_1q_4+2\ q_2q_3\\\ 2\ q_1q_3+2\ q_2q_4&\ 2\ q_2q_3-2\ q_1q_4&\ -q_1^2-\ q_2^2+q_3^2+q_4^2\\\end{matrix}\right]=\Xi\left(q\right)^T\Psi\left(q\right).
    }
\end{equation}

In Eq \ref{eq:quat2dcm}, $\Psi$ is defined to be \cite{crassidis1ed}

\begin{equation} \label{eq:quat_psi}
    \Psi = \left[\begin{matrix}q_4&q_3&-q_2\\{-q}_3&q_4&q_1\\q_2&-q_1&q_4\\-q_1&-q_2&-q_3\\\end{matrix}\right].
\end{equation}

\section{Quaternion Attitude Kinematics}

Because it is cheap to convert between attitude representations, only one set of kinematic equations are needed for propagating a rigid body attitude profile from an initial condition. Quaternion kinematic differential equations are chosen as they have no singularity and produce very smooth dynamics that are easy to integrate when compared to three-variable representations that possess singularities. Given the current orientation quaternion $\vctr{q} = [q_1, q_2, q_3, q_4]^T$ and angular velocity $\vctr{\omega} = [\omega_1, \omega_2, \omega_3]^T$ the quaternion derivative is computed via Eq \ref{eq:quat_kde}

\begin{equation} \label{eq:quat_kde}
    \left[\begin{matrix}\dot{\epsilon_1}\\\dot{\epsilon_2}\\\dot{\epsilon_3}\\\dot{\epsilon_4}\\\end{matrix}\right]
    =
    \frac{1}{2}\left[\begin{matrix}\epsilon_4&-\epsilon_3&\epsilon_2&\epsilon_1\\\epsilon_3&\epsilon_4&-\epsilon_1&\epsilon_2\\-\epsilon_2&\epsilon_1&\epsilon_4&\epsilon_3\\-\epsilon_1&-\epsilon_2&-\epsilon_3&\epsilon_4\\\end{matrix}\right]
    \left[\begin{matrix}\omega_1\\\omega_2\\\omega_3\\0\\\end{matrix}\right].
\end{equation} 

\section{Quaternion Attitude Dynamics}

Rigid body dynamics can be easily expressed in the body principal axes with an arbitrary torque $\vctr{M} = \left[M_1, M_2, M_3\right]^T$ in the same frame via Eq \ref{eq:rbtf_dynamics}

\begin{equation} \label{eq:rbtf_dynamics}
    \left[\begin{matrix}\dot{\omega_1}\\\dot{\omega_2}\\\dot{\omega_3}\\\end{matrix}\right]
    =
    \left[\begin{matrix}
        \left(M_1+I_2\omega_2\omega_3-I_3\omega_2\omega_3\right) / I_1 \\
        \left(M_2-I_1\omega_1\omega_3+I_3\omega_1\omega_3\right) / I_2 \\
        \left(M_3+I_1\omega_1\omega_2-I_2\omega_1\omega_2\right) / I_3 \\
    \end{matrix}\right].
\end{equation}

Equations \ref{eq:rbtf_dynamics} and \ref{eq:quat_kde} are numerically integrated to yield the orientation time history which is necessary for later light curve simulations.