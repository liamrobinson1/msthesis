\ProvidesFile{ch-appendices.tex}[Appendices]
\chapter{Appendices}

\section{Shadow Mapping} \label{data:shadow_mapping}

\subsection{Shading} \label{data:shading}

\begin{algorithm}
  \caption{Pixel-wise shading algorithm with shadow mapping}\label{alg:pix_shading}
  \begin{algorithmic}
    \State $L \in \mathbb{S}^2$ \Comment{Unit vector from object origin towards Sun}
    \State $O \in \mathbb{S}^2$ \Comment{Unit vector from object origin towards observer}
    \State $N \in \mathbb{S}^2$ \Comment{Outward-pointing surface normal vector at pixel coordinates}
    \State $(C_d, C_s, n) \in \mathbb{S}^2$ \Comment{Reflection coefficients and exponent for the BRDF}
    \Require $C_d + C_s \leq 1$ \Comment{Enforce energy conservation}
    \State $MVP_{Sun} \in \mathbb{R}^{4 \times 4}$ \Comment{MVP matrix for the Sun camera}
    \State $(x, y) \in \mathbb{Z}^2$ \Comment{Integer pixel coordinates from the observer camera}
    \State $R_{pix,obs} \in \mathbb{R}^3$ \Comment{World coordinates of the pixel; provided by OpenGL}
    \State $\left[(x_{homo,Sun}, y_{homo,Sun}, ...\right] \gets MVP_{Sun} \left[ R_{pix,obs}, 1 \right]^T$
    \State $x_{Sun} \gets \left(1 + p_{x, homo}\right) \frac{w_{pix}}{2} $ \Comment{Homogeneous coordinates from the Sun camera}
    \State $y_{Sun} \gets \left(1 + p_{y, homo}\right) \frac{a \cdot w_{pix}}{2} $
    \State $D_{Sun} \gets  d(x_{sun}, y_{sun})$ \Comment{Closest pixel depth to the Sun direction}
    \State $D_{obs} \gets \left( L - O \right) \cdot L$ \Comment{Closest pixel depth in the Observer direction}
    \If{$D_{obs} > D_{Sun}$}
      \State $\delta_{ss} = 1$ \Comment{Pixel is self-shadowed}
    \Else 
      \State $\delta_{ss} = 0$ \Comment{Pixel may be illuminated}
    \EndIf
    \If{$\left(N \cdot L\right) > 0 \textrm{ and } \left(N \cdot O\right) > 0$}
      \State $f_r(\vctr{x}, L \rightarrow O) = 0$ \Comment{Pixel cannot be both observed and illuminated}
    \Else 
      \State $f_r(\vctr{x}, L \rightarrow O) = \mathrm{Phong}(L, O, N, C_d, C_s, n)$ \Comment{The pixel is shaded with the BRDF}
    \EndIf
    \State $\mathrm{IM}(x, y) = f_r(\vctr{x}, L \rightarrow O) \left(N \cdot L\right) $ \Comment{Image pixel value}
  \end{algorithmic}
\end{algorithm}

\section{Astronomical Spectra Data} \label{data:spectra}

\subsubsection{Atmospheric Extinction} \label{data:atm}
The atmospheric extinction coefficient is dimensionless.
\begin{listing}[H]
\inputminted[breaklines=true, breakanywhere=true, breaksymbol=\hspace{0pt}, fontsize=\footnotesize]{json}{/Users/liamrobinson/Documents/PyLightCurves/mirage/resources/data/atmos_extinction.json}
\end{listing}

% \subsubsection{Quantum Efficiency} \label{data:qe}
% The quantum efficiency spectrum has units $\left[ \frac{e^-}{m} \right]$.
% \begin{listing}[H]
% \inputminted[breaklines=true, breakanywhere=true, breaksymbol=\hspace{0pt}, fontsize=\footnotesize]{json}{/Users/liamrobinson/Documents/PyLightCurves/mirage/resources/data/kaf16803_quantum_efficiency.dat}
% \end{listing}

\subsection{Background Source Data}

\subsubsection{Lunar Phase Factor}
The lunar phase factor is a function of the phase angle in radians and is dimensionless.
\begin{listing}[H]
\inputminted[breaklines=true, breakanywhere=true, breaksymbol=\hspace{0pt}, fontsize=\footnotesize]{json}{/Users/liamrobinson/Documents/PyLightCurves/mirage/resources/data/lunar_phase.json}
\end{listing}

\subsubsection{Scattered Moonlight}
The scattered moonlight radiance in $\left[ \frac{W}{sr \cdot m^2 \cdot m} \right]$ is a function of the difference in the line of sight and Moon azimuths \texttt{delta\_az} in radians, the zenith angle of the moon \texttt{z\_moon} in radians, and the zenith angle of the line of sight \texttt{z\_obs} in radians.

\begin{listing}[H]
\inputminted[breaklines=true, breakanywhere=true, breaksymbol=\hspace{0pt}, fontsize=\footnotesize]{json}{/Users/liamrobinson/Documents/PyLightCurves/mirage/resources/data/moonlight.json}
\end{listing}

\subsubsection{Zodiacal Light} \label{data:roach_zod}

The zodiacal light surface brightness in $S_10$ is a function of the latitude \texttt{"ecliptic\_lat"} and longitude \texttt{"ecliptic\_lon} of the line of sight in the solar system ecliptic reference frame, both expressed in radians.

\begin{listing}[H]
\inputminted[breaklines=true, breakanywhere=true, breaksymbol=\hspace{0pt}, fontsize=\footnotesize]{json}{/Users/liamrobinson/Documents/PyLightCurves/mirage/resources/data/zodiacal.json}
\end{listing}

\subsection{Telescope Parameters}

\subsubsection{Purdue Optical Ground Station}

\begin{table}[ht] \label{tb:pogs_parameters}
    \centering
    \begin{tabular}{|l|l|}
    \hline
    \textbf{Parameter} & \textbf{Value} \\ \hline
    FWHM                & $1.5$                              \\ \hline
    Sensor dimensions    & $ 0.03690 \times 0.03690 \: [m]$                               \\ \hline
    $f$ number   & $7.2$                              \\ \hline
    Aperture diameter $D$       & $0.35560 \: [m]$                              \\ \hline
    Secondary diameter         & $0.1724660 \: [m]$                              \\ \hline
    Sensor pixels               & $4096 \times 4096$                              \\ \hline
    Pixel size               & $9.009 \cdot 10^{-6} \: [m / \textrm{pix}]$                              \\ \hline
    Pixel scale $s_{pix}$              & $0.72545 \: [arcsec]$                              \\ \hline
    Field of view               & $0.824425^\circ \times 0.824425^\circ$                              \\ \hline
    Integration time $\Delta t$              & $10 \: [s]$                              \\ \hline
    Integration dark noise $\lambda_{dark}$ & $3 \: \left[ e^- / \mathrm{pix} / s\right]$ \\ \hline
    Read noise variance $\sigma_{read}^2$ & $9 \: \left[ e^- \right]$ \\ \hline
    CCD gain $g$ & $1$ \\ \hline
    \end{tabular}
    \caption{Purdue Optical Ground Station telescope parameters}
  \end{table}


\subsection{File Formats}

\subsubsection{Wavefront OBJ Example} \label{sec:obj_listing}

\begin{listing}[ht]
    \inputminted[breaklines=true, breakanywhere=true, breaksymbol=\hspace{0pt}, fontsize=\footnotesize]{text}{/Users/liamrobinson/Documents/PyLightCurves/mirage/resources/models/cube.obj}
\end{listing}