\ProvidesFile{ch-introduction.tex}[Introduction]
\graphicspath{{/Users/liamrobinson/Documents/PyLightCurves/docs/build/html/_images}}

\chapter{Introduction}

Humankind has been creating space debris since the dawn of the space age \cite{esareport2022}. Early missions like Vanguard 1, launched in March of 1958, set a precedent by leaving both their satellite and the launch vehicle's upper stage in orbit, both of which are still in orbit in 2023 \cite{vanguard1}. Vanguard was launched into Low Earth Orbit (LEO) --- defined by the European Space Agency (ESA) as any orbit with an altitude below $2000$ kilometers \cite{esareport2022}. Above LEO lies Medium Earth Orbit (MEO) for orbital altitudes between $2000$ and $31570$ kilometers, and Geostationary Earth Orbit (GEO) between $35586$ and $35986$ kilometers altitude \cite{esareport2022}. Half a century of increasingly frequent launches has created a space environment cluttered with thousands of debris objects, increasing the number of serious conjunction events that may require avoidance maneuvers for large satellites in LEO to over $100$ during 2021 \cite{esareport2022}. The rapid changes in the space environment over the last two decades is illustrated in Figure \ref{fig:catalog_comparison}. While very few mission-ending collisions are occurring on a yearly basis in the early 2020s, simulations predict over 200 catastrophic collisions per year if the trend in new launches and disposal practices continue \cite{esareport2022}. This uncontrolled proliferation of human-made space debris puts space operations at risk. High-profile satellite collisions like Iridium-Cosmos in 2009 have added fuel to the fire, producing shells of debris that further pollute LEO \cite{vallado4ed}. Anti-satellite tests carried out by the USA, Russia, China, and India since the 1960s see nations destroying their own satellites, projecting military strength at the cost of creating more debris \cite{vallado4ed}. Beyond LEO in Geostationary Transfer Orbit (GTO), exploding launch vehicle upper stages produce large amounts of debris \cite{esareport2022}. While higher orbits are not yet as polluted as LEO, they do not decay due to atmospheric drag, allowing debris objects to remain in the environment indefinitely \cite{vallado4ed}.

\begin{figure}[ht]
    \centering
    \includegraphics[width=\figbig]{sphx_glr_propagate_catalog_001.png}
    \caption{Public tracked object catalog within $5 \cdot 10^4$ kilometers of Earth in 2000 and 2023}
    \label{fig:catalog_comparison}
\end{figure}

In the context of the modern space environment, determining the current state and predicting the future dynamics of space objects is critical for many areas of Space Domain Awareness (SDA) \cite{frueh2019notes}. While the current orbits of objects can be determined accurately from astrometry --- through passive optical imagery or active radar --- their future dynamics are perturbed by non-conservative forces driven by their shape, attitude profile, and material properties that cannot be observed directly. In particular, objects in orbits with altitudes higher than LEO are most efficiently observed with optical telescopes as the power required for radar scales with the square of the distance \cite{frueh2019notes}. Because optical observations are already commonly used to characterize the orbit of the objects in orbits past LEO, it is advantageous to use the same existing sensors --- or in some cases even the same images --- to extract these other useful characteristics. 

Characterizing an object's shape, attitude, and material properties is fundamentally difficult as distance from the sensor and atmospheric turbulence leaves only a distribution of brightness in the image \cite{fan2020thesis}. One part of the remaining information is the total brightness of the object, and this value over time is known as the light curve. Optical brightness observations are fundamentally limited by background and sensor noise \cite{frueh2019notes}. Light curves are computed from observed optical data by estimating the background mean of the image, identifying which pixels of the image likely belong to the object, subtracting the mean background level from those pixels, and calibrating the remaining object signal using known stars elsewhere in the image \cite{schildknecht2008}. Each image must also be monitored for contamination from background stars and over- and under-exposures \cite{schildknecht2015}. Despite these realities, the light curve is a function of the parameters of interest: the object's shape, attitude, and material properties \cite{fan2020thesis, burton2021mapping}. Solving light curve shape inversion in a general case would enable robust active debris removal, anomaly detection, and collision avoidance, all of which are benefitted by accurate shape information.

Due to the environmental noise and fundamental physical limitations on the processes driving the light curve, the measured brightness is dependent on the overall brightness and hence varies from data point to data point in the light curve. Furthermore, given the Poisson nature of the light collection process, a constant Gaussian assumption of the measurement noise in the light curve may not be suitable \cite{fan2020thesis, krag2003}. A realistic representation of the light curve can only be achieved by accounting for the physical processes simulating the lighting and measurement process, followed by the measurement reduction and correction processing steps. 

\section{State of the Art}

The state of the art in light curve simulation differs between approaches and the object class under study. Kaasalainen and Torppa, as well Fan, Friedman, Kobayashi, and Frueh employ a Lambertian model for convex objects with a facetwise ray tracing scheme for nonconvex objects \cite{kaasalainen2001, fan2016, fan2020thesis,friedman2020,kobayashi2020,frueh2014}. This approach has the advantage of being simple, but can be computationally intensive for complex objects. Furfaro et al.\ \cite{furfaro2019} and Cabrera and Bradley \cite{cabrera2021,bradley2014} use a simple Lambertian model with no self-shadowing. Many more authors apply a more specialized non-Lambertian Bidirectional Reflectance Distribution Function (BRDF) for their lighting \cite{linares2018space, mcnally2021, blacketer2022}. Few papers in the literature focus solely on light curve simulation, most treating simulation as a background consideration for the real goal of inversion. In the minority, Allworth et al.\ developed a ray traced light curve simulator based on Blender's cycles renderer, allowing them to account for photorealistic shadowing and motion blur \cite{allworth2020, allworth2021}. Often, advances in light curve simulation are driven by developments in the field of computer graphics. The ray tracing schemes rely on work by Möller and Trumbore \cite{moller2005}, while the more complex reflection functions used to simulate light transport are all contributions from the computer graphics community \cite{phong1975, ashikhmin2000, cook1982, oren1994, matusik2003}. Throughout the literature, there is a clear gap between the simulated light curves and their observed counterparts. Due to the difference in quality, authors often treat real and simulated data very differently \cite{allworth2021}. This work presents a physically-based lighting, shadowing, and noise model to produce synthetic light curves of similar quality to observed data, enabling more robust validation of the presented inversion techniques.

Light curve shape inversion was first investigated by Russell in 1906, who proposed a spherical harmonic representation that could be fit to an asteroid shape \cite{russell1906}. Russell noted that there would be ambiguity in the shape solution such that many solutions would fit the data equally well. The next major contribution to the field was due to Kaasalainen and Torppa in 2000, who successfully reconstructed the shapes of asteriods by directly optimizing the directions and areas of candidate faces --- encapsulated by the so-called Extended Gaussian Image (EGI) --- to find a convex shape that produces a similar light curve \cite{kaasalainen2000, kaasalainen2001}. Once the EGI is estimated, Kaasalainen and Torppa recover the vertices and faces of the corresponding convex object using a result of Minkowski and a nonlinear, convex optimization problem implemented by Little \cite{minkowski1909, little1983}. Any EGI-based method in the asteroid or human-made object shape inversion literature uses some variation of this final stage to reconstruct the final estimated geometry. Kaasalainen and Torppa also addressed nonconvex shape inversion by optimizing a spherical harmonics shape representation to reconstruct the largest nonconvex features of an asteroid, noting that smoothness regularization was sometimes needed to prevent the shape from degenerating \cite{kaasalainen2000}. In the work of Kaasalainen and Torppa, the EGI optimization takes place in a single step as the asteroid shapes under study do not have sparse EGIs. By contrast, this work introduces stages that increase shape accuracy and lower computation time by leveraging the natural sparsity of human-made objects. Durech and Kaasalainen extended on this work in 2003 by investigating the observability of nonconvex features in asteroid light curves, finding that concave features are often observable only at high phase angles, supporting the conclusion that robust nonconvex shape inversion requires very different considerations than its convex counterpart \cite{durech2003}. In 2022, Chng et al.\ proposed a method to determine an optimal spin pole and convex shape via the EGI, offering computational benefits over Kaasalainen and Torppa while guaranteeing global optimality in the solution with respect to the input brightness data, while being limited to convex shape estimates\cite{chng2022}. Using the methods originally proposed by Kaasalainen and Torppa, a collaborative effort of dozens of observatories lead to the publication of Database of Asteroid Models from Inversion Techniques (DAMIT), a publicly-available repository of convex asteroid models \cite{durech2010}. As of October 2023, DAMIT currently hosts 16,086 models for 10,751 asteroids \cite{damit2014}.

% Viikinkoski et al.\ investigated recovering large concavities from adaptive optics imagery, noting the fundamental non-uniqueness of any solution \cite{viikinkoski2017}. They discuss how a single large concavity may produce identical scattering behavior to multiple smaller concave features \cite{viikinkoski2017}. # removed due to AO not truly unresolved.  While the field of asteroid shape inversion has been alive in the intervening years, most works are not relevant to human-made objects.

Shape inversion for human-made space objects differs from the asteroid inversion in a few important aspects. More diverse methods exist, being generally segmented into EGI-based methods drawing from the asteroid shape inversion literature, filter-based methods for simultaneous attitude and shape solutions, and machine learning for classifying object shape from the light curve. Due to the increased number of unknowns in the material properties and attitude profile when observing an arbitrary human-made object, the inverted light curves are often simulated as part of the same work. This highlights the importance of realistic light curve simulation to effectively test proposed inversion methods. 

Direct shape inversion for human-made space objects was first investigated by Calef et al.\, who adopted Kaasalainen and Torppa's methods applied to multispectrum measurements to reduce the ambiguities of the different material properties common in human-made objects \cite{calef2006photometric}. Bradley and Axelrad also used asteroid inversion techniques to recover convex approximations of CubeSats, rocket bodies, and box-wing satellites using the inversion codes developed and released by Kaasalainen, yielding good results for rocket body-like shapes but limited success for box-wing satellites and other high area-to-mass ratio (HAMR) objects \cite{bradley2014}. The most recent major contributions to the direct shape inversion literature are due to Fan and Frueh, who inverted the shape of convex human-made objects from noisy light curves using the EGI with a multi-hypothesis scheme to reduce the ambiguity introduced by noisy measurements \cite{fan2019, fan2020thesis, fan2021}. Fan notes that full observability is crucial for successful direct shape inversion, pointing to work by Friedman and Frueh, who quantified the observability of EGI inversion to inform sensor tasking schemes \cite{friedman2020, friedman2022}. Cabrera et al.\ applied area regularization to Fan and Friedman's methods, achieving more accurate convex shape estimates and finding that natural constraints on the EGI area optimization renders the problem estimatable before it becomes classically observable \cite{cabrera2021}. 

Throughout the shape inversion literature, two themes are clear. Effective and efficient methods for nonconvex shape inversion for human-made objects are needed, and existing convex inversion methods have not been designed to work with realistic measurement noise. This work seeks to address both of these challenges by presenting a method for inverting large singular concave features in addition to a scheme for robustly inverting convex and nonconvex shapes with physically-based measurement noise.

Outside of the asteroid-inspired EGI methods, the literature falls into two broad categories: filter-based inversion and machine learning categorization. Each offers different advantages while imposing unique limitations. Filter-based shape inversion was primarily pioneered by Linares through work with various co-authors. These filter-based methods often seek to perform multiple types of object characterization simultaneously, estimating attitude and material properties in addition to shape \cite{linares2012, linares2014space, linares2018space}. Because the input data for filter-based approaches is still only unresolved brightness measurements, estimating more properties in an already ill-posed problem requires a loss of fidelity in the solution elsewhere. Often, the shape model is highly simplified to make the problem more tractable \cite{linares2012, linares2014space, linares2018space}. Linares et al.\ have implemented unscented Kalman filters \cite{linares2012}, multiple-model adaptive estimation algorithms \cite{linares2014space}, and adaptive Hamiltonian Markov chain Monte Carlo schemes \cite{linares2018space} which achieve good results for simple shapes, but have not been tested on complex and realistic geometries \cite{linares2018space}. In general, filter-based approaches are limited by the nonlinearity of highly specular and complex human-made objects, but require less information to run. Direct inversion methods that use the EGI require more \textit{a priori} information, but are able to deliver more accurate shape estimates.

By contrast, machine learning categorization methods indirectly recover shape information by predicting which class of objects an observed light curve belongs to. Linares and Furfaro used a deep convolutional neural network to classify novel light curves as rocket bodies, payloads, or debris, achieving good classification accuracy at the cost of uncertainty about how the model would behave for light curves collected for objects outside of its training dataset \cite{linares2016}. Other authors, including Kerr et al.\ and McNally et al.\ have adapted the architecture developed by Furfaro et al.\ to classify novel light curves into an extended set of object types, demonstrating that these models are flexible enough to differentiate between many object types and attitude profiles, although with higher error rates \cite{kerr2021, mcnally2021}. Allworth et al.\ applied transfer learning to classify real measurements using a synthetically-trained model, supporting the applicability of these approaches to operational decision-making \cite{allworth2021}.

There has also been significant work published on extracting light curves of human-made objects from real optical observations. Schildknecht et al.\ used color photometry to investigate isolate material properties of high area-to-mass ratio (HAMR) objects in GEO \cite{schildknecht2008}. Karpov et al.\ used wide-field monitoring system to collect light curves from LEO objects \cite{karpov2016}. Benson et al.\ collected light curves from retired GOES, Inmarsat, and Astra satellites in geosynchronous orbit to characterize their spin states \cite{benson2017}. Koshkin et al.\ collected light curves of TOPEX/Poseidon, among other inactive satellites, to determine their spin poles and rates \cite{koshkin2018}. Wang et al.\ collected light curves from GOES-8, an active GEO satellite, and simulated material properties and attitude profile to attribute peaks in its observed brightness to different parts of the spacecraft \cite{wang2018}.

\section{Research Questions and Contributions}

\noindent This work seeks to address three major gaps in the state of the art.

\begin{enumerate}
    \item Light curve simulation methods do not model much of the physics of the observation, leading to noiseless measurements or noisy values that do not match the real-world processes.
    \item Direct shape inversion for human-made objects is limited to convex objects, a problem given that most human-made space objects are highly nonconvex.
    \item Noisy measurements degrade the performance of standard shape inversion procedures. Existing noisy shape inversion procedures assume convexity in the truth object and a constant noise level, two assumptions that often do not hold in reality. 
\end{enumerate}

\noindent These gaps lead to three research questions that guided this work.

\begin{enumerate}
    \item What physical processes influence the mean and variance of the light curve, and how can these processes be efficiently modeled?
    \item How can concave features in the object's geometry be identified from the light curve and introduced into the shape estimate?
    \item What effect does noise have on the shape guesses produced by standard algorithms, and how information from multiple inaccurate shape estimates be combined into a better final estimate?
\end{enumerate}

\noindent This work presents three contributions, aligned with the identified research questions.

\begin{enumerate}
    \item A high-fidelity light curve simulator was developed, efficiently modeling self-shadowing, background environmental effects, and sensor effects to produce light curves with or without realistic noise.
    \item Improvements to the convex shape inversion procedure through the addition of resampling and merging stages produce sparser geometry in less computation time. A new method for estimating large and unilateral concavities is also developed which imposes no new assumptions on the convex inversion scheme.
    \item For noisy shape inversion, a novel method for estimating the uncertainty in the estimated geometry is developed, leading to a weighted averaging procedure to combine the most reliable features of multiple shape estimates.
\end{enumerate}

% \section{Contributions}

% \subsection{Simulation Advances}

% . This simulator is one to four orders of magnitude faster than ray tracing-based renderers commonly used in literature \cite{fan2019, allworth2020}. It supports self-shadowing, variable material properties, a variety of reflection functions, and dynamic solar panel rotation. In concert with a constrained observer model and orbit propabation, the simulator generates realistic, noisy light curves for inactive debris, highly nonconvex objects, and actively-controlled satellites. 

% \subsection{Advances in Convex Shape Inversion}

% This work presents a suite of changes that build on the classical shape inversion algorithm for convex shapes \cite{robinson2022}. New resampling and merging steps in the Extended Gaussian Image optimization stage yield more accurate shapes that are easier to reconstruct. An alternative optimization method for the shape support vector decreases convergence time for highly symmetric objects where the classical optimization algorithm fails.

% The approach presented in this work solves the shape inversion problem beginning from the direct geometry reconstruction methods of \cite{kaasalainen2001,fan2020thesis}. The EGI optimization processes of \cite{fan2020thesis,cabrera2021,kaasalainen2001} are improved using novel resampling and merging steps. These improvements circumvent the need for the regularization terms explored by Cabrera et al.\ \cite{cabrera2021}.

% This convex shape inversion method has a number of general advantages. It does not require any \textit{a priori} information about the truth geometry. Unlike MMAE methods \cite{linares2014space}, the presented algorithm does not require a bank of reference models to recover shape information. Unlike deep learning methods, the presented method does not require a diverse set of training data to achieve good results \cite{furfaro2019,kerr2021}.

% \subsection{Advances in nonconvex Shape Inversion}

% While natural space objects like asteroids are largely convex, nearly all human-made space objects are highly nonconvex, highlighting the need for a robust inversion scheme for nonconvex space objects. If the object's material properties are well-known, the algorithm developed in this work is able to predict whether any single nonconvex feature is self-shadowed in the observed light curve and introduces a concavity in the shape estimate in the correct location.