\ProvidesFile{ch-introduction.tex}[Introduction]
\graphicspath{{/Users/liamrobinson/Documents/PyLightCurves/docs/build/html/_images}}

\chapter{Introduction}

\section{Problem Statement}

Humanity has been creating space debris since the dawn of the space age. Early missions like Vanguard 1 set a precedent by leaving both the satellite and the launch vehicle upper stage in orbit, of which are still in orbit in 2023 \cite{vanguard1}. Half a century of increasingly frequent launches has created a space environment cluttered with thousands of debris objects, requiring active satellites in low Earth orbit (LEO) to perform regular avoidance maneuvers. This uncontrolled proliferation of human-made space debris puts space operations at risk. High-profile satellite collisions like Iridium-Cosmos in 2009 have added fuel to the fire, producing shells of debris that further pollute LEO \cite{vallado4ed}. Anti-satellite (ASAT) tests carried out by the USA, Russia, China, and India since the 1960s see nations destroying their own satellites, projecting military strength at the cost of creating more debris \cite{vallado4ed}.

\begin{figure}[ht]
    \centering
    \includegraphics[width=\figbig]{sphx_glr_propagate_catalog_001.png}
    \caption{Public tracked catalog in 2000 and 2023}
    \label{fig:catalog_comparison}
\end{figure}

Determining the current state and predicting the future dynamics of space objects is critical for many fields within Space Domain Awareness (SDA) \cite{frueh2019notes}. High-fidelity orbital propagation, collision avoidance, and fragmentation analysis all rely to accurate object characterization. Characterizing an object refers to estimating one or multiple unknown relevant properties. Estimating the shape of an object helps analysts characterize it, but doing so is difficult as distance and atmospheric turbulence prevents direct imaging \cite{fan2020thesis}. As a result, passive techniques for object characterization often rely on light curves --— optical brightness measurements collected over time. Light curves are particularly efficient for the task as they are inexpensive to collect and contain information about the shape, orientation, and material properties of the object that produced them \cite{fan2020thesis, burton2021mapping}. Solving light curve inversion in a general case would enable robust active debris removal, anomaly detection, and collision avoidance, all of which rely on accurate shape information.

\section{State of the Art}

Light curve shape inversion was first investigated by Russell in 1906, who proposed a spherical harmonic shape representation that could be fit to an asteroid shape \cite{russell1906}. Russell was generally skeptical of the applicability of the method, hypothesizing that there would be ambiguity in the convex shape such that many solutions would fit the data equally well. After nearly a century of relative inactivity in the field, Kaasalainen and Torppa began successfully reconstructing the shapes of asteriods by directly optimizing the directions and areas of candidate faces to produce a convex shape that produces a similar light curve \cite{kaasalainen2000, kaasalainen2001}. Kaasalainen and Torppa also addressed non-convex shape inversion, developing an optimization procedure which performed well at reconstructed the largest nonconvex features \cite{kaasalainen2000}. The authors noted that this method was highly computationally demanding. The work of Kaasalainen and Torppa was extended by Durech and Kaasalainen in 2003, who investigated the observability of nonconvex features in asteroid light curves \cite{durech2003}. The authors noted that many nonconvex features are only observable at high phase angles, where self-shadowing effects become pronounced in the light curve. They further note that uncertainty in the surface optical properties generally means that small local features cannot be recovered through light curve inversion \cite{durech2003}. Using the methods originally proposed by Kaasalainen and Torppa, a collaborative effort between observatories lead to the publication of Database of Asteroid Models from Inversion Techniques (DAMIT), a publicly-available repository of convex asteroid models \cite{damit2014}. In 2022, Chng et al.~ investigated the 

Papers in the literature usually propose methods for both light curve simulation and shape inversion. Often, the simulation method

Light curve simulation methods differ between approaches and the object class under study. Kaasalainen and Torppa employ a Lambertian model for convex objects with a facetwise ray tracing scheme for non-convex objects \cite{kaasalainen2001}. Fan, Friedman, Kobayashi, and Frueh use a nearly identical scheme for human-made objects \cite{fan2016, fan2020thesis,friedman2020,kobayashi2020,frueh2014}. Allworth et al. developed a ray traced simulator for light curves in Blender, accounting for photorealistic shadowing and motion blur \cite{allworth2020, allworth2021}. Many deep learning approaches including Furfaro et al. \cite{furfaro2019} and Cabrera and Bradley \cite{cabrera2021,bradley2014} use a simple Lambertian model with no self-shadowing. Linares and Crassidis apply a more specialized approach with a non-Lambertian Bidirectional Reflectance Distribution Function (BRDF) for lighting \cite{linares2018space}. McNally et al. use a Phong BRDF without shadowing shadowing, citing computational intensity \cite{mcnally2021}. Blacketer implemented a Cook-Torrance BRDF for lighting with a plane stacking method for self-shadowing \cite{blacketer2022}.

Methods for shape inversion fall into three major categories: Extended Gaussian Image (EGI), statistical estimation, and deep learning based methods, each approaching the problem from a different perspective.

Direct light curve inversion with the EGI uses a series of optimization problems to fit a convex shape to measurements. These methods were pioneered by Kaasalainen and Torppa for asteroids in \cite{kaas2001shape} with simultaneous attitude inversion in \cite{kaasalainen2001}. The work of Kaasalainen et al. on asteroids was extended by Chng et al. to find globally optimal spin pole and area vector solutions \cite{chng2022}. Calef et al. were early adopters of Kaasalainen and Torppa's EGI methods for human-made objects, focusing on multispectrum measurements \cite{calef2006photometric}. Bradley and Axelrad applied EGI methods to recover convex approximations of representative GEO objects \cite{bradley2014}. Fan and Frueh used the EGI with a multi-hypothesis scheme to recover human-made object shapes with measurement noise \cite{fan2019, fan2020thesis, fan2021}. Friedman quantified the observability of EGI inversion to inform sensor tasking schemes \cite{friedman2020, friedman2022}. Cabrera et al. studied the effects of applying area regularization to Fan and Friedman's methods to achieve more accurate reconstructions \cite{cabrera2021}.

A second approach leverages statistical estimation to retrieve shape information. Linares et al. applied an unscented Kalman filter to estimate attitude and convex shape simultaneously, representing shape with vertex displacement on a sphere \cite{linares2012}. Linares et al. used a Multiple-Model Adaptive Estimation (MMAE) algorithm to predict the truth geometry and attitude by comparing observations with a bank of reference objects \cite{linares2014space}. Linares and Crassidis used an Adaptive Hamiltonian Markov Chain Monte Carlo scheme to estimate shape and other characteristics simultaneously \cite{linares2018space}.

A third approach for recovering shape information from light curves relies on deep learning. Linares and Furfaro used a deep convolutional neural network to classify novel light curves as rocket bodies, payloads, or debris \cite{linares2016}. Furfaro et al. used similar methods classify novel light curves into four truth object classes \cite{furfaro2019}. Kerr et al. adapted the architecture developed by Furfaro et al. to classify object shape and size in an extended training set \cite{kerr2021}. McNally et al. investigated the use of different model training methods to identify satellites from simulated and real light curves \cite{mcnally2021}. Allworth et al. applied transfer learning to simulated and real measurements to classify object type \cite{allworth2021}.

Various unique methods have been applied to the light curve shape inversion problem. Hall et al. investigated methods for independently solving shape parameters in isolation without attitude information \cite{hall2007}. Fulcoly et al. used measurements from different sensor locations to determine shape under various attitude profiles \cite{fulcoly2012}. Yanagisawa and Kurosaki fit an analytical light curve model for a tri-axial ellipsoid to derive the shape and attitude profile of a Cosmos rocket body  \cite{yanagisawa2012}. Kobayashi used techniques from compressed sensing to recover shape information from light curves by taking advantage of shadowing geometry \cite{kobayashi2020,kobayashi2021}.

Shape inversion for non-convex objects --- mainly applied to asteroids --- has been studied by others in the past. Durech and Kaasalainen \cite{durech2003} determined a relationship between concavity size and the minimum solar phase angle where self-shadowing impacts the light curve. Viikinkoski et al. \cite{viikinkoski2017} investigated recovering large concavities from adaptive optics imagery, noting the fundamental non-uniqueness of any solution. They discuss how a single large concavity may produce identical scattering behavior to multiple smaller concave features \cite{viikinkoski2017}. Cabrera et al. \cite{cabrera2021} studied convex solutions for non-convex objects, concluding that the convex fit diverges from the true shape as the relative concavity size increases. 

TODO: complete this lit review, add new papers

\section{Contributions}

The framework detailed in this work contributes to the light curve simulation and shape inversion literature in a few notable areas. 

\subsection{Simulation Advances}

A high-fidelity light curve simulator called LightCurveEngine was developed to support inversion algorithm development. Depending on the target shape, the simulator is one to four orders of magnitude faster than ray tracing-based renderers commonly used in literature \cite{fan2019, allworth2020}. It supports self-shadowing, variable material properties, a variety of reflection functions, and dynamic solar panel rotation. In concert with a constrained observer model and orbit propabation, the engine generates realistic light curves for inactive debris, highly non-convex objects, and actively-controlled satellites. 

\subsection{Advances in Convex Shape Inversion}

This work presents a suite of changes that build on the classical shape inversion algorithm for convex shapes \cite{robinson2022}. New resampling and merging steps in the Extended Gaussian Image optimization stage yield more accurate shapes that are easier to reconstruct. An alternative optimization method for the shape support vector decreases convergence time for highly symmetric objects where the classical optimization algorithm fails.

The approach presented in this work solves the shape inversion problem beginning from the direct geometry reconstruction methods of \cite{kaasalainen2001,fan2020thesis}. The EGI optimization processes of \cite{fan2020thesis,cabrera2021,kaasalainen2001} are improved using novel resampling and merging steps. These improvements circumvent the need for the regularization terms explored by Cabrera et al. \cite{cabrera2021}. The support optimization procedure is accelerated and strengthened with a preconditioning term proposed by Nicolet et al. \cite{nicolet2021}, enabling the rapid reconstruction of more detailed convex objects than previously feasible.

This convex shape inversion method has a number of general advantages. It does not require any \textit{a priori} information about the truth geometry. Thus, unlike MMAE methods \cite{linares2014space}, The presented algorithm does not require a bank of reference models to recover shape information. Unlike deep learning methods, the presented method does not rely on the diversity of a training set to achieve realistic results \cite{furfaro2019,kerr2021}.

\subsection{Advances in Non-Convex Shape Inversion}

While natural space objects like asteroids are largely convex, nearly all human-made space objects are highly non-convex, highlighting the need for a robust inversion scheme for both convex and non-convex space objects.