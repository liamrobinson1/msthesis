\ProvidesFile{ch-introduction.tex}[Introduction]
\graphicspath{{/Users/liamrobinson/Documents/PyLightCurves/docs/build/html/_images}}

\chapter{Introduction}

Humankind has been creating space debris since the dawn of the space age \cite{esareport2022}. Early missions like Vanguard 1, launched in March of 1958, set a precedent by leaving both their satellite and the launch vehicle's upper stage in orbit, both of which are still in orbit in 2023 \cite{vanguard1}. Half a century of increasingly frequent launches has created a space environment cluttered with thousands of debris objects, increasing the rate of conjunction events that may require avoidance maneuvers for large satellites in low Earth orbit (LEO) \cite{esareport2022}. This uncontrolled proliferation of human-made space debris puts space operations at risk. High-profile satellite collisions like Iridium-Cosmos in 2009 have added fuel to the fire, producing shells of debris that further pollute LEO \cite{vallado4ed}. Anti-satellite tests carried out by the USA, Russia, China, and India since the 1960s see nations destroying their own satellites, projecting military strength at the cost of creating more debris \cite{vallado4ed}. Beyond LEO in Geostationary Transfer Orbit (GTO), exploding launch vehicle upper stages produce large amounts of debris \cite{esareport2022}. While higher orbits are not yet as polluted as LEO, they do not decay due to atmospheric drag, allowing debris objects to remain in the environment for thousands of years \cite{vallado4ed}.

\begin{figure}[ht]
    \centering
    \includegraphics[width=\figbig]{sphx_glr_propagate_catalog_001.png}
    \caption{Public tracked catalog in 2000 and 2023}
    \label{fig:catalog_comparison}
\end{figure}

In the context of the modern space environment, determining the current state and predicting the future dynamics of space objects is critical for many areas of Space Domain Awareness (SDA) \cite{frueh2019notes}. While the current orbits of objects can be determined accurately from astrometry --- through passive optical imagery or active radar --- their future dynamics are perturbed by non-conservative forces drive by their shape, attitude profile, and material properties that cannot be observed directly. In particular, objects in orbits past LEO are most efficiently observed with optical telescopes as the power required for radar scales with the square of the distance \cite{frueh2019notes}. Because optical observations are already commonly used to characterize the orbit of the objects in orbits past LEO, it is advantageous to use the same images to extract these other useful characteristics. Characterizing an object's shape, attitude, and material properties is fundamentally difficult as distance from the sensor and atmospheric turbulence leaves only a distribution of brightness in the image \cite{fan2020thesis}. The leftover information is the total brightness of the object, and this value over time is known as the light curve. The light curve is a function of the parameters of interest: the object's shape, attitude, and material properties \cite{fan2020thesis, burton2021mapping}. Solving light curve shape inversion in a general case would enable robust active debris removal, anomaly detection, and collision avoidance, all of which are benefitted by accurate shape information.

\section{State of the Art}

Light curve shape inversion was first investigated by Russell in 1906, who proposed a spherical harmonic representation that could be fit to an asteroid shape \cite{russell1906}. Russell was skeptical of the applicability of the approach, hypothesizing that there would be ambiguity in the convex shape such that many solutions would fit the data equally well. The next major contribution to the field was due to Kaasalainen and Torppa in 2000, who successfully reconstructed the shapes of asteriods by directly optimizing the directions and areas of candidate faces to find a convex shape that produces a similar light curve \cite{kaasalainen2000, kaasalainen2001}. Their methods use a data structure called the Extended Gaussian Image (EGI). Kaasalainen and Torppa also addressed non-convex shape inversion, developing an optimization procedure which could reconstruct the largest nonconvex features of an asteroid, noting that this method was highly computationally demanding \cite{kaasalainen2000}. The work of Kaasalainen and Torppa was extended by Durech and Kaasalainen in 2003, who investigated the observability of nonconvex features in asteroid light curves \cite{durech2003}. The authors noted that many nonconvex features are only observable at high phase angles, where self-shadowing effects become pronounced in the light curve. They further found that uncertainty in the surface optical properties renders small local features unobservable through light curve inversion \cite{durech2003}. Using the methods originally proposed by Kaasalainen and Torppa, a collaborative effort between observatories lead to the publication of Database of Asteroid Models from Inversion Techniques (DAMIT), a publicly-available repository of convex asteroid models \cite{damit2014}. Work has also continued in non-convex shape estimation for asteroids. Viikinkoski et al. \cite{viikinkoski2017} investigated recovering large concavities from adaptive optics imagery, noting the fundamental non-uniqueness of any solution. They discuss how a single large concavity may produce identical scattering behavior to multiple smaller concave features \cite{viikinkoski2017}. In 2022, Chng et al. proposed a method to efficiently determine a globally optimal spin pole and convex shape estimate \cite{chng2022}. While the field of asteroid shape inversion has been alive in the intervening years, most works are not relevant to human-made objects.

Direct shape inversion for human-made space objects was first investigated by Calef et al., who adopted Kaasalainen and Torppa's methods applied to multispectrum measurements \cite{calef2006photometric}. Bradley and Axelrad also used asteroid inversion techniques to recover convex approximations of representative GEO objects, yielding good results for rocket body-like shapes but limited success for box-wing satellites and other high area-to-mass ratio (HAMR) objects \cite{bradley2014}. Bradley and Axelrad set a precedent that will become a common theme for many later publications, simulating their own light curves and performing the shape inversion in a separate black box environment \cite{bradley2014}. The most recent major contributions to the direct shape inversion literature are due to Fan and Frueh, who used the EGI with a multi-hypothesis scheme to invert human-made object shapes with measurement noise \cite{fan2019, fan2020thesis, fan2021}. Fan and Frueh studied the non-uniqueness of even convex inversion results under noisy observation conditions, noting the importance of self-shadowing for simulating realistic light curves \cite{fan2020thesis}. Fan notes that full observability is crucial for successful direct shape inversion, pointing to work by Friedman and Frueh, who quantified the observability of EGI inversion to inform sensor tasking schemes \cite{friedman2020, friedman2022}. Cabrera et al. studied the effects of applying area regularization to Fan and Friedman's methods to achieve more accurate convex shape reconstructions \cite{cabrera2021}. Cabrera et al. also found that the convex shape solutions for non-convex objects diverge from the true shape as the concavity size grows \cite{cabrera2021}. Throughout the shape inversion literature, two themes are clear. Effective and efficient methods for non-convex shape inversion for human-made objects are needed, and there are no existing methods for inverting the shapes of actively-controlled box-wing satellites. This work seeks to address both of these challenges.

While this work is based primarily in furthering EGI-based shape inversion methods, it is important to recognize the variety other approaches that have been applied to problem. Outside of the asteroid-inspired EGI methods, the literature falls into two broad categories: filter-based inversion, and machine learning categorization. Each offers different advantages while imposing unique limitations. Filter-based shape inversion was been pioneeded by Linares through work with a variety of co-authors. These filter-based methods often seek to perform multiple types of object characterization simultaneously, estimating attitude and material properties in addition to shape \cite{linares2012, linares2014space, linares2018space}. Towards this end, Linares et al. have implemented unscented Kalman filters \cite{linares2012}, multiple-model adaptive estimation algorithms \cite{linares2014space}, and adaptive Hamiltonian Markov chain Monte Carlo schemes \cite{linares2018space}. While these methods achieve good results for simple simulated shapes, the authors note the importance of further investigation with a more accurate BRDF and more complex shapes \cite{linares2018space}. In general, filter-based approaches are limited by the nonlinearity of highly specular and complex human-made objects. Direct inversion methods require more information \textit{a priori}, but are able to deliver more accurate shape estimates in return.

By contrast, machine learning categorization methods indirectly recover shape information by predicting which class of objects an observed light curve belongs to. Linares and Furfaro used a deep convolutional neural network to classify novel light curves as rocket bodies, payloads, or debris \cite{linares2016}. Other authors, including Kerr et al. and McNally et al. have adapted the architecture developed by Furfaro et al. to classify novel light curves into an extended set of object types \cite{kerr2021, mcnally2021}. These models are trained on simulated data, although Allworth et al. successfully applied transfer learning to classify real measurements using a synthetically-trained model \cite{allworth2021}. Due to their indirect nature, machine learning approaches may be viewed as not strictly shape \textit{inversion} methods, but nevertheless fill an operational niche by providing object class probabilities to inform further direct methods. The highly nonlinear nature of light curves makes even this task difficult to generalize, as models may behave unpredictably when provided a light curve from an object type outside the training classes.

The state of the art in light curve simulation differs between approaches and the object class under study. Kaasalainen and Torppa, as well Fan, Friedman, Kobayashi, and Frueh employ a Lambertian model for convex objects with a facetwise ray tracing scheme for non-convex objects \cite{kaasalainen2001, fan2016, fan2020thesis,friedman2020,kobayashi2020,frueh2014}. This approach has the advantage of being simple, but can be computationally intensive for complex objects. Allworth et al. developed a ray traced light curve simulator in based on Blender's cycles renderer, allowing them to account for photorealistic shadowing and motion blur \cite{allworth2020, allworth2021}. Furfaro et al. \cite{furfaro2019} and Cabrera and Bradley \cite{cabrera2021,bradley2014} use a simple Lambertian model with no self-shadowing. Many more authors apply a more specialized non-Lambertian Bidirectional Reflectance Distribution Function (BRDF) for their lighting \cite{linares2018space, mcnally2021, blacketer2022}. Throughout the literature, there is a clear gap between the simulated light curves and their observed counterparts. Due to the difference in quality, authors often treat real and simulated data very differently \cite{allworth2021}.

% \section{Contributions}

% \subsection{Simulation Advances}

% A high-fidelity light curve simulator was developed to act as a digital twin of the Purdue Optical Ground station and support inversion algorithm development. This simulator is one to four orders of magnitude faster than ray tracing-based renderers commonly used in literature \cite{fan2019, allworth2020}. It supports self-shadowing, variable material properties, a variety of reflection functions, and dynamic solar panel rotation. In concert with a constrained observer model and orbit propabation, the simulator generates realistic, noisy light curves for inactive debris, highly non-convex objects, and actively-controlled satellites. 

% \subsection{Advances in Convex Shape Inversion}

% This work presents a suite of changes that build on the classical shape inversion algorithm for convex shapes \cite{robinson2022}. New resampling and merging steps in the Extended Gaussian Image optimization stage yield more accurate shapes that are easier to reconstruct. An alternative optimization method for the shape support vector decreases convergence time for highly symmetric objects where the classical optimization algorithm fails.

% The approach presented in this work solves the shape inversion problem beginning from the direct geometry reconstruction methods of \cite{kaasalainen2001,fan2020thesis}. The EGI optimization processes of \cite{fan2020thesis,cabrera2021,kaasalainen2001} are improved using novel resampling and merging steps. These improvements circumvent the need for the regularization terms explored by Cabrera et al. \cite{cabrera2021}.

% This convex shape inversion method has a number of general advantages. It does not require any \textit{a priori} information about the truth geometry. Unlike MMAE methods \cite{linares2014space}, the presented algorithm does not require a bank of reference models to recover shape information. Unlike deep learning methods, the presented method does not require a diverse set of training data to achieve good results \cite{furfaro2019,kerr2021}.

% \subsection{Advances in Non-Convex Shape Inversion}

% While natural space objects like asteroids are largely convex, nearly all human-made space objects are highly non-convex, highlighting the need for a robust inversion scheme for non-convex space objects. If the object's material properties are well-known, the algorithm developed in this work is able to predict whether any single non-convex feature is self-shadowed in the observed light curve and introduces a concavity in the shape estimate in the correct location.