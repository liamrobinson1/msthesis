\ProvidesFile{ch-background-signals.tex}[Background Signals]
\chapter{Background Signals}
\graphicspath{{/Users/liamrobinson/Documents/PyLightCurves/docs/build/html/_images}}

Whenever an optical telescope is observing a space object, the object's signal is necessarily superimposed on whatever signals exist in the background. In this context, background does not only refer to sources physically further than the object, but all sources that impact the image apart from the object signal. As we will see, some of these sources even originate within the telescope. To faithfully simulate a telescope observing an object, many position-based SDA tasks are able to ignore background effects while acquiring or tracking objects. For photometry-based SDA, the background is critical. Background signals can be broken up into atmospheric effects, exoatmospheric effects, and sensor effects. 

\section{Atmospheric Effects}

\subsection{Airglow}

Certain chemical reactions from 80-110 km altitude in the upper atmosphere release visible light
\cite{krag2003}. This effect is known as
airglow. Since these reactions are assumed to be isotropic ---  equally intense when integrated along any
vertical line extending upwards from the surface. We model the airglow signal $\textrm{AINT}$ in a
similar fashion to integrated starlight. Given the airglow spectra $\textrm{GLINT}(\lambda) \:
\left[ \frac{W}{m^2\cdot m \cdot rad^2} \right]$, we compute Eq \ref{eq:aint}.

\begin{equation} \label{eq:aint}
 \textrm{AINT} = \frac{\pi D^2}{4}
  \int_{10^{-8}}^{10^{-6}}{ \textrm{GLINT}(\lambda) \cdot \textrm{QE}(\lambda) \cdot \textrm{ATM}(\lambda)
  \cdot \left( \frac{\lambda}{h c} \right) \: d\lambda}  
\end{equation}

The quantity $\textrm{AINT}$ has units $\left[ \frac{1}{s\cdot rad^2} \right]$, meaning that the
mean airglow signal in ADU per pixel is simply given by Eq \ref{eq:airglow_adu}

\begin{equation} \label{eq:airglow_adu}
  \bar{S}_{airglow} = AINT \cdot \frac{1}{\cos(\theta_z)} \cdot \Delta t \cdot \left( \frac{\pi s_{pix}}{648000} \right)^2
\end{equation}

In Eq \ref{eq:airglow_adu}, $\frac{1}{\cos(\theta_z)}$ is known as the Van Rhijn factor, which
accounts for the accumulation of airmass near the horizon \cite{frueh2019notes}.

\begin{figure}[ht]
  \centering
  \includegraphics[width=\figmed]{sphx_glr_background_signals_005.png}
  \caption{Mean airglow signal on the local observer hemisphere. The observer is in New Mexico, USA at
  \pogslla}
  \label{fig:airglowhemi}
\end{figure}

\subsection{Light Pollution}

The final source of background noise light pollution. On a cloudless night with negligible light
pollution, the zenith surface brightness is approximately $22 \: \left[ \frac{mag}{arcsec^2}
\right]$ (MPSAS) \cite{krag2003}. As light pollution increases, this zenith brightness may dip down to
$14-15 \: \left[ \frac{mag}{arcsec^2} \right]$. To get accurate localized zenith brightness values,
we use the 2015 World Atlas of Sky Brightness dataset \cite{falchi2016_data}. The data is reported in $\left[
	\frac{mcd}{cm^2} \right]$ on a 30-arcsecond grid, requiring us to convert to MPSAS. A subset of the global dataset is displayed in \ref{fig:pollution_data} This conversion is listed in Eq \ref{eq:cd_per_m2_to_mpsas}, using a monochromatic $\lambda = 474$ nm to fit Falchi et al.'s example conversions \cite{falchi2016}.  

\begin{figure}[ht]
  \centering
  \includegraphics[width=\figmed]{sphx_glr_nightlights_001_2_00x.png}
  \caption{Zenith light pollution in the eastern USA, data from \cite{falchi2016_data}}
  \label{fig:pollution_data}
\end{figure}

The mean light pollution CCD signal in ADU per pixel is formulated similarly to airglow. Given the station's zenith surface brightness $B_{poll,z}$ in MPSAS, we first convert to irradiance per steradian via \ref{eq:mpsas_to_irrad_per_ster}, which we then input into \ref{eq:pollution_adu} to compute the mean signal in ADU per pixel. Note that Krag does not implement a specific light pollution model, but instead takes the dark sky site zenith brightness of $22$ MPSAS as an input to an atmospherically scattered light model. Here we simply use that model's formulation with a variable zenith brightness.
 
\begin{equation} \label{eq:pollution_adu}
  \bar{S}_{pollution} = B_{poll,z} \cdot SINT \cdot \frac{1}{\cos(\theta_z)} \cdot \Delta t \cdot \left( \frac{\pi s_{pix}}{648000} \right)^2
\end{equation}

\begin{figure}[ht]
  \centering
  \includegraphics[width=\figmed]{sphx_glr_background_signals_003.png}
  \caption{Mean light pollution signal on the local observer hemisphere. The observer is in New Mexico, USA at
  \pogslla}
  \label{fig:pollution_hemi}
\end{figure}

\subsection{Twilight}

Even after the Sun sets, scattered sunlight in the upper atmosphere creates a signal on our CCD. The twilight model implemented for this work is due to Patat et al. and was developed for the European Southern Observatory at Paranal in Chile \cite{patat2006}. This model implements the zenith brightness as a function of the solar zenith angle $\gamma$ --- the angle from zenith to the Sun's apparent centroid. Patat et al.'s model fits a second-degree polynomial in $\gamma$ to approximately 2000 observations, yielding separate curves for each of the UBVRI passbands. For example, for the V band, the twilight zenith brightness in MPSAS is given by \ref{eq:b_zenith_twilight} \cite{patat2006}.

\begin{equation} \label{eq:b_zenith_twilight}
  B_{twi,z} = 11.84 + 1.518(\gamma - 95^\circ) - 0.057 (\gamma -  95^\circ)^2
\end{equation}

Eq \ref{eq:b_zenith_twilight} is valid from $95^\circ \leq \gamma \leq 105^\circ$. Before $\gamma 95^\circ$, we take the zenith brightness to be constant and equal to the brightness at $\gamma 95^\circ$. This is not accurate, but is sufficiently bright to correctly forbid daytime observations by lowering the SNR drastically. After $\gamma = 105^\circ$ we set the zenith surface brightness to $B_{twi,z} == 22$ MPSAS to match the optimal observation condition of the light pollution model \cite{krag2003}. Zenith twilight brightness is plotted as a function of $\gamma$ in Figure \ref{fig:twilight_model}.

\begin{figure}[ht]
  \centering
  \includegraphics[width=\figmed]{sphx_glr_twilight_model_001_2_00x.png}
  \caption{Twilight model surface brightness at zenith as a function of solar zenith angle}
  \label{fig:twilight_model}
\end{figure}

Computing the mean CCD signal in ADU per pixel due to the twilight brightness proceeds identically to the light pollution formulation. 

\begin{equation} \label{eq:twilight_adu}
  \bar{S}_{twilight} = B_{twi,z} \cdot SINT \cdot \frac{1}{\cos(\theta_z)} \cdot \Delta t \cdot \left( \frac{\pi s_{pix}}{648000} \right)^2
\end{equation}

\begin{figure}[ht]
  \centering
  \includegraphics[width=\figmed]{sphx_glr_background_signals_006.png}
  \caption{Mean twilight signal on the local observer hemisphere. The observer is in New Mexico, USA at
  \pogslla}
  \label{fig:pollution_hemi}
\end{figure}

\section{Exoatmospheric Effects}

\subsection{Integrated Starlight}

Stars are almost always present in optical images of space objects. The brightest stars streaking across the field of view in Figure \ref{fig:pollution_hemi} have high SNRs and stand out clearly against the dark background. This raises a question: if we're observing a full $1^\circ \times 1^\circ$ area of the sky, where are the rest of the stars given that the Milky Way alone contains approximately $1\cdot10^{11}$ stars? The answer is relatively obvious: many more stars are present in the image than we can pick out individually, most of them fall into the background. We call this residual faint starlight "integrated" starlight as we are effectively integrating the signals from thousands or millions of stars across the image plane. 

\begin{figure}[ht]
  \centering
  \includegraphics[width=\figmed]{static_images/static_pogs_raw_image.png}
  \caption{Raw image of three GEO objects with stars streaking through the background. Taken by the Purdue Optical Ground station at \pogslla by Nathan Houtz}
  \label{fig:pogs_observation_example}
\end{figure}

In Figure \ref{fig:pogs_observation_example}, most stars are too faint to appear as points of light on the image plane. Instead, they merge into the background. The signal due to these faint stars is known as integrated starlight.
Krag \cite{krag2003} modeled this signal by building a $1^\circ \times 1^\circ$ grid of surface
brightness values for the full right ascension (RA) and declination (Dec) sphere. Krag used the
Guide Star catalog, which contains 15 million stars down to magnitude 16. Exponential extrapolation
was used to predict star counts in each bin for higher magnitudes \cite{krag2003}. Twenty years later, we have
access to larger star catalogs that are nearly complete to much dimmer magnitudes. The integrated
starlight catalog used in this work was built from the GAIA catalog with approximately 1.5 billion
stars down to magnitude 21-22 \cite{gaia_dr3}. The same $1^\circ \times 1^\circ$ grid was computed
using the \texttt{astroquery.gaia} Python package \cite{astroquery_gaia}. Figure
\ref{fig:gaiapatched} shows the computed patched catalog, in units of $S_{10}$. 

\begin{figure}[ht]
  \centering
  \includegraphics[width=\figbig]{sphx_glr_gaia_patched_catalog_001_2_00x.png}
  \caption{Integrated starlight patched catalog}
  \label{fig:gaiapatched}
\end{figure}

Now that we have a data source for the exoatmospheric mean brightness of the night sky due to integrated
starlight, we can compute the corresponding signal mean for a telescope equipped with a CCD sensor.
Again, we adopt Krag's formulation \cite{krag2003}.

\begin{equation} \label{eq:bint}
 \textrm{BINT} = \frac{\pi D^2}{4}
  \int_{10^{-8}}^{10^{-6}}{ \textrm{STRINT}(\lambda) \cdot \textrm{QE}(\lambda) \cdot \textrm{ATM}(\lambda)
  \cdot \left( \frac{\lambda}{h c} \right) \: d\lambda}  
\end{equation}

In Eq \ref{eq:bint}, $D$ is the telescope aperture diameter in meters, $h$ is Plank's constant in
$\left[ \frac{m^2 kg}{s} \right]$, and $c$
is the speed of light in vacuum in $\left[ \frac{m}{s} \right]$. The resulting quantity
$\textrm{BINT}$ has units of $\left[ \frac{1}{s} \right]$, representing the mean total photons passing
through the telescope aperture due to integrated starlight. 

\begin{equation} \label{eq:starlightmean}
  \bar{S}_{star} = 10^{-4} \cdot BINT \cdot \left( \frac{s_{pix}}{3600} \right)^2 \cdot \Delta t \cdot
  b_{cat}
\end{equation}

In Eq \ref{eq:starlightmean}, $b_{cat}$ is the patched catalog brightness in $\left[ S_{10}
\right]$, $s_{pix}$ is the telescope pixel scale in $\left[ \frac{arcsecond}{pix} \right]$, and $\Delta t$ is the integration time in seconds. Note the addition of the $10^{-4}$ factor to reconcile catalog surface brightness in terms of 10th magnitude stars, and the 0th magnitude source in $\textrm{BINT}$. This yields $\bar{S}_{star}$ with units $\left[ \frac{e^-}{pix^2} \right]$; photoelectron counts (ADU) per pixel. Figure \ref{fig:starlight_hemi} shows the background signal mean due to integrated starlight.

\begin{figure}[ht]
  \centering
  \includegraphics[width=\figmed]{sphx_glr_background_signals_002.png}
  \caption{Integrated starlight signal on the local observer hemisphere. The observer is in New Mexico, USA at
  \pogslla}
  \label{fig:starlight_hemi}
\end{figure}

\subsection{Scattered Moonlight}

Moonlight scattering through the atmosphere significant increases background brightness \cite{krag2003}. This scattering effect can be decomposed into Rayleigh (isotropically distributed) and Mie (exponentially distributed) scattering modes. The Rayleigh scattered component is computed with Table 4 published by Daniels parameterized by the angle from the observation to zenith $z_{obs}$, the angle from the Moon to zenith $z_{moon}$, and the angle between the observation and the Moon on the horizon $\Delta Az$ \cite{daniels1977}. Interpolating this table yields the intensity of the Rayleigh scattering $F_{rs}$ in $10^{-10}$ $W/(cm^2 \cdot \mu m \cdot sr)$ \cite{krag2003}. The Mie scattered component is formulated with Eq \ref{eq:mie_scattering_moon}.

\begin{equation} \label{eq:mie_scattering_moon}
  F_{ms}(\lambda) = a_1 \left[ e^{-\left(\frac{\Psi}{\Psi_1}\right)} + a_2 e^{-\left(\frac{\pi - \Psi}{\Psi_2}\right)} \right] F_{rs}(\lambda)
\end{equation}

Daniels recommends $a_1 \in [50, 100]$, $a_2 \in [0.01, 0.02]$, $\Psi_1 \in [10^\circ, 20^\circ]$, and $\Psi_2 \approx 50$ \cite{daniels1977}. Prior to any station-specific fitting, we choose the middle of these intervals, yielding $a_1 = 75$, $a_2 = 0.015$, $\Psi_1 = 15^\circ$, and $\Psi_2 = 50^\circ$. $a_1$ and $a_2$ are dimensionless, such that $F_{ms}$ also has units of $10^{-10}$ $W/(cm^2 \cdot \mu m \cdot sr)$, allowing us to compute the total intensity of the scattered moonlight $F_{mt}$ via Eq \ref{eq:total_scattered_moonlight} following Krag's formulation \cite{krag2003}.

\begin{equation} \label{eq:total_scattered_moonlight}
  F_{mt} = f(\theta) \left[ F_{rs}(\lambda) + F_{ms}(\lambda) \right]
\end{equation}

in Eq \ref{eq:total_scattered_moonlight}, $f(\theta)$ is the lunar phase function which describes the fraction of the full Moon brightness is reflected at an observer viewing the Moon an angle $\theta$ from the Sun vector. This function is linearly interpolated within Table 3 in \cite{daniels1977}. Finally, Krag introduces a correction factor $f_{corr}$ to account for the difference between the Sun's irradiance spectrum and the spectrum of scattered moonlight, defined in Eq \ref{eq:krag_f_corr}.

\begin{equation} \label{eq:krag_f_corr}
  f_{corr} = \frac{I_0}{SUN(550 \: \left[\textrm{nm}\right])}
\end{equation}

With all these pieces, we can put together the mean scattered moonlight signal in ADU per pixel in Eq \ref{eq:moonlight_adu}.

\begin{equation} \label{eq:moonlight_adu}
  \bar{S}_{moon} = F_{mt}(550 \: \left[\textrm{nm}\right]) \cdot SINT \cdot \left( \frac{s_{pix}}{3600} \right)^2 \cdot \Delta t \cdot f_{corr}
\end{equation}

\begin{figure}[ht]
  \centering
  \includegraphics[width=\figmed]{sphx_glr_background_signals_001.png}
  \caption{Mean scattered moonlight signal on the local observer hemisphere. The observer is in New Mexico, USA at
  \pogslla}
  \label{fig:moonlight_hemi}
\end{figure}

\subsection{Zodiacal Light}

Zodiacal light is an effect created by sunlight reflecting off of dust in the ecliptic plane \cite{krag2003}. Zodiacal light is strongest around the Sun --- an area that is not of interets for us --- but also reaches a peak directly away from the Sun due to the opposition effect. This peak is known as the Gegenschein, meaning "opposing light". We compute the of the zodiacal light via Table 1 of \cite{roach1972}. This reports the surface brightness of the zodiacal light in $S_{10}$, which we use without conversion to find the mean CCD signal in ADU per pixel via Eq \ref{eq:zodiacal_adu}.

\begin{equation} \label{eq:zodiacal_adu}
  \bar{S}_{zod} = BINT \cdot \left( \frac{s_{pix}}{3600} \right)^2 \cdot \Delta t \cdot ZOD \cdot 10^{-4}
\end{equation}

As in the integrated starlight signal, the $10^{-4}$ factor reconciles the $S_{10}$ surface brightness with the 0th magnitude source in $\textrm{BINT}$. 

\begin{figure}[ht]
  \centering
  \includegraphics[width=\figmed]{sphx_glr_background_signals_004.png}
  \caption{Mean zodiacal light signal on the local observer hemisphere. The observer is in New Mexico, USA at
  \pogslla}
  \label{fig:zod_hemi}
\end{figure}

\subsection{Sampling Background}

Notice that each background signal is only defined in terms of its mean. On a pixel-by-pixel basis, the signal for an exposure is sampled from a Poisson distribution for each background term. This distribution can be interpreted as modeling the number of independent events that occur during a time period. In our case, this translates to individual photons being incident on our sensor. A Poisson distribution is defined on the positive integers by a single parameter $\lambda$ which is both the mean and variance of the distribution. The probability density function (PDF) for the Poisson distribution takes the form of Eq \ref{eq:poisson_pdf} \cite{frueh2019notes}.

\begin{equation} \label{eq:poisson_pdf}
  P_\lambda(x=k) = \frac{\lambda^k e^{-\lambda}}{k!}
\end{equation}

This distribution has a useful property that $P_{\lambda_1 + \lambda_2}(x=k) = P_{\lambda_1}(x=k) + P_{\lambda_2}(x=k)$ so long as the distributions described by $\lambda_1$ and $\lambda_2$ are independent. Since our background sources are assumed to be independent as sources like moonlight and zodiacal light are clearly distinct; if the Moon vanished, interplanetary dust across the solar system would reflect light identically. This means that we can formulate the total background signal as a single Poisson variable.

\begin{equation} \label{eq:background_poisson}
  \lambda_{background} = \bar{S}_{airglow} + \bar{S}_{pollution} + \bar{S}_{twilight} + \bar{S}_{star} + \bar{S}_{moon} + \bar{S}_{zod}
\end{equation}

To compute the background of the CCD image, we simply sample from the Poisson distribution defined by $\lambda_{background}$. 

\subsection{Background Source Importance}

Some background signals are more impactful than others. Table \ref{tb:signal_importance} details the approximate magnitudes in photoelectrons per pixel one can expect from a telescope similar to the Purdue Optical Ground Station.

\begin{table}[] \label{tb:signal_importance}
  \begin{tabular}{|l|l|}
  \hline
  \textbf{Signal source} & \textbf{Magnitude} $\mathbf{\left[ e^- / \textbf{pix}\right]}$ \\ \hline
  Airglow                & $10^3 - 10^4$                              \\ \hline
  Scattered moonlight    & $0 - 10^5$                                 \\ \hline
  Integrated starlight   & $10^1 - 10^2$                              \\ \hline
  Light pollution        & $10^2 - 10^3$                              \\ \hline
  Zodiacal light         & $10^2 - 10^4$                              \\ \hline
  Twilight               & $10^1 - 10^7$                              \\ \hline
  \end{tabular}
  \caption{Background signal importance}
\end{table}

\section{Sensor Effects}

\subsection{Dark Noise}


\subsection{Readout Noise}
