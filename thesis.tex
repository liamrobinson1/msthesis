\ProvidesFile{thesis.tex}[2022-10-05 PurdueThesis thesis.tex file]

%
%  The home page for the PurdueThesis software is
%      https://engineering.purdue.edu/~mark/PurdueThesis/
%
%  Be sure to sign up for the PurdueThesis mailing list at
%      https://engineering.purdue.edu/ECN/mailman/listinfo/purduethesis-list
%  so you learn of new versions of this software.  You must be on that
%  mailing list to receive help with this software.
%
%  This is the template root file for an example thesis (for master's
%  degree) or dissertation (for a Ph.D.).  From now on "thesis" will
%  refer to both of these unless stated otherwise.
%
%  LaTeX systems include auxiliary programs to do bibliographies,
%  indexes, etc.  The latexmk program runs the fewest programs needed
%  to update your thesis.  latexmk runs automatically on Overleaf.  If
%  you're LaTeXing your document on a non-Overleaf system you may need
%  to run latexmk manually.
%
%  This thesis contains Feynman diagrams in the ap-physics.tex file.
%  For these to be processed correctly you must use the lualatex
%  program:
%      latexmk -lualatex thesis
%  (If your thesis doesn't have Feynman diagrams---the
%      \ProvidesFile{ap-physics.tex}[2022-10-05 Physics appendix]

\begin{VerbatimOut}{z.out}
\chapter{PHYSICS}
\ix{physics//Physics appendix}

Feynman diagrams\ix{Feynman diagram}
show what happens
when elementary particles collide
\cite{feynman-diagram}.
The Feynman diagrams below are from the
\citetitle{ellis2016} documentation \cite{ellis2016}.
\textbf{%
  You must use \texttt{lualatex} instead
  of \texttt{pdflatex}
  to process documents that use the \texttt{tikz-feynman} package.%
}

The input
in the documentation
is different than here because a different random number generator
is used \cite{menke2019}.
I expect this to be corrected.
In the meantime try replacing \texttt{vertical}
with \texttt{vertical'}
and/or switch some \texttt{fermion}
to \texttt{anti} \texttt{fermion} lines \cite{ellis2017}.
\end{VerbatimOut}

\MyIO


\begin{VerbatimOut}{z.out}
\feynmandiagram [large, vertical'=e to f] {
  a -- [fermion] b -- [photon, momentum=\(k\)] c -- [fermion] d,
  b -- [fermion, momentum'=\(p_{1}\)] e -- [fermion, momentum'=\(p_{2}\)] c,
  e -- [gluon] f,
  h -- [anti fermion] f -- [anti fermion] i,
};
\end{VerbatimOut}

\MyIO


\begin{VerbatimOut}{z.out}
\feynmandiagram [horizontal=a to b] {
  i1 -- [anti fermion] a -- [anti fermion] i2,
  a -- [photon] b,
  f1 -- [fermion] b -- [fermion] f2,
};
\end{VerbatimOut}

\MyIO

%  command may be commented out by prefixing it with a
%  '%') use pdflatex instead of lualatex:
%      latexmk thesis
%
%  To make a final PDF file before you turn in your thesis do
%      latexmk -g thesis
%  This makes sure than everything is done for your final version.
%
%  References cited below:
%
%  TM2017 is short for Thesis Manual 2017:
%    A Manual for the Preparation of Graduate Theses,
%    eighth revised edition,
%    Thesis and Dissertation Office,
%    Purdue University,
%    2017,
%    revised August 30, 2017,
%    http://www.purdue.edu/gradschool/documents/thesis/graduate-thesis-manual.pdf,
%    last retrieved on May, 8, 2021.
%
%  In this file, change the example information to your information.
%

\def\ZZinstitution{Purdue University}
\def\ZZcampus{West\space Lafayette}
\def\ZZprogram{Aeronautics and Astronautics}
\def\ZZdegree{Master of Science in Aeronautics and Astronautics}
\def\ZZauthor{Liam Robinson}
\def\ZZdocument{A Thesis}
\def\ZZgraduation{December 2023}

% title
% If you need to manually split the title,
% over several lines do, for example,
%     \def\ZZtitle{%
%       This is the First Line\\[-6pt]
%       and this is the Second Line%
%     }
\def\ZZtitle{An Integrated Framework for Light Curve Simulation and Shape Inversion of Human-Made Space Objects}
% Must all be false
\def\ZZshowcolophon{false}
\def\ZZshowdiagonalline{false}
\def\ZZshowgridlines{false}
\def\ZZshowmarginlines{false}
\def\ZZshowtimestamp{false}
\def\ZZtodonotes{false}

\documentclass{PurdueThesis}
\def\ZZatinformation{}
\graphicspath{{graphics/}}
\makeatletter
  \def\input@path{{packages/}}
\makeatother
\ConfigureBibliography

%
% This is only done if you are using BibLaTeX.
%
%
% If you don't want to ignore urldate fields,
% comment out (put "%" before) the next ten lines.
%
\DeclareSourcemap
  {
    \maps[datatype=bibtex]
    {
      % Ignore "urldate = {...}" in .bib files.
      % See the first complete example on page 201 of
      %     https://mirrors.rit.edu/CTAN/macros/latex/contrib/biblatex/doc/biblatex.pdf
      \map
        {
          \step[fieldset=urldate, null]
        }
        % Enter approximate (circa) dates using, for example,
        % "year = c2020"  See
        %     https://tex.stackexchange.com/questions/224617/what-is-the-correct-way-to-handle-approximate-dates-in-biblatex
      \map[overwrite=false]
        {
          \step[fieldsource=year]
          \step[fieldset=sortyear, origfieldval, final]
          \step[fieldsource=sortyear, match={c}, replace={}]
        }
    }
  }

% To let {\bfseries\scshape text} work as expected.
% See
%     https://tex.stackexchange.com/questions/27411/small-caps-and-bold-face
\usepackage{bold-extra}


% For typesetting cryptography pseudocode, algorithms, and protocols.
% See
%     https://mirror.las.iastate.edu/tex-archive/macros/latex/contrib/cryptocode/cryptocode.pdf
\usepackage
[
  n,
  advantage,
  operators,
  sets,
  adversary,
  landau,
  probability,
  notions,
  logic,
  ff,
  mm,
  primitives,
  events,
  complexity,
  oracles,
  asymptotics,
  keys,
]
{cryptocode}

\usepackage{fancyvrb}
  \DefineShortVerb{\|}  % so "|verbatim|" will be verbatim

% For \InpuutIfFileExists.
\usepackage{filehook}

% So "_" will work in URLs when using BibTeX.
\usepackage[T1]{fontenc}

% For nlui testing.
\usepackage{listings}


\usepackage{cancel}


\usepackage{etoc}
\usepackage{makeidx}
  % By default \index ignores its argument.
  % This activates indexing.
  \makeindex
  % The "chapter name" for the index.
  \renewcommand{\indexname}{INDEX}

\usepackage{mathtools}

% Define \includemedia.
\usepackage{media9}

% Define \begin{multicols}{number_of_columns} ... \end{multicolumns}.
% Used in ap-text.tex.
\usepackage{multicol}

% Define \ditto.
\usepackage{pa-ditto}

% Define \FigureDash.
% \FigureDash is a dash the width of a digit in the current font.
\usepackage{pa-figure-dash}

% For PurdueThesis, PuTh, TeX, LaTeX, METAFONT, METAPOST, etc. related logos.
\usepackage{pa-logos}

% (Or maybe use isomath instead?  -mark  2021-06-20)
% Follow ISO 80000-2:2019
%     o   put e, i, j, and pi in upright font automatically
%     o   use, for example, "\di x" to get "\,mathrm{d}\/x"
% This loads
%     o   amsmath.sty (which is already loaded above)
%     o   mathtools.sty
%     o   upgreek.sty
% Load the package.
\usepackage{pa-mismath}
  % Tell mismath to put e, i, j, and pi in upright font automatically.
  \enumber
  \inumber
  \jnumber
  \pinumber
  % To typeset math italic e, i, j, and pi use
  %     \mathit e
  %     \mathit i
  %     \mathit j
  %     \itpi

% Define \MyRepeat{what}{repeat}.
% Do "what" "repeat" number of times.
\usepackage{pa-repeat}

% Define \FloatBarrier.
% \FloatBarrier process all unproccesed floats (tables, figures, etc.).
\usepackage{placeins}

% Define \hl.
% Undefine \st so soul will load without an error.
% I hope \st wasn't used for something important!
\let\st\relax
\usepackage{soul}

% Define \textcent.
\usepackage{textcomp}

% !!! This doesn't work yet, figure it out later.
% For \textprimstress.
% \usepackage{tipa}

% Needed for chapter "Graphics", section "TikZ and PGF".
\usepackage{tikz}
  % Needed to customize arrows.
  \usetikzlibrary{arrows.meta}
  % For electrical diagrams.
  % Uses the TikZ package.
  % The circuitikz name is short for "circuit TikZ".
  \usepackage{circuitikz}
  %
  \usepackage{menukeys}
  %
  % Needed for 3D TikZ stuff.
  \usetikzlibrary{3d}
  %
  % Needed for pa-typographic-conventions package.
  \usetikzlibrary{calc,shadows,shapes.misc,shapes.symbols}
  %
  % Needed for putting text along a path.
  \usetikzlibrary{decorations.text}
  %
  % Draw TikZ decorations.
  % Needed for at least the Kalman filter system model graphic.
  \usetikzlibrary{decorations.pathmorphing} % noisy shapes
  %
  % Fit shapes to coordinates.
  % Needed for at least the Kalman filter system model graphic.
  \usetikzlibrary{fit}
  %
  % Draw the background after the foreground.
  \usetikzlibrary{backgrounds}	% drawing the background after the foreground

% Needed for the Feynman diagram in ap-physics.tex.
% Tikz-feynman requires LuaLaTeX instead of pdflatex be run.
% LuaLaTeX screws up spacing in the list of figures so this
% is not loaded and LuaLaTeX should not be used.
\usepackage[compat=1.1.0]{tikz-feynman}

% The vertical space between a table heading and the table contents
% in a tabular environment.
\newcommand{\tabularspace}{\noalign{\vspace*{2pt}}}

% For \sfrac, used to do slanted fractions, similar to, e.g., 1/2,
% but 1 is small and high and 2 is small and low.
\usepackage{xfrac}


% Define \I.
% \I1 does \indent once, \I2 does \indent twice, etc.
\newcommand{\I}[1]{\MyRepeat{\indent}{#1}}

% Define \MyI.
% Typeset my input.
\long\def\MyI#1%
  {%
    {%
      \fontsize{8}{10}\tt
      \VerbatimInput
        [
          firstnumber = 1,
          numbers     = left,
          xleftmargin = 0.33in,
        ]%
        {#1}
    }%
  }

% Define \MyIO.
% Typeset my input and output.
% The input will all be on the same page.
% The output may be split over multiple pages.
\newcommand{\MyIO}
  {%
    \input{z.out}

    {%
      \fontsize{8}{10}\tt
      \VerbatimInput
        [
          firstnumber = 1,
          numbers     = left,
          xleftmargin = 0.33in,
        ]
        {z.out}
    }
    \FloatBarrier
  }

% Define \MyIOS.
% Typeset my input and output.
% The input may be split over multiple pages.
% The output may be split over multiple pages.
% This doesn't work right:
%     o  Putting a \vbox around the input and output
%        does not allow todoindex entries to be listed.
%     o  Using \vfilneg at beginning and end of definition
%        screws up vertical spacing.
% \newcommand{\MyIOS}
% {%
%   \input{z.out}
%
%   {%
%     \fontsize{8}{10}\tt
%     \VerbatimInput
%     [
%       firstnumber = 1,
%       numbers     = left,
%       xleftmargin = 0.33in,
%     ]{z.out}%
%   }
% }

% Define \MyIOT.
% Typset my input and output together on the same page.
% This doesn't work right:
%     o  Putting a \vbox around the input and output
%        does not allow todoindex entries to be listed.
%     o  Using \vfilneg at beginning and end of definition
%        screws up vertical spacing.
% \def\MyIOT
% {%
%   \vfilneg
%   % \vbox
%   {%
%     \input{z.out}%
%     \fontsize{8}{10}\tt
%     \VerbatimInput[
%       firstnumber = 1,
%       numbers     = left,
%       xleftmargin = 0.33in,
%     ]{z.out}%
%   }%
%   \FloatBarrier
%   \vfilneg
% }

% Define \NL (newline) so LaTeX goes to the next output line.
% Just doing \\ complains
%     ! LaTeX Error: There's no line here to end.
% \mbox{} is an empty math box.
\newcommand{\NL}{\mbox{}\\}

% Print a list of files used and their version numbers in the log file.
\listfiles


\usepackage{pa-typographic-conventions}


% For the \begin{example} ... \end{example} environment
% used in ap-linguistics.tex.
\usepackage{covington}
\usepackage{slgloss}

% "CTAN---Comprehensive" did not get hyphenated and extended
% into the right margin when using BibLaTeX and the apa style.
% These did not change it:
%     \hyphenation{Com-pre-hen-sive}
%     \hyphenation{CTAN---Com-pre-hen-sive}
% I changed    publisher = {CTAN---Comprehensive TeX Archive Network},
% to           publisher = {CTAN---Com\-pre\-hen\-sive TeX Archive Network},
% in my all-biblatex.bib file and it worked as expeceted.
% If you need to change the hyphenation points of a word in the text
% you can do, for example,
%     \hyphenation{ve-ry-od-dly-hy-phen-at-ed}

\newcommand\labelAndRemember[2]
  {\expandafter\gdef\csname labeled:#1\endcsname{#2}%
   \label{#1}#2}
\newcommand\recallLabel[1]
   {\csname labeled:#1\endcsname}
\newcommand{\recalleq}[1]{$\recallLabel{eq:#1}$}

\newcommand{\vctr}[1]{\mathbf{#1}}
\newcommand{\unitv}[1]{\hat{\mathbf{#1}}}
\newcommand{\preup}[1]{\prescript{#1}{}{}}
\newcommand{\rf}[1]{\mathcal{\MakeUppercase{#1}}}
\newcommand{\dcm}[1]{\left[\rf{#1}\right]}
\newcommand{\vrf}[2]{\preup{\rf{#1}}\vctr{#2}}

\newcommand{\matcp}[1]{\left[#1 \times\right]}
\newcommand{\rthree}[0]{$\mathbb{R}^3$\:}
\newcommand{\sthree}[0]{$\mathbb{S}^3$\:}
\newcommand{\sothree}[0]{$SO_3$}
\begin{document}

\setcounter{tocdepth}{3}

\maketitle


\ProvidesFile{ch-front.tex}[2022-10-05 front matter chapter]
%
%  This is the ``front matter'' for the thesis.
%
%  REFERENCES
%
%    TCMOS17
%      The Chicago Manual of Style Online, 17th edition.
%      https://www.chicagomanualofstyle.org/home.html
%      retrieved on 2020-02-29
%
%    TEMPL
%      Thesis and Disertation Office Templates.
%      https://www.purdue.edu/gradschool/research/thesis/templates.html
%      retrieved on 2020-02-29
%
%    WNNCD
%    Webster's Ninth New Collegiate Dictionary.
%

%
%   Only Purdue University uses this page
%
%   Comment out \begin{statement} through \end{statement}
%   if you are not at Purdue University.
%
% Statement of Thesis/Dissertation Approval Page
% This page is REQUIRED.  The page should be numbered "2"
% and should NOT be listed in your TABLE OF CONTENTS.
\begin{statement}
  % Delete or add \entry commands as needed for all committe members.
  \entry{Dr.~Carolin Frueh}{School of Aeronautics and Astronautics}
  \entry{Dr.~Kenshiro Oguri}{School of Aeronautics and Astronautics}
  \entry{Dr.~Keith LeGrand}{School of Aeronautics and Astronautics}
  % There should be one \approvedby command containing the
  % "FORM 9 THESIS FORM HEAD NAME HERE" (from TEMPL, retrieved on 2020-03-01).
  \approvedby{Dr.~Gregory Blaisdell}
\end{statement}

% Dedication page is optional.
% A name and often a message in tribute to a person or cause.
% References: WEB9 332.

% \begin{dedication}
%   To graduate students
% \end{dedication}

% Acknowledgements page is optional but most theses include
% a brief statement of appreciation or recognition of special
% assistance.

\begin{acknowledgments}
  I would like to thank my advisor, Dr.~Carolin Frueh, for her guidance and mentorship. Her feedback has guided me through undergraduate research, publishing a conference paper, applying for fellowships, and now writing this thesis. For that, I am forever grateful.

  I must thank my committee, Dr.~Oguri and Dr.~LeGrand, for their feedback and attendance at my defense. Their comments have made this thesis what it is today.

  Finally,  my friends and family have been an amazing resource throughout my research. They've helped me workshop ideas, find bugs in code, and proofread proposals. Most importantly, they've always been willing to look at my unusual plots.

  This work was supported by Katalyst Space Technologies grant number FA6451-22-P-0019, Boeing work number SSOW-BRT-Z0722-5045, and the National Defense Science and Engineering Graduate Fellowship through the Air Force Office of Scientific Research under grant number FA9550-18-1-0154.
\end{acknowledgments}

% The preface is optional.
% References: TCMOS17 1.49, WEB9 927.

% \begin{preface}
%   This is the preface.
% \end{preface}

% The Table of Contents is required.
% The Table of Contents will be automatically created for you
% using information you supply in
%     \chapter
%     \section
%     \subsection
%     \subsubsection
%     commands.
\pdfbookmark{TABLE OF CONTENTS}{Contents}
\tableofcontents

% If your thesis has tables, a list of tables is required.
% The List of Tables will be automatically created for you using
% information you supply in
%     \begin{table} ... \end{table}
% environments.
\listoftables

% If your thesis has figures, a list of figures is required.
% The List of Figures will be automatically created for you using
% information you supply in
%     \begin{figure} ... \end{figure}
% environments.
\listoffigures

% If your thesis has listings, a list of listings is required.
% The List of Listings will be automatically created for you using
% information you supply in
%     \begin{ZZlisting} ... \end{ZZlisting}
% environments.

% \ZZlistoflistings

% List of Symbols is optional.
\begin{symbols}
  $I$& irradiance in $\left[ \frac{W}{m^2} \right]$\cr
  $\check{I}$& normalized irradiance in $\left[ W \right]$\cr
  $I_0$ & Irradiance of Vega $\left[ \frac{W}{m^2} \right]$ \cr
  $m$ & Apparent magnitude $[nondim]$ \cr
  $JD$ & Julian date \cr
  $T$ & Julian centuries \cr
  $\theta_\mathrm{GMST}$ & Greenwich mean sidereal time \cr
  $\theta_{r}$ & Angular offset of the first zero of the Airy disk diffraction pattern \cr
  $C_\mathrm{airy}(\theta)$ & CCD signal amplitude due to an Airy disk diffraction pattern $[ADU]$\cr
  $k$ & Wavenumber \cr
  $r_d$ & Telescope aperture radius $[m]$ \cr
  $r_o$ & Telescope central obstruction radius $[m]$ \cr
  $d$ & Telescope aperture diameter $[m]$ \cr
  $A_\mathrm{aperture}$ & Telescope aperture area $[m^2]$ \cr
  $A_\mathrm{eff}$ & Telescope effective aperture area $[m^2]$ \cr
  $f$ & Telescope focal length $[m]$ \cr
  $\lambda$ & Wavelength $[m]$ \cr
  $FWHM$ & Full width at half maximum \cr
  $C_\mathrm{gauss}(\theta)$ & CCD signal amplitude due to a Gaussian approximation of the Airy disk $[ADU]$ \cr
  $\mathcal{ZM}(\lambda)$ & Representative zero apparent magnitude star irradiance spectrum $\left[ \frac{W}{m^2 \cdot m} \right]$ \cr
  $\textrm{QE}(\lambda)$ & Quantum efficiency spectrum $\left[ \frac{ADU}{m} \right]$ \cr
  $\textrm{ATM}(\lambda)$ & Atmospheric transmission spectrum $\left[ \frac{1}{m} \right]$ \cr
  $K_\mathrm{cd}(\lambda)$ & Luminous efficacy spectrum $\left[ \frac{lm}{W} \right]$ \cr
  $\mathcal{S}_\mathrm{int}$ & CCD ADU conversion factor $\left[ \frac{ADU}{W \cdot m^{-2} \cdot s} \right]$ \cr
  $\textrm{sun}(\lambda)$ & Solar irradiance spectrum $\left[ \frac{W}{m^2 \cdot m} \right]$ \cr
  $\mathcal{A}(\lambda)$ & Airglow radiance spectrum $\left[ \frac{W}{m^2 \cdot m \cdot sr} \right]$ \cr
  $\mathcal{A}_\mathrm{int}$ & Intermediate airglow signal $\left[ \frac{1}{s \cdot sr} \right]$ \cr
  $\theta_z$ & Zenith angle $[rad]$ \cr
  $\textrm{AM}(\theta_z)$ & Relative airmass function $[nondim]$ \cr
  $s_\mathrm{pix}$ & Telescope pixel scale $\left[ \frac{arcsec}{pix} \right]$ \cr
  $\Delta t$ & CCD integration time $[s]$ \cr
  $B_{\mathrm{poll},z}$ & Zenith light pollution brightness in magnitudes per square arcsecond \cr
  $\bar{S}_\mathrm{airglow}$ & Mean airglow signal $[ADU]$ \cr
  $\gamma$ & Solar zenith angle $[deg]$ \cr
  $B_{twi,z}$ & Zenith twilight brightness in magnitudes per square arcsecond \cr
  $\bar{S}_\mathrm{twilight}$ & Mean twilight signal $[ADU]$ \cr
  $\mathcal{Z}$ & Zero magnitude starlight signal $[\frac{ADU}{s}]$ \cr
  $\bar{S}_\mathrm{star}$ & Mean integrated starlight signal $[ADU]$ \cr
  $F_{rs}$ & Moonlight Rayleigh scattering radiance spectrum $\left[ \frac{W}{m^2 \cdot m \cdot sr} \right]$ \cr
  $F_{ms}$ & Moonlight Mie scattering radiance spectrum $\left[ \frac{W}{m^2 \cdot m \cdot sr} \right]$ \cr
  $F_{mt}$ & Total scattered moonlight radiance spectrum $\left[ \frac{W}{m^2 \cdot m \cdot sr} \right]$ \cr
  $f(\theta)$ & Lunar brightness phase function $[nondim]$ \cr
  $\bar{S}_\mathrm{moon}$ & Mean scattered moonlight signal $[ADU]$ \cr
  $\bar{S}_\mathrm{zod}$ & Mean zodiacal light signal $[ADU]$ \cr
  $\lambda_\mathrm{background}$ & Mean of background signal Poisson distribution $[ADU]$ \cr
  $\unitv{n}$ & Face outward pointing unit normal vector \cr
  $\left( v_1, v_2, v_3 \right)$ & First, second, and third vertices $v_i \in \mathbb{R}^3$ on a given triangular face \cr
  $h_i$ & Support of the $i$th face \cr
  $\vctr{E}$ & Extended Gaussian Image \cr
  $f_r$ & Bidirectional Reflectance Distribution Function \cr
  $\unitv{\ell}$ & Illumination direction unit vector \cr
  $\unitv{o}$ & Observation direction unit vector \cr
  $\unitv{h}$ & Halfway unit vector \cr
  $C_d$ & Coefficient of diffuse reflection \cr
  $C_s$ & Coefficient of specular reflection \cr
  $C_a$ & Coefficient of absorption \cr
  $n$ & Specular exponent \cr


\end{symbols}

% Abstract is required.
% Note that the information for the first paragraph of the output
% doesn't need to be input here...it is put in automatically from
% information you supplied earlier using \title, \author, \degree,
% and \majorprof.
% Reference: PU 17.
\begin{abstract}%
  Characterizing unknown space objects is an important component of robust space situational awareness. Estimating the shape of an object allows analysts to perform more accurate orbit propagation, probability of collision, and inference analysis about the object's origin. Due to the sheer distance from the camera combined with diffraction and atmospheric effects, most resident space objects of interest are unresolved when observed from the ground with electro-optical sensors. State of the art techniques for object characterization often rely on light curves --- the time history of the object's observed brightness. The brightness of the object is a function of the object's shape, material properties, attitude profile, as well as the observation geometry. The process of measuring real light curves is complex, involving the physics of the object, the sensor, and the background environment. The process of recovering shape information from brightness measurements is known as the light curve shape inversion problem. This problem is ill-posed without further assumptions: modern direct shape inversion methods require that the attitude profile and material properties of the object is known, or at least can be hypothesized. This work describes improvements to light curve simulation that faithfully model the environmental and sensor effects present in true light curves, yielding synthetic measurements with more accurate noise characteristics. Having access to more accurate light curves is important for developing and validating light curve inversion methods. This work also presents new methods for direct shape inversion for convex and nonconvex objects with realistic measurement noise. In particular, this work finds that improvements to the convex shape inversion process produce more accurate, sparser geometry in less time. The proposed nonconvex shape inversion method is effective at resolving singular large concave feature.
\end{abstract}


% \ProvidesFile{ch-introduction.tex}[Introduction]
\graphicspath{{/Users/liamrobinson/Documents/PyLightCurves/docs/build/html/_images}}

\chapter{Introduction}

Humankind has been creating space debris since the dawn of the space age \cite{esareport2022}. Early missions like Vanguard 1, launched in March of 1958, set a precedent by leaving both their satellite and the launch vehicle's upper stage in orbit, both of which are still in orbit in 2023 \cite{vanguard1}. Vanguard was launched into Low Earth Orbit (LEO) --- defined by the European Space Agency (ESA) as any orbit with an altitude below $2000$ kilometers \cite{esareport2022}. Above LEO lies Medium Earth Orbit (MEO) for orbital altitudes between $2000$ and $31570$ kilometers, and Geostationary Earth Orbit (GEO) between $35586$ and $35986$ kilometers altitude \cite{esareport2022}. Half a century of increasingly frequent launches has created a space environment cluttered with thousands of debris objects, increasing the number of serious conjunction events that may require avoidance maneuvers for large satellites in LEO to over $100$ during 2021 \cite{esareport2022}. While very few mission-ending collisions are ocurring on a yearly basis in the early 2020s, simulations predict over 200 catastrophic collisions per year if the trend in new launches and disposal practices continue \cite{esareport2022}. This uncontrolled proliferation of human-made space debris puts space operations at risk. High-profile satellite collisions like Iridium-Cosmos in 2009 have added fuel to the fire, producing shells of debris that further pollute LEO \cite{vallado4ed}. Anti-satellite tests carried out by the USA, Russia, China, and India since the 1960s see nations destroying their own satellites, projecting military strength at the cost of creating more debris \cite{vallado4ed}. Beyond LEO in Geostationary Transfer Orbit (GTO), exploding launch vehicle upper stages produce large amounts of debris \cite{esareport2022}. While higher orbits are not yet as polluted as LEO, they do not decay due to atmospheric drag, allowing debris objects to remain in the environment indefinitely \cite{vallado4ed}.

\begin{figure}[ht]
    \centering
    \includegraphics[width=\figbig]{sphx_glr_propagate_catalog_001.png}
    \caption{Public tracked catalog in 2000 and 2023}
    \label{fig:catalog_comparison}
\end{figure}

In the context of the modern space environment, determining the current state and predicting the future dynamics of space objects is critical for many areas of Space Domain Awareness (SDA) \cite{frueh2019notes}. While the current orbits of objects can be determined accurately from astrometry --- through passive optical imagery or active radar --- their future dynamics are perturbed by non-conservative forces drive by their shape, attitude profile, and material properties that cannot be observed directly. In particular, objects in orbits with altitudes higher than LEO are most efficiently observed with optical telescopes as the power required for radar scales with the square of the distance \cite{frueh2019notes}. Because optical observations are already commonly used to characterize the orbit of the objects in orbits past LEO, it is advantageous to use the same existing sensors --- or in some cases even the same images --- to extract these other useful characteristics. 

Characterizing an object's shape, attitude, and material properties is fundamentally difficult as distance from the sensor and atmospheric turbulence leaves only a distribution of brightness in the image \cite{fan2020thesis}. The leftover information is the total brightness of the object, and this value over time is known as the light curve. Optical brightness observations are fundamentally limited by background and sensor noise \cite{frueh2019notes}. Light curves are computed from observed optical data by estimating the background mean of the image, identifying which pixels of the image likely belong to the object, subtracting the mean background level from those pixels, and calibrating the remaining object signal using known stars elsewhere in the image \cite{schildknecht2008}. Each image must also be monitored for contamination from background stars and over- and under-exposures \cite{schildknecht2015}. Despite these realities, the light curve is a function of the parameters of interest: the object's shape, attitude, and material properties \cite{fan2020thesis, burton2021mapping}. Solving light curve shape inversion in a general case would enable robust active debris removal, anomaly detection, and collision avoidance, all of which are benefitted by accurate shape information.

Due to the environmetal noise and fundamental physical limitations on the processes driving the light curve, the measured brightness is dependent on the overall brightness and hence varies from data point to data point in the light curve. Furthermore, given the Poisson nature of the light collection process, a constant Gaussian assumption of the measurement noise in the light curve may not be suitable \cite{fan2020thesis, krag2003}. A realistic representation of the light curve can only be achieved by accounting for the physical processes simulating the lighting and measurement process, followed by the measurement reduction and correction processing steps. 

\section{State of the Art}

Light curve shape inversion was first investigated by Russell in 1906, who proposed a spherical harmonic representation that could be fit to an asteroid shape \cite{russell1906}. Russell noted that there would be ambiguity in the shape solution such that many solutions would fit the data equally well. The next major contribution to the field was due to Kaasalainen and Torppa in 2000, who successfully reconstructed the shapes of asteriods by directly optimizing the directions and areas of candidate faces --- encapsulated by the so-called Extended Gaussian Image (EGI) --- to find a convex shape that produces a similar light curve \cite{kaasalainen2000, kaasalainen2001}. Once the EGI is estimated, Kaasalainen and Torppa recover the vertices and faces of the corresponding convex object using a result of Minkowski and a nonlinear, convex optimization problem implemented by Little \cite{minkowski1909, little1983}. Any EGI-based method in the asteroid or human-made object shape inversion literature uses some variation of this final stage to reconstruct the final estimated geometry. Kaasalainen and Torppa also addressed nonconvex shape inversion by optimizing a spherical harmonics shape representation to reconstruct the largest nonconvex features of an asteroid, noting that smoothness regularization was sometimes needed to prevent the shape from degenerating \cite{kaasalainen2000}. In the work of Kaasalainen and Torppa, the EGI optimization takes place in a single step as the asteroid shapes under study do not have sparse EGIs. By contrast, this work introduces stages that increase shape accuracy and lower computation time by leveraging the natural sparsity of human-made objects. Durech and Kaasalainen extended on this work in 2003 by investigating the observability of nonconvex features in asteroid light curves, finding that concave features are often observable only at high phase angles, supporting the conclusion that robust nonconvex shape inversion requires very different considerations than its convex counterpart \cite{durech2003}. In 2022, Chng et al. proposed a method to determine a optimal spin pole and convex shape via the EGI, offering computational benefits over Kaasalainen and Torppa while guaranteeing global optimality in the solution with respect to the input brightness data, while being limited to convex shape estimates\cite{chng2022}. Using the methods originally proposed by Kaasalainen and Torppa, a collaborative effort of dozens of observatories lead to the publication of Database of Asteroid Models from Inversion Techniques (DAMIT), a publicly-available repository of convex asteroid models \cite{durech2010}. As of October 2023, DAMIT currently hosts 16,086 models for 10,751 asteroids \cite{damit2014}.

% Viikinkoski et al. investigated recovering large concavities from adaptive optics imagery, noting the fundamental non-uniqueness of any solution \cite{viikinkoski2017}. They discuss how a single large concavity may produce identical scattering behavior to multiple smaller concave features \cite{viikinkoski2017}. # removed due to AO not truly unresolved.  While the field of asteroid shape inversion has been alive in the intervening years, most works are not relevant to human-made objects.

Shape inversion for human-made space objects differs from the asteroid inversion in a few important aspects. More diverse methods exist, being generally segmented into EGI-based methods drawing from the asteroid shape inversion literature, filter-based methods for simultaneous attitude and shape solutions, and machine learning for classifying object shape from the light curve. Due to the increased number of unknowns in the material properties and attitude profile when observing an arbitrary human-made object, the inverted light curves are often simulated as part of the same work. This highlights the importance of realistic light curve simulation to effectively test proposed inversion methods. 

Direct shape inversion for human-made space objects was first investigated by Calef et al., who adopted Kaasalainen and Torppa's methods applied to multispectrum measurements to reduce the ambiguities of the different material properties common in human-made objects \cite{calef2006photometric}. Bradley and Axelrad also used asteroid inversion techniques to recover convex approximations of CubeSats, rocket bodies, and box-wing satellites using the inversion codes developed and released by Kaasalainen, yielding good results for rocket body-like shapes but limited success for box-wing satellites and other high area-to-mass ratio (HAMR) objects \cite{bradley2014}. The most recent major contributions to the direct shape inversion literature are due to Fan and Frueh, who inverted the shape of convex human-made objects from noisy light curves using the EGI with a multi-hypothesis scheme to reduce the ambiguity introduced by noisy measurements \cite{fan2019, fan2020thesis, fan2021}. Fan notes that full observability is crucial for successful direct shape inversion, pointing to work by Friedman and Frueh, who quantified the observability of EGI inversion to inform sensor tasking schemes \cite{friedman2020, friedman2022}. Cabrera et al. applied area regularization to Fan and Friedman's methods, achieving more accurate convex shape estimates and finding that natural constraints on the EGI area optimization renders the problem estimatable before it becomes classically observable \cite{cabrera2021}. 

Throughout the shape inversion literature, two themes are clear. Effective and efficient methods for nonconvex shape inversion for human-made objects are needed, and existing convex inversion methods have not been designed to work with realistic measurement noise. This work seeks to address both of these challenges by presenting a method for inverting large singular concave features in addition to a scheme for robustly inverting convex and nonconvex shapes with physically-based measurement noise.

Outside of the asteroid-inspired EGI methods, the literature falls into two broad categories: filter-based inversion and machine learning categorization. Each offers different advantages while imposing unique limitations. Filter-based shape inversion was been pioneered by Linares through work with various co-authors. These filter-based methods often seek to perform multiple types of object characterization simultaneously, estimating attitude and material properties in addition to shape \cite{linares2012, linares2014space, linares2018space}. Because the input data for filter-based approaches is still only unresolved brightness measurements, estimating more properties in an already ill-posed problem requires a loss of fidelity in the solution elsewhere. Often, the shape model is highly simplified to make the problem more tractable \cite{linares2012, linares2014space, linares2018space}. Linares et al. have implemented unscented Kalman filters \cite{linares2012}, multiple-model adaptive estimation algorithms \cite{linares2014space}, and adaptive Hamiltonian Markov chain Monte Carlo schemes \cite{linares2018space} which achieve good results for simple shapes, but have not been tested on complex and realistic geometries \cite{linares2018space}. In general, filter-based approaches are limited by the nonlinearity of highly specular and complex human-made objects, but require less information to run. Direct inversion methods that use the EGI require more \textit{a priori} information, but are able to deliver more accurate shape estimates.

By contrast, machine learning categorization methods indirectly recover shape information by predicting which class of objects an observed light curve belongs to. Linares and Furfaro used a deep convolutional neural network to classify novel light curves as rocket bodies, payloads, or debris, achieving good classification accuracy at the cost of uncertainty about how the model would behave for light curves collected for objects outside of its training dataset \cite{linares2016}. Other authors, including Kerr et al. and McNally et al. have adapted the architecture developed by Furfaro et al. to classify novel light curves into an extended set of object types, demonstrating that these models are flexible enough to differentiate between many object types and attitude profiles, although with higher error rates \cite{kerr2021, mcnally2021}. Allworth et al. applied transfer learning to classify real measurements using a synthetically-trained model, supporting the applicability of these approaches to operational decision-making \cite{allworth2021}.

There has also been significant work published on extracting light curves of human-made objects from real optical observations. Schildknecht et al. used color photometry to investigate isolate material properties of high area-to-mass ratio (HAMR) objects in GEO \cite{schildknecht2008}. Karpov et al. used wide-field monitoring system to collect light curves from LEO objects \cite{karpov2016}. Benson et al. collected light curves from retired GOES, Inmarsat, and Astra satellites in geosynchronous orbit to characterize their spin states \cite{benson2017}. Koshkin et al. collected light curves of TOPEX/Poseidon, among other inactive satellites, to determine their spin poles and rates \cite{koshkin2018}. Wang et al. collected light curves from GOES-8, an active GEO satellite, and simulated material properties and attitude profile to attribute peaks in its observed brightness to different parts of the spacecraft \cite{wang2018}.

The state of the art in light curve simulation differs between approaches and the object class under study. Kaasalainen and Torppa, as well Fan, Friedman, Kobayashi, and Frueh employ a Lambertian model for convex objects with a facetwise ray tracing scheme for nonconvex objects \cite{kaasalainen2001, fan2016, fan2020thesis,friedman2020,kobayashi2020,frueh2014}. This approach has the advantage of being simple, but can be computationally intensive for complex objects. Allworth et al. developed a ray traced light curve simulator in based on Blender's cycles renderer, allowing them to account for photorealistic shadowing and motion blur \cite{allworth2020, allworth2021}. Furfaro et al. \cite{furfaro2019} and Cabrera and Bradley \cite{cabrera2021,bradley2014} use a simple Lambertian model with no self-shadowing. Many more authors apply a more specialized non-Lambertian Bidirectional Reflectance Distribution Function (BRDF) for their lighting \cite{linares2018space, mcnally2021, blacketer2022}. Throughout the literature, there is a clear gap between the simulated light curves and their observed counterparts. Due to the difference in quality, authors often treat real and simulated data very differently \cite{allworth2021}. This work presents a physically-based lighting, shadowing, and noise model to produce synthetic light curves of similar quality to observed data, enabling more robust validation of the presented inversion techniques.

% \section{Contributions}

% \subsection{Simulation Advances}

% A high-fidelity light curve simulator was developed to act as a digital twin of the Purdue Optical Ground station and support inversion algorithm development. This simulator is one to four orders of magnitude faster than ray tracing-based renderers commonly used in literature \cite{fan2019, allworth2020}. It supports self-shadowing, variable material properties, a variety of reflection functions, and dynamic solar panel rotation. In concert with a constrained observer model and orbit propabation, the simulator generates realistic, noisy light curves for inactive debris, highly nonconvex objects, and actively-controlled satellites. 

% \subsection{Advances in Convex Shape Inversion}

% This work presents a suite of changes that build on the classical shape inversion algorithm for convex shapes \cite{robinson2022}. New resampling and merging steps in the Extended Gaussian Image optimization stage yield more accurate shapes that are easier to reconstruct. An alternative optimization method for the shape support vector decreases convergence time for highly symmetric objects where the classical optimization algorithm fails.

% The approach presented in this work solves the shape inversion problem beginning from the direct geometry reconstruction methods of \cite{kaasalainen2001,fan2020thesis}. The EGI optimization processes of \cite{fan2020thesis,cabrera2021,kaasalainen2001} are improved using novel resampling and merging steps. These improvements circumvent the need for the regularization terms explored by Cabrera et al. \cite{cabrera2021}.

% This convex shape inversion method has a number of general advantages. It does not require any \textit{a priori} information about the truth geometry. Unlike MMAE methods \cite{linares2014space}, the presented algorithm does not require a bank of reference models to recover shape information. Unlike deep learning methods, the presented method does not require a diverse set of training data to achieve good results \cite{furfaro2019,kerr2021}.

% \subsection{Advances in nonconvex Shape Inversion}

% While natural space objects like asteroids are largely convex, nearly all human-made space objects are highly nonconvex, highlighting the need for a robust inversion scheme for nonconvex space objects. If the object's material properties are well-known, the algorithm developed in this work is able to predict whether any single nonconvex feature is self-shadowed in the observed light curve and introduces a concavity in the shape estimate in the correct location.

\ProvidesFile{ch-meshes.tex}[Mesh chapter]

\chapter{MESHES}

A computer represents 3D objects 
% \ProvidesFile{ch-attitude-reprs.tex}[Attitude]

\chapter{Attitude}

\section{Attitude Representations}

When we talk about the orientation ---  also known as attitude ---  of a rigid body in three dimensions, that orientation is always implicitly understood to be relative to some other reference frame. The orientation of a book might be expressed using a frame fixed in the table it sits on. If that same book was sitting in an empty void, we would have no way to talk ---  or even think ---  about its orientation. Orientation itself is a three-dimensional quantity. Consider a coordinate system fixed in a rigid object and a second reference frame in which we want to express the orientation of the object. For convenience, we will call the frame fixed in the object the body frame, and the second frame the world frame. Any effective attitude representation must let us express the directions of all three body axes in terms of the world frame basis vectors. This raises an important question: how many numbers do we need to express an object's attitude? We can express the direction of any unit vector with two numbers ---  the azimuth and elevation of that vector. Naïvely, we might extrapolate from this to conclude that we will need six numbers to express an orientation. Because the basis vectors form an orthonormal set $\left\{ \hat{b}_1, \hat{b}_2, \hat{b}_3\right\}$, we know we can express $\hat{b}_3 = \hat{b}_1 \times \hat{b}_2$, $\hat{b}_2 = \hat{b}_3 \times \hat{b}_1$, and $\hat{b}_1 = \hat{b}_2 \times \hat{b}_3$. Each of these equations constrains one further degree of freedom, indicating that only three quantities are necessary to express the relative orientation of two reference frames. The most obvious parameterization for attitude is the direction cosine matrix (DCM), a $3\times3$ symmetric matrix with determinant 1. We notate the DCM with two capital letters, the rightmost indicating the reference frame of the input vectors and the leftmost indicating the transformed frame. Alternatively, the DCM is sometimes expressed as $C$ when the frames involved are arbitrary or do not need to be denoted. For example, the DCM $\dcm{bn}$ takes vectors in the $\rf{n}$ frame to the $\rf{b}$ frame:

\begin{equation}
    \vrf{b}{r} = \dcm{bn} \vrf{n}{r}
\end{equation}

The orthogonal property of the DCM implies $\dcm{bn}^{-1} = \dcm{bn}^T$ such that $\dcm{bn}^T = \dcm{nb}$. 

Another core attitude representation is the Euler angle-axis, or principal rotation parameter, form. Euler's rotation theorem guarantees that any relative orientation can be expressed as a single rotation about an axis $\hat{\lambda} \in \mathbb{S}^2$ by an angle $\theta \in [0, 2\pi]$. The set $\left\{\hat{\lambda},\theta\right\}$ is known as a principal rotation parameter, abbreviated PRP hereafter. The DCM is mapped to the PRP representation via \ref{eq:dcm_to_prp} \cite{shuster1993}.

\begin{align*} \numberthis \labelAndRemember{eq:dcm_to_prp}
    {
    \theta &= \cos^{-1}\left(\frac{1}{2} \left[C_{1,1} + C_{2,2} + C_{3,3} - 1 \right] \right) \\
    \hat{\lambda} &= \frac{1}{2\sin{\theta}} 
    \begin{bmatrix} C_{2,3} - C_{3,2} \\ C_{3,1}-C_{1,3} \\ C_{1,2} - C_{2,1}\end{bmatrix}
    }
\end{align*}

Where $C_{i,j}$ refers to the $i$th row and $j$th column of $C$. The mapping from PRP to DCM is also relatively straightforward.

\begin{equation} \labelAndRemember{eq:prp2dcm}
    {C = I_3 + \sin\theta\matcp{\hat{\lambda}} + (1-\cos\theta)\matcp{\hat{\lambda}}^2}
\end{equation}

Where $\matcp{v}$ is the matrix cross product operator, defined on $\vctr{v} \in \mathbb{R}^3$ as:

\begin{equation}
    \matcp{\vctr{v}} = \begin{bmatrix}
        0 & -v_3 & v_2 \\
        v_3 & 0 & -v_1 \\
        -v_2 & v_1 & 0
    \end{bmatrix}
\end{equation}

This operator is useful as it rephrases the cross product as matrix multiplication, i.e. $\vctr{v} \times \vctr{u} = \matcp{\vctr{v}}\vctr{u}$. While the PRP $\{\theta, \hat{\lambda}\}$ is a four element set, there are only three degrees of freedom due to the unit norm constraint on $\hat{\lambda}$. 

The quaternion represents attitude with a point on the surface of the hypersphere \sthree. In terms of the PRP, the quaternion is given by Eq \ref{eq:prp2quat}.

\begin{equation} \labelAndRemember{eq:prp2quat}
    {
    \vctr{q} = \begin{bmatrix} \hat{\lambda} \sin\left( \theta \right) \\ \cos(\theta) \end{bmatrix}
    }
\end{equation}

The first three entries of the quaternion are often called the vector component, with the final entry being the scalar component. Some authors reorder the quaternion, placing the scalar term first. Often the entries of a single quaternion are referenced by index such that $\vctr{q} = \left[ q_1, q_2, q_3, q_4 \right]$. Similarly, we can reference the vector portion of the quaterion with $\vctr{q}_{1:3}$. The quaternion can be mapped back to the PRP via Eqs \ref{eq:quat2prp_theta} and \ref{eq:quat2prp_lambda}.

\begin{align*} \numberthis \labelAndRemember{eq:quat2prp_theta} 
    {
    \theta &= \cos^{-1}\left(q_4 \right) \\
    \lambda &= \frac{\vctr{q}_{1:3}}{\sin{\theta}}
    }
\end{align*}

The quaternion maps to the DCM via Eq \ref{eq:quat2dcm}

\begin{equation} \labelAndRemember{eq:quat2dcm}
    {
        C = \left[\begin{matrix}\ -\ q_2^2-\ q_3^2+q_1^2+q_4^2\ &\ 2\ q_1q_2+2\ q_3q_4&\ 2\ q_1q_3-2\ q_2q_4\\\ 2\ q_1q_2-2\ q_3q_4&\ -\ q_1^2-\ q_3^2+q_2^2+q_4^2\ &\ 2\ q_1q_4+2\ q_2q_3\\\ 2\ q_1q_3+2\ q_2q_4&\ 2\ q_2q_3-2\ q_1q_4&\ -q_1^2-\ q_2^2+q_3^2+q_4^2\\\end{matrix}\right]=\Xi\left(q\right)^T\Psi\left(q\right)
    }
\end{equation}

In Eq \ref{eq:quat2dcm}, $\Psi$ is defined to be

\begin{equation} \label{eq:quat_psi}
    \Psi = \left[\begin{matrix}q_4&q_3&-q_2\\{-q}_3&q_4&q_1\\q_2&-q_1&q_4\\-q_1&-q_2&-q_3\\\end{matrix}\right].
\end{equation}

Multiplying the Euler angle by the axis, we get an attitude representation similar to the PRP known as the rotation vector (RV), generally denoted $\vctr{p}$. 

\begin{equation} \labelAndRemember{eq:prp2rv}
    {\vctr{p} = \theta\hat{\lambda}}
\end{equation}

The RV is the first truly three dimensional representation we have come across so far. This is advantageous for visualizing sets of orientations, but there are multiple notable issues with any three dimensional embedding of $SO(3)$. Any representation embedded in \rthree loses some of the spherical qualities of \sthree, leading to singularities ---  regions where attitudes are not uniquely defined or are impossible to compute in the first place.

To summarize, we can transform to and from all attitude representations with relatively simple algebraic operations:

\begin{table}[]
\begin{tabular}{|l|l|l|l|l|}
\cline{1-5}
\textbf{} & DCM & PRP & RV & MRP \\ \cline{1-5}
DCM       & ---     &     &    &     \\ \cline{1-5}
PRP       &  \recalleq{prp2dcm}   &  ---    &  \recalleq{prp2rv}   &     \\ \cline{1-5}
RV        &     &     &  ---   &     \\ \cline{1-5}
MRP       &     &     &    &    ---  \\ \cline{1-5}
\end{tabular}
\end{table}

\section{Attitude Kinematics}

Because it is cheap to convert between attitude representations, we only need to discuss a single set of kinematic equations for propagating a rigid body attitude profile from an initial condition. We choose the quaternion kinematic differential equations as they have no singularity and produce very smooth dynamics that are comparably easy to integrate. Given the current orientation quaternion $\vctr{q}$ and angular velocity $\vctr{\omega}$ we can compute the quaternion derivative via Eq \ref{eq:quat_kde}

\begin{equation} \label{eq:quat_kde}
    \left[\begin{matrix}\dot{\epsilon_1}\\\dot{\epsilon_2}\\\dot{\epsilon_3}\\\dot{\epsilon_4}\\\end{matrix}\right]
    =
    \frac{1}{2}\left[\begin{matrix}\epsilon_4&-\epsilon_3&\epsilon_2&\epsilon_1\\\epsilon_3&\epsilon_4&-\epsilon_1&\epsilon_2\\-\epsilon_2&\epsilon_1&\epsilon_4&\epsilon_3\\-\epsilon_1&-\epsilon_2&-\epsilon_3&\epsilon_4\\\end{matrix}\right]
    \left[\begin{matrix}\omega_1\\\omega_2\\\omega_3\\0\\\end{matrix}\right]
\end{equation} 

\section{Attitude Dynamics}

Rigid body dynamics can be easily expressed in the body principal axes with an arbitrary torque $\vctr{M} = \left[M_1, M_2, M_3\right]^T$ in the same frame via Eq \ref{eq:rbtf_dynamics}

\begin{equation} \label{eq:rbtf_dynamics}
    \left[\begin{matrix}\dot{\omega_1}\\\dot{\omega_2}\\\dot{\omega_3}\\\end{matrix}\right]
    =
    \left[\begin{matrix}
        \left(M_1+I_2\omega_2\omega_3-I_3\omega_2\omega_3\right) / I_1 \\
        \left(M_2-I_1\omega_1\omega_3+I_3\omega_1\omega_3\right) / I_2 \\
        \left(M_3+I_1\omega_1\omega_2-I_2\omega_1\omega_2\right) / I_3 \\
    \end{matrix}\right]
\end{equation}
\ProvidesFile{ch-lighting.tex}[Lighting]
\graphicspath{{../PyLightCurves/docs/source/gallery/01-light_curves/images}}

\chapter{LIGHTING}

Although light curves come from unresolved measurements, the interactions that produce them are directly driven by the shape and material properties of the object being observed. In order to simulate accurate light curves, we must model all important optical interactions. In broad terms, this boils down to determining how the object is lit and how it is shadowed. 

At the microscopic scale, the surface of an object is composed of facets -- small areas sharing a normal vector. The macroscopic optical properties of the material is driven by the distribution of sizes and normal directions of the facets. If the facets are distributed in biased orientations, the macroscopic surface may show anisotropy, leading to the appearance of brushed metal. If the facets normals are at large angles to each other, the surface may appear dull as the outgoing direction of the light may be completely independent from the incoming direction. Subsurface effects -- where incoming light rays scatter \textit{inside} the surface can also change the macroscopic properties of the material. 

This discussion raises an important question; how can we model the macroscopic outcomes of the true microscopic interactions of incident light on a surface? The bidirectional reflectance distribution function (BRDF) is a tool developed in computer graphics to address this exact problem. The BRDF is a function on the hemisphere which expresses the fraction of light per solid angle (radiance $\mathcal{R}$) leaving the surface in a given direction, divided by the incident power per unit area (irradiance $\mathcal{I}$). The general formulation for a BRDF $f_r$ is given by: TODO CITE DUVENHAGE2013

\begin{equation}
    f_r(\vctr{x}, L \rightarrow O) = \frac{d\mathcal{R}\left(\vctr{x} \rightarrow O\right)}{d\mathcal{I}\left(L \rightarrow \vctr{x}\right)}.
    \label{eq:brdf_def}
\end{equation}

In Equation \ref{eq:brdf_def}, $\vctr{x} \in \mathbb{R}^3$ is the point on the object's surface the BRDF is evaluated at. $L \in \mathbb{S}^2$ is the incoming illumination unit vector and $O \in \mathbb{S}^2$ is the outgoing unit vector. This definition is useful for building intuition about the form of the BRDF, but to represent a physically plausible reflection process, a candidate function must satisfy a few additional constraints. Assuming that the material is not luminescent, more light cannot be reflected from a unit of surface area than what hit it. This means that integrating the BRDF over the hemisphere centered around the surface normal vector should yield a value smaller or equal to one.

\begin{figure}[ht]
  \includegraphics{sphx_glr_brdf_renders_002_2_00x.png}
  \caption
  {%
    By default figures are not centered.
    This is a long caption to demonstrate that captions are single spaced.
    This is a long caption to demonstrate that captions are single spaced.%
  }
  \label{fi:not-centered}
\end{figure}

%
% This is only done if you are using BibLaTeX.
%
\makeatletter  % commented out on 2022-01-26
  \defbibenvironment{bibliography}
    {%
      \list
        {%
          \printtext[labelnumberwidth]%
          {%
            \printfield{prefixnumber}%
            \printfield{labelnumber}%
          }%
        }%
        {%
          \setlength{\bibhang}{1in} %%%%% was 0pt
          \setlength{\itemindent}{1in}%  -\leftmargin} %%%%% was 0pt
          \setlength{\itemsep}{\bibitemsep}%
          \setlength{\leftmargin}{0pt}%  .22in} % 0.42in}
          \setlength{\parsep}{\bibparsep}%
           \setlength{\rightmargin}{0.33in}%
        }%
    }
    {\endlist}
    {\item}
\makeatother  % commented out on 2022-01-26

% \immediate\setlength{\labelnumberwidth}{1.5in} %%%%% was commented out
\setlength{\labelwidth}{1.5in}
\def\sllnsez{[999] }

{%
  % Make _ in URLs visible.
  % \def\t{\char'137}%
  \catcode`*=\active
  \def*{\char'137}%  \char'137 is _
  \PrintBibliography
}

% LaTeX won't read after the \end{document} command.
% You can put notes to yourself or LaTeX input not
% ready for use after "\end{document}" if you'd like.
\end{document}
