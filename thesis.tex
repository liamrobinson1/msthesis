\ProvidesFile{thesis.tex}[2022-10-05 PurdueThesis thesis.tex file]

%
%  The home page for the PurdueThesis software is
%      https://engineering.purdue.edu/~mark/PurdueThesis/
%
%  Be sure to sign up for the PurdueThesis mailing list at
%      https://engineering.purdue.edu/ECN/mailman/listinfo/purduethesis-list
%  so you learn of new versions of this software.  You must be on that
%  mailing list to receive help with this software.
%
%  This is the template root file for an example thesis (for master's
%  degree) or dissertation (for a Ph.D.).  From now on "thesis" will
%  refer to both of these unless stated otherwise.
%
%  LaTeX systems include auxiliary programs to do bibliographies,
%  indexes, etc.  The latexmk program runs the fewest programs needed
%  to update your thesis.  latexmk runs automatically on Overleaf.  If
%  you're LaTeXing your document on a non-Overleaf system you may need
%  to run latexmk manually.
%
%  This thesis contains Feynman diagrams in the ap-physics.tex file.
%  For these to be processed correctly you must use the lualatex
%  program:
%      latexmk -lualatex thesis
%  (If your thesis doesn't have Feynman diagrams---the
%      \include{ap-physics}
%  command may be commented out by prefixing it with a
%  '%') use pdflatex instead of lualatex:
%      latexmk thesis
%
%  To make a final PDF file before you turn in your thesis do
%      latexmk -g thesis
%  This makes sure than everything is done for your final version.
%
%  References cited below:
%
%  TM2017 is short for Thesis Manual 2017:
%    A Manual for the Preparation of Graduate Theses,
%    eighth revised edition,
%    Thesis and Dissertation Office,
%    Purdue University,
%    2017,
%    revised August 30, 2017,
%    http://www.purdue.edu/gradschool/documents/thesis/graduate-thesis-manual.pdf,
%    last retrieved on May, 8, 2021.
%
%  In this file, change the example information to your information.
%

\def\ZZinstitution{Purdue University}
\def\ZZcampus{West\space Lafayette}
\def\ZZprogram{Aeronautics and Astronautics}
\def\ZZdegree{Master of Science in Aeronautics and Astronautics}
\def\ZZauthor{Liam Robinson}
\def\ZZdocument{A Thesis}
\def\ZZgraduation{December 2023}
\def\ZZinscampro{pu-wl-aae}

% title
% If you need to manually split the title,
% over several lines do, for example,
%     \def\ZZtitle{%
%       This is the First Line\\[-6pt]
%       and this is the Second Line%
%     }
\def\ZZtitle{Light Curve Simulation and Shape Inversion for Human-Made Space Objects}
% Must all be false
\def\ZZshowcolophon{false}
\def\ZZshowdiagonalline{false}
\def\ZZshowgridlines{false}
\def\ZZshowmarginlines{false}
\def\ZZshowtimestamp{false}
\def\ZZtodonotes{false}

\documentclass{PurdueThesis}
\def\ZZatinformation{}
\graphicspath{{graphics/}}
\makeatletter
  \def\input@path{{packages/}}
\makeatother
\ConfigureBibliography

%
% This is only done if you are using BibLaTeX.
%
%
% If you don't want to ignore urldate fields,
% comment out (put "%" before) the next ten lines.
%
\DeclareSourcemap
  {
    \maps[datatype=bibtex]
    {
      % Ignore "urldate = {...}" in .bib files.
      % See the first complete example on page 201 of
      %     https://mirrors.rit.edu/CTAN/macros/latex/contrib/biblatex/doc/biblatex.pdf
      \map
        {
          \step[fieldset=urldate, null]
        }
        % Enter approximate (circa) dates using, for example,
        % "year = c2020"  See
        %     https://tex.stackexchange.com/questions/224617/what-is-the-correct-way-to-handle-approximate-dates-in-biblatex
      \map[overwrite=false]
        {
          \step[fieldsource=year]
          \step[fieldset=sortyear, origfieldval, final]
          \step[fieldsource=sortyear, match={c}, replace={}]
        }
    }
  }

% To let {\bfseries\scshape text} work as expected.
% See
%     https://tex.stackexchange.com/questions/27411/small-caps-and-bold-face
\usepackage{bold-extra}


% For typesetting cryptography pseudocode, algorithms, and protocols.
% See
%     https://mirror.las.iastate.edu/tex-archive/macros/latex/contrib/cryptocode/cryptocode.pdf
\usepackage
[
  n,
  advantage,
  operators,
  sets,
  adversary,
  landau,
  probability,
  notions,
  logic,
  ff,
  mm,
  primitives,
  events,
  complexity,
  oracles,
  asymptotics,
  keys,
]
{cryptocode}

\usepackage{fancyvrb}
  \DefineShortVerb{\|}  % so "|verbatim|" will be verbatim

% For \InpuutIfFileExists.
\usepackage{filehook}

% So "_" will work in URLs when using BibTeX.
\usepackage[T1]{fontenc}

% For nlui testing.
\usepackage{listings}
\usepackage{minted}

\usepackage{cancel}


\usepackage{etoc}
\usepackage{makeidx}
  % By default \index ignores its argument.
  % This activates indexing.
  \makeindex
  % The "chapter name" for the index.
  \renewcommand{\indexname}{INDEX}

\usepackage{mathtools}

% Define \includemedia.
\usepackage{media9}

% Define \begin{multicols}{number_of_columns} ... \end{multicolumns}.
% Used in ap-text.tex.
\usepackage{multicol}

% Define \ditto.
\usepackage{pa-ditto}

% Define \FigureDash.
% \FigureDash is a dash the width of a digit in the current font.
\usepackage{pa-figure-dash}

% For PurdueThesis, PuTh, TeX, LaTeX, METAFONT, METAPOST, etc. related logos.
\usepackage{pa-logos}

% (Or maybe use isomath instead?  -mark  2021-06-20)
% Follow ISO 80000-2:2019
%     o   put e, i, j, and pi in upright font automatically
%     o   use, for example, "\di x" to get "\,mathrm{d}\/x"
% This loads
%     o   amsmath.sty (which is already loaded above)
%     o   mathtools.sty
%     o   upgreek.sty
% Load the package.
\usepackage{pa-mismath}
  % Tell mismath to put e, i, j, and pi in upright font automatically.
  \enumber
  \inumber
  \jnumber
  \pinumber
  % To typeset math italic e, i, j, and pi use
  %     \mathit e
  %     \mathit i
  %     \mathit j
  %     \itpi

% Define \MyRepeat{what}{repeat}.
% Do "what" "repeat" number of times.
\usepackage{pa-repeat}

% Define \FloatBarrier.
% \FloatBarrier process all unproccesed floats (tables, figures, etc.).
\usepackage{placeins}

% Define \hl.
% Undefine \st so soul will load without an error.
% I hope \st wasn't used for something important!
\let\st\relax
\usepackage{soul}

% Define \textcent.
\usepackage{textcomp}

% !!! This doesn't work yet, figure it out later.
% For \textprimstress.
% \usepackage{tipa}

% Needed for chapter "Graphics", section "TikZ and PGF".
\usepackage{tikz}
  % Needed to customize arrows.
  \usetikzlibrary{arrows.meta}
  % For electrical diagrams.
  % Uses the TikZ package.
  % The circuitikz name is short for "circuit TikZ".
  \usepackage{circuitikz}
  %
  \usepackage{menukeys}
  %
  % Needed for 3D TikZ stuff.
  \usetikzlibrary{3d}
  %
  % Needed for pa-typographic-conventions package.
  \usetikzlibrary{calc,shadows,shapes.misc,shapes.symbols}
  %
  % Needed for putting text along a path.
  \usetikzlibrary{decorations.text}
  %
  % Draw TikZ decorations.
  % Needed for at least the Kalman filter system model graphic.
  \usetikzlibrary{decorations.pathmorphing} % noisy shapes
  %
  % Fit shapes to coordinates.
  % Needed for at least the Kalman filter system model graphic.
  \usetikzlibrary{fit}
  %
  % Draw the background after the foreground.
  \usetikzlibrary{backgrounds}	% drawing the background after the foreground

% Needed for the Feynman diagram in ap-physics.tex.
% Tikz-feynman requires LuaLaTeX instead of pdflatex be run.
% LuaLaTeX screws up spacing in the list of figures so this
% is not loaded and LuaLaTeX should not be used.
\usepackage[compat=1.1.0]{tikz-feynman}

% The vertical space between a table heading and the table contents
% in a tabular environment.
\newcommand{\tabularspace}{\noalign{\vspace*{2pt}}}

% For \sfrac, used to do slanted fractions, similar to, e.g., 1/2,
% but 1 is small and high and 2 is small and low.
\usepackage{xfrac}


% Define \I.
% \I1 does \indent once, \I2 does \indent twice, etc.
\newcommand{\I}[1]{\MyRepeat{\indent}{#1}}

% Define \MyI.
% Typeset my input.
\long\def\MyI#1%
  {%
    {%
      \fontsize{8}{10}\tt
      \VerbatimInput
        [
          firstnumber = 1,
          numbers     = left,
          xleftmargin = 0.33in,
        ]%
        {#1}
    }%
  }

% Define \MyIO.
% Typeset my input and output.
% The input will all be on the same page.
% The output may be split over multiple pages.
\newcommand{\MyIO}
  {%
    \input{z.out}

    {%
      \fontsize{8}{10}\tt
      \VerbatimInput
        [
          firstnumber = 1,
          numbers     = left,
          xleftmargin = 0.33in,
        ]
        {z.out}
    }
    \FloatBarrier
  }

\newcommand{\NL}{\mbox{}\\}

% Print a list of files used and their version numbers in the log file.
\listfiles

\usepackage{pa-typographic-conventions}

% For the \begin{example} ... \end{example} environment
% used in ap-linguistics.tex.
\usepackage{covington}
\usepackage{slgloss}

% "CTAN---Comprehensive" did not get hyphenated and extended
% into the right margin when using BibLaTeX and the apa style.
% These did not change it:
%     \hyphenation{Com-pre-hen-sive}
%     \hyphenation{CTAN---Com-pre-hen-sive}
% I changed    publisher = {CTAN---Comprehensive TeX Archive Network},
% to           publisher = {CTAN---Com\-pre\-hen\-sive TeX Archive Network},
% in my refs.bib file and it worked as expeceted.
% If you need to change the hyphenation points of a word in the text
% you can do, for example,
%     \hyphenation{ve-ry-od-dly-hy-phen-at-ed}

\newcommand\labelAndRemember[2]
  {\expandafter\gdef\csname labeled:#1\endcsname{#2}%
   \label{#1}#2}
\newcommand\recallLabel[1]
   {\csname labeled:#1\endcsname}
\newcommand{\recalleq}[1]{$\recallLabel{eq:#1}$}

\newcommand{\vctr}[1]{\mathbf{#1}}
\newcommand{\unitv}[1]{\hat{\mathbf{#1}}}
\newcommand{\preup}[1]{\prescript{#1}{}{}}
\newcommand{\rf}[1]{\mathcal{\MakeUppercase{#1}}}
\newcommand{\dcm}[1]{\left[\rf{#1}\right]}
\newcommand{\vrf}[2]{\preup{\rf{#1}}\vctr{#2}}

\newcommand{\matcp}[1]{\left[#1 \times\right]}
\newcommand{\rthree}[0]{$\mathbb{R}^3$\:}
\newcommand{\sthree}[0]{$\mathbb{S}^3$\:}
\newcommand{\sothree}[0]{$SO_3$}
\newcommand{\pogslla}[0]{$32.900^\circ \textrm{ N}, -105.533^\circ \textrm{ W} \textrm{ }$}

\newcommand{\figbig}[0]{0.8\textwidth}
\newcommand{\figmed}[0]{0.6\textwidth}
\newcommand{\figsmall}[0]{0.4\textwidth}

\newcommand\numberthis{\addtocounter{equation}{1}\tag{\theequation}}

\DeclareMathOperator{\atantwo}{atan2}
\DeclareMathOperator{\fl}{floor}

\begin{document}

\setcounter{tocdepth}{3}

\maketitle

%%% INTRODUCTION
\ProvidesFile{ch-front.tex}[2022-10-05 front matter chapter]
%
%  This is the ``front matter'' for the thesis.
%
%  REFERENCES
%
%    TCMOS17
%      The Chicago Manual of Style Online, 17th edition.
%      https://www.chicagomanualofstyle.org/home.html
%      retrieved on 2020-02-29
%
%    TEMPL
%      Thesis and Disertation Office Templates.
%      https://www.purdue.edu/gradschool/research/thesis/templates.html
%      retrieved on 2020-02-29
%
%    WNNCD
%    Webster's Ninth New Collegiate Dictionary.
%

%
%   Only Purdue University uses this page
%
%   Comment out \begin{statement} through \end{statement}
%   if you are not at Purdue University.
%
% Statement of Thesis/Dissertation Approval Page
% This page is REQUIRED.  The page should be numbered "2"
% and should NOT be listed in your TABLE OF CONTENTS.
\begin{statement}
  % Delete or add \entry commands as needed for all committe members.
  \entry{Dr.~John Doe, Chair}{School of Aeronautics and Astronautics}
  \entry{Dr.~Jane Doe}{School of Aeronautics and Astronautics}
  \entry{Dr.~Jim Doe}{School of Aeronautics and Astronautics}
  % There should be one \approvedby command containing the
  % "FORM 9 THESIS FORM HEAD NAME HERE" (from TEMPL, retrieved on 2020-03-01).
  \approvedby{Dr.~Buck Doe}
\end{statement}

% Dedication page is optional.
% A name and often a message in tribute to a person or cause.
% References: WEB9 332.
\begin{dedication}
  To graduate students
\end{dedication}

% Acknowledgements page is optional but most theses include
% a brief statement of appreciation or recognition of special
% assistance.
\begin{acknowledgments}
  Purdue University's Engineering Computer Network
  and Graduate School helped fund \PurdueThesisLogo\ development.
\end{acknowledgments}

% The preface is optional.
% References: TCMOS17 1.49, WEB9 927.
\begin{preface}
  This is the preface.
\end{preface}

% The Table of Contents is required.
% The Table of Contents will be automatically created for you
% using information you supply in
%     \chapter
%     \section
%     \subsection
%     \subsubsection
%     commands.
\pdfbookmark{TABLE OF CONTENTS}{Contents}
\tableofcontents

% If your thesis has tables, a list of tables is required.
% The List of Tables will be automatically created for you using
% information you supply in
%     \begin{table} ... \end{table}
% environments.
\listoftables

% If your thesis has figures, a list of figures is required.
% The List of Figures will be automatically created for you using
% information you supply in
%     \begin{figure} ... \end{figure}
% environments.
\listoffigures

% If your thesis has listings, a list of listings is required.
% The List of Listings will be automatically created for you using
% information you supply in
%     \begin{ZZlisting} ... \end{ZZlisting}
% environments.
\ZZlistoflistings

% List of Symbols is optional.
\begin{symbols}
  $I$& irradiance in $\left[ \frac{W}{m^2} \right]$\cr
  $\hat{I}$& normalized irradiance in $\left[ W \right]$\cr
\end{symbols}

% Abstract is required.
% Note that the information for the first paragraph of the output
% doesn't need to be input here...it is put in automatically from
% information you supplied earlier using \title, \author, \degree,
% and \majorprof.
% Reference: PU 17.
\begin{abstract}%
  \PurdueThesisLogo\ is a \LaTeX\ document class used for
  master's bypass reports,
  master's theses,
  PhD dissertations,
  and PhD preliminary reports.
  This template demonstrates how to use \PurdueThesisLogo.

\end{abstract}

\ProvidesFile{ch-introduction.tex}[Introduction]
\graphicspath{{/Users/liamrobinson/Documents/PyLightCurves/docs/build/html/_images}}

\chapter{Introduction}

\section{Problem Statement}

Humanity has been creating space debris since the dawn of the space age. Early missions like Vanguard 1 set a precedent by leaving both the satellite and the launch vehicle upper stage in orbit, of which are still in orbit in 2023 \cite{vanguard1}. Half a century of increasingly frequent launches has created a space environment cluttered with thousands of debris objects, requiring active satellites in low Earth orbit (LEO) to perform regular avoidance maneuvers. This uncontrolled proliferation of human-made space debris puts space operations at risk. High-profile satellite collisions like Iridium-Cosmos in 2009 have added fuel to the fire, producing shells of debris that further pollute LEO \cite{vallado4ed}. Anti-satellite (ASAT) tests carried out by the USA, Russia, China, and India since the 1960s see nations destroying their own satellites, projecting military strength at the cost of creating more debris \cite{vallado4ed}.

\begin{figure}[ht]
    \centering
    \includegraphics[width=\figbig]{sphx_glr_propagate_catalog_001.png}
    \caption{Public tracked catalog in 2000 and 2023}
    \label{fig:catalog_comparison}
\end{figure}

Determining the current state and predicting the future dynamics of space objects is critical for many fields within Space Domain Awareness (SDA) \cite{frueh2019notes}. High-fidelity orbital propagation, collision avoidance, and fragmentation analysis all rely to accurate object characterization. Characterizing an object refers to estimating one or multiple unknown relevant properties. Estimating the shape of an object helps analysts characterize it, but doing so is difficult as distance and atmospheric turbulence prevents direct imaging \cite{fan2020thesis}. As a result, passive techniques for object characterization often rely on light curves --— optical brightness measurements collected over time. Light curves are particularly efficient for the task as they are inexpensive to collect and contain information about the shape, orientation, and material properties of the object that produced them \cite{fan2020thesis, burton2021mapping}. Solving light curve inversion in a general case would enable robust active debris removal, anomaly detection, and collision avoidance, all of which rely on accurate shape information.

\section{State of the Art}

Light curve shape inversion was first investigated by Russell in 1906, who proposed a spherical harmonic shape representation that could be fit to an asteroid shape \cite{russell1906}. Russell was generally skeptical of the applicability of the method, hypothesizing that there would be ambiguity in the convex shape such that many solutions would fit the data equally well. After nearly a century of relative inactivity in the field, Kaasalainen and Torppa began successfully reconstructing the shapes of asteriods by directly optimizing the directions and areas of candidate faces to produce a convex shape that produces a similar light curve \cite{kaasalainen2000, kaasalainen2001}. Kaasalainen and Torppa also addressed non-convex shape inversion, developing an optimization procedure which performed well at reconstructed the largest nonconvex features \cite{kaasalainen2000}. The authors noted that this method was highly computationally demanding. The work of Kaasalainen and Torppa was extended by Durech and Kaasalainen in 2003, who investigated the observability of nonconvex features in asteroid light curves \cite{durech2003}. The authors noted that many nonconvex features are only observable at high phase angles, where self-shadowing effects become pronounced in the light curve. They further note that uncertainty in the surface optical properties generally means that small local features cannot be recovered through light curve inversion \cite{durech2003}. Using the methods originally proposed by Kaasalainen and Torppa, a collaborative effort between observatories lead to the publication of Database of Asteroid Models from Inversion Techniques (DAMIT), a publicly-available repository of convex asteroid models \cite{damit2014}. In 2022, Chng et al.~ investigated the 

Papers in the literature usually propose methods for both light curve simulation and shape inversion. Often, the simulation method

Light curve simulation methods differ between approaches and the object class under study. Kaasalainen and Torppa employ a Lambertian model for convex objects with a facetwise ray tracing scheme for non-convex objects \cite{kaasalainen2001}. Fan, Friedman, Kobayashi, and Frueh use a nearly identical scheme for human-made objects \cite{fan2016, fan2020thesis,friedman2020,kobayashi2020,frueh2014}. Allworth et al. developed a ray traced simulator for light curves in Blender, accounting for photorealistic shadowing and motion blur \cite{allworth2020, allworth2021}. Many deep learning approaches including Furfaro et al. \cite{furfaro2019} and Cabrera and Bradley \cite{cabrera2021,bradley2014} use a simple Lambertian model with no self-shadowing. Linares and Crassidis apply a more specialized approach with a non-Lambertian Bidirectional Reflectance Distribution Function (BRDF) for lighting \cite{linares2018space}. McNally et al. use a Phong BRDF without shadowing shadowing, citing computational intensity \cite{mcnally2021}. Blacketer implemented a Cook-Torrance BRDF for lighting with a plane stacking method for self-shadowing \cite{blacketer2022}.

Methods for shape inversion fall into three major categories: Extended Gaussian Image (EGI), statistical estimation, and deep learning based methods, each approaching the problem from a different perspective.

Direct light curve inversion with the EGI uses a series of optimization problems to fit a convex shape to measurements. These methods were pioneered by Kaasalainen and Torppa for asteroids in \cite{kaas2001shape} with simultaneous attitude inversion in \cite{kaasalainen2001}. The work of Kaasalainen et al. on asteroids was extended by Chng et al. to find globally optimal spin pole and area vector solutions \cite{chng2022}. Calef et al. were early adopters of Kaasalainen and Torppa's EGI methods for human-made objects, focusing on multispectrum measurements \cite{calef2006photometric}. Bradley and Axelrad applied EGI methods to recover convex approximations of representative GEO objects \cite{bradley2014}. Fan and Frueh used the EGI with a multi-hypothesis scheme to recover human-made object shapes with measurement noise \cite{fan2019, fan2020thesis, fan2021}. Friedman quantified the observability of EGI inversion to inform sensor tasking schemes \cite{friedman2020, friedman2022}. Cabrera et al. studied the effects of applying area regularization to Fan and Friedman's methods to achieve more accurate reconstructions \cite{cabrera2021}.

A second approach leverages statistical estimation to retrieve shape information. Linares et al. applied an unscented Kalman filter to estimate attitude and convex shape simultaneously, representing shape with vertex displacement on a sphere \cite{linares2012}. Linares et al. used a Multiple-Model Adaptive Estimation (MMAE) algorithm to predict the truth geometry and attitude by comparing observations with a bank of reference objects \cite{linares2014space}. Linares and Crassidis used an Adaptive Hamiltonian Markov Chain Monte Carlo scheme to estimate shape and other characteristics simultaneously \cite{linares2018space}.

A third approach for recovering shape information from light curves relies on deep learning. Linares and Furfaro used a deep convolutional neural network to classify novel light curves as rocket bodies, payloads, or debris \cite{linares2016}. Furfaro et al. used similar methods classify novel light curves into four truth object classes \cite{furfaro2019}. Kerr et al. adapted the architecture developed by Furfaro et al. to classify object shape and size in an extended training set \cite{kerr2021}. McNally et al. investigated the use of different model training methods to identify satellites from simulated and real light curves \cite{mcnally2021}. Allworth et al. applied transfer learning to simulated and real measurements to classify object type \cite{allworth2021}.

Various unique methods have been applied to the light curve shape inversion problem. Hall et al. investigated methods for independently solving shape parameters in isolation without attitude information \cite{hall2007}. Fulcoly et al. used measurements from different sensor locations to determine shape under various attitude profiles \cite{fulcoly2012}. Yanagisawa and Kurosaki fit an analytical light curve model for a tri-axial ellipsoid to derive the shape and attitude profile of a Cosmos rocket body  \cite{yanagisawa2012}. Kobayashi used techniques from compressed sensing to recover shape information from light curves by taking advantage of shadowing geometry \cite{kobayashi2020,kobayashi2021}.

Shape inversion for non-convex objects --- mainly applied to asteroids --- has been studied by others in the past. Durech and Kaasalainen \cite{durech2003} determined a relationship between concavity size and the minimum solar phase angle where self-shadowing impacts the light curve. Viikinkoski et al. \cite{viikinkoski2017} investigated recovering large concavities from adaptive optics imagery, noting the fundamental non-uniqueness of any solution. They discuss how a single large concavity may produce identical scattering behavior to multiple smaller concave features \cite{viikinkoski2017}. Cabrera et al. \cite{cabrera2021} studied convex solutions for non-convex objects, concluding that the convex fit diverges from the true shape as the relative concavity size increases. 

TODO: complete this lit review, add new papers

\section{Contributions}

The framework detailed in this work contributes to the light curve simulation and shape inversion literature in a few notable areas. 

\subsection{Simulation Advances}

A high-fidelity light curve simulator called LightCurveEngine was developed to support inversion algorithm development. Depending on the target shape, the simulator is one to four orders of magnitude faster than ray tracing-based renderers commonly used in literature \cite{fan2019, allworth2020}. It supports self-shadowing, variable material properties, a variety of reflection functions, and dynamic solar panel rotation. In concert with a constrained observer model and orbit propabation, the engine generates realistic light curves for inactive debris, highly non-convex objects, and actively-controlled satellites. 

\subsection{Advances in Convex Shape Inversion}

This work presents a suite of changes that build on the classical shape inversion algorithm for convex shapes \cite{robinson2022}. New resampling and merging steps in the Extended Gaussian Image optimization stage yield more accurate shapes that are easier to reconstruct. An alternative optimization method for the shape support vector decreases convergence time for highly symmetric objects where the classical optimization algorithm fails.

The approach presented in this work solves the shape inversion problem beginning from the direct geometry reconstruction methods of \cite{kaasalainen2001,fan2020thesis}. The EGI optimization processes of \cite{fan2020thesis,cabrera2021,kaasalainen2001} are improved using novel resampling and merging steps. These improvements circumvent the need for the regularization terms explored by Cabrera et al. \cite{cabrera2021}. The support optimization procedure is accelerated and strengthened with a preconditioning term proposed by Nicolet et al. \cite{nicolet2021}, enabling the rapid reconstruction of more detailed convex objects than previously feasible.

This convex shape inversion method has a number of general advantages. It does not require any \textit{a priori} information about the truth geometry. Thus, unlike MMAE methods \cite{linares2014space}, The presented algorithm does not require a bank of reference models to recover shape information. Unlike deep learning methods, the presented method does not rely on the diversity of a training set to achieve realistic results \cite{furfaro2019,kerr2021}.

\subsection{Advances in Non-Convex Shape Inversion}

While natural space objects like asteroids are largely convex, nearly all human-made space objects are highly non-convex, highlighting the need for a robust inversion scheme for both convex and non-convex space objects.
\ProvidesFile{ch-literature.tex}[Literature]

\chapter{Literature}

Light curve simulation methods differ between approaches and the object class under study. Kaasalainen and Torppa employ a Lambertian model for convex objects with a facetwise ray tracing scheme for non-convex objects \cite{kaasalainen2001}. Fan, Friedman, Kobayashi, and Frueh \cite{fan2016, fan2020thesis,friedman2020,kobayashi2020,frueh2014} use a nearly identical scheme for human-made objects. Allworth et al. developed a ray traced simulator for light curves in Blender, accounting for photorealistic shadowing and motion blur \cite{allworth2020, allworth2021}. Many deep learning approaches including Furfaro et al. \cite{furfaro2019} and Cabrera and Bradley \cite{cabrera2021,bradley2014} use a simple Lambertian model with no self-shadowing. Linares and Crassidis \cite{linares2018space} apply a more specialized approach with a non-Lambertian Bidirectional Reflectance Distribution Function (BRDF) for lighting. McNally et al. \cite{mcnally2021} use a Phong BRDF without shadowing shadowing, citing computational intensity. Blacketer \cite{blacketer2022} implemented a Cook-Torrance BRDF for lighting with a plane stacking method for self-shadowing.

Methods for shape inversion fall into three major categories: Extended Gaussian Image (EGI), statistical estimation, and deep learning based methods, each approaching the problem from a different perspective.

Direct light curve inversion with the EGI uses a series of optimization problems to fit a convex shape to measurements. These methods were pioneered by Kaasalainen and Torppa for asteroids in \cite{kaasalainen2001} with simultaneous attitude inversion in \cite{kaasalainen2001}. While natural space objects like asteroids are largely convex, nearly all human-made space objects are highly non-convex, highlighting the need for a robust inversion scheme for both convex and non-convex space objects. The work of Kaasalainen et al. on asteroids was extended by Chng et al. \cite{chng2022} to find globally optimal spin pole and area vector solutions. Calef et al. \cite{calef2006photometric} were early adopters of Kaasalainen and Torppa's EGI methods for human-made objects, focusing on multispectrum measurements. Bradley and Axelrad \cite{bradley2014} applied EGI methods to recover convex approximations of representative GEO objects. Fan and Frueh \cite{fan2019, fan2020thesis, fan2021} used the EGI with a multi-hypothesis scheme to recover human-made object shapes with measurement noise. Friedman \cite{friedman2020, friedman2022} quantified the observability of EGI inversion to inform sensor tasking schemes. Cabrera et al. \cite{cabrera2021} studied the effects of area regularization on Fan and Friedman's methods to achieve more accurate reconstructions.

A second approach leverages statistical estimation to retrieve shape information. Linares et al. \cite{linares2012} applied an unscented Kalman filter to estimate attitude and convex shape simultaneously, representing shape with vertex displacement on a sphere. Linares et al. \cite{linares2014space} used a Multiple-Model Adaptive Estimation (MMAE) algorithm to predict the truth geometry and attitude by comparing observations with a bank of reference objects. Linares and Crassidis \cite{linares2018space} used an an Adaptive Hamiltonian Markov Chain Monte Carlo scheme to estimate shape and other characteristics simultaneously. 

A third approach relies on deep learning. Linares and Furfaro \cite{linares2016} used a deep convolutional neural network to classify novel light curves as rocket bodies, payloads, or debris. Furfaro et al. \cite{furfaro2019} used similar methods classify novel light curves into four truth object classes. Kerr et al. \cite{kerr2021} adapted the architecture developed by Furfaro et al. to classify object shape and size in an extended training set. McNally et al. \cite{mcnally2021} use AI and differential approaches to identify satellites from simulated and real light curves. Allworth et al. \cite{allworth2021} applied transfer learning to simulated and real measurements to classify object type.

Various other unique methods have been applied to the light curve shape inversion problem. Hall et al. \cite{hall2007} investigated methods for independently solving shape parameters in isolation without attitude information. Fulcoly et al. \cite{fulcoly2012} used measurements from different sensor locations to determine shape under various attitude profiles. Yanagisawa and Kurosaki \cite{yanagisawa2012} fit an analytical light curve model for a tri-axial ellipsoid to derive the shape and attitude profile of a Cosmos rocket body. Kobayashi applied compressed sensing to recover shape information from light curves by taking advantage of shadowing geometry \cite{kobayashi2020,kobayashi2021}.

Shape inversion for non-convex objects --- mainly asteroids --- has been studied by others in the past. Durech and Kaasalainen \cite{vdurech2003photometric} determined a relationship between concavity size and the minimum solar phase angle where self-shadowing impacts the light curve. Viikinkoski et al. \cite{viikinkoski2017} investigated recovering large concavities from adaptive optics imagery, noting the fundamental non-uniqueness of any solution. They discuss how a single large concavity may produce identical scattering behavior to multiple smaller concave features \cite{viikinkoski2017}. Cabrera et al. \cite{cabrera2021} studied convex solutions for non-convex objects, concluding that the convex fit diverges from the true shape as the relative concavity size increases. 

We approach the shape inversion problem with the foundational EGI optimization and object reconstruction methods of \cite{kaasalainen2001,fan2020thesis}. The EGI optimization processes of \cite{fan2020thesis,cabrera2021,kaasalainen2001} are improved using novel resampling and merging steps. These improvements circumvent the need for the regularization terms explored by Cabrera et al. \cite{cabrera2021}. We also address the reconstruction scaling issues present in Fan's work \cite{fan2020thesis} with an objective function proposed by Ikeuchi et al. \cite{ikeuchi1981} in place of Little's \cite{little1985}. The support optimization procedure is accelerated and strengthened with a preconditioning term proposed by Nicolet et al. \cite{nicolet2021}, enabling the rapid reconstruction of more detailed convex objects than previously feasible.

Our approach has a number of general advantages. We do not require any \textit{a priori} information about the truth geometry. Thus, unlike MMAE methods \cite{linares2014space}, we do not require a bank of reference models to recover shape information. Unlike deep learning methods, our method does not rely on the diversity of a training set to achieve realistic results \cite{furfaro2019,kerr2021}. Our light curve simulation method improves on the facetwise ray traced shadows of \cite{kaasalainen2001,fan2020thesis,frueh2014} with shadow mapping, increasing shadow fidelity per unit computation time.


%%% BACKGROUND
\ProvidesFile{ch-coordinate-systems.tex}[Coordinate Systems]
\graphicspath{{/Users/liamrobinson/Documents/PyLightCurves/docs/build/html/_images}}

\section{Time Systems}

In order to accurately predict the position of a space object and an observer, it is necessary to understand how a given time translates to the orientation of the relevant reference frames. These calculations require conversions between various time scales, the Julian date, and sidereal time.

\subsection{Time Scales}

There are a variety of scales used to measure time. What follows is a minimal treatment of each. For a more comprehensive overview, see Section 3.5 of \cite{vallado4ed}. International Atomic Time (TAI) is based on measurements from atomic clocks and is independent of astronomical effects or observations. By definition, TAI proceeds at the rate of $1$ SI second per second. Universal Time (UT0) is derived directly from observations of the apparent position of the stars. UT1 is derived from UT0 by adjusting for polar motion. UT1 is offset from TAI by $\Delta UT1$, which is a dynamic quantity that must be continually observed. Universal Coordinated Time (UTC) is a truncation of UT1 that uses an integer number of leap seconds $\Delta AT$ to stay within $0.9$ seconds of TAI. Terrestrial Time (TT) is defined by a constant offset of $TT - TAI = 32.184$ seconds from TAI and preceeding at the same rate as TAI. These time scale relations are summarized in Eq \ref{eq:time_scale_conversions}.

\begin{align*}  \label{eq:time_scale_conversions} \numberthis
  UTC &= UT1 - \Delta UT1 \\
  TAI &= UTC + \Delta AT \\
  TT &= TAI + 32.184^s \\
\end{align*}

These time scales are relevant for this research as the precise coordinate frame transformation from ITRF to the J2000.0 realization of ICRF relies on quanities expressed in UT1. Date timestamps are usually standardized to UTC, requiring the transformations in Eq \ref{eq:time_scale_conversions} for full accuracy. Figure \ref{fig:time_scales} shows the evolution of UTC, UT1, and TT with respect to TAI. Notice that $\Delta UT1$ continually changes while $\Delta AT$ is always truncated to a nearby integer.

\begin{figure}[ht]
  \centering
  \includegraphics[width=\figbig]{sphx_glr_time_systems_001.png}
  \caption{Time scales relative to TAI}
  \label{fig:time_scales}
\end{figure}

\subsection{Julian Date}

Most tasks in astrodynamics are easier when using a continuous time system. For this reason, the Julian date is adopted. This quantity is defined is the number of days elapsed since January 1, 4713 B.C., at 12:00 \cite{vallado4ed}. Given a date timestamp of the form D/M/Y h:m:s between the years of 1900 and 2100, the Julian date is computed via:

\begin{equation} \label{eq:date_to_jd}
  JD = 376Y - \fl{\left[ \frac{7Y + 7 \cdot \fl{\left(\frac{M + 9}{12} \right)}}{4} \right]}
      + \fl{\left(\frac{275M}{9}\right)} 
      + d
      + 1721013.5
      + \frac{\frac{\left(\frac{s}{60} + 60\right)}{60} + h}{24}.
\end{equation}

Note that Eq \ref{eq:date_to_jd} is always a function of the time scale used in the input, i.e., a UTC timestamp yields $JD_{UTC}$ whereas a UT1 timestamp yields $JD_{UT1}$. Another useful quantity for later time and coordinate system calculations is the number of Julian centuries since a particular epoch. The J2000.0 epoch is used unless otherwise stated, resulting in \cite{vallado4ed}:

\begin{equation} \label{eq:jd_to_t}
  T = \frac{JD - 2451545.0}{36535}.
\end{equation}

Often, more specificity is needed with respect to the time scale used in Eq \ref{eq:jd_to_t}. For example, computing $T$ with an input date in UT1 yields $T_{UT1}$ using $JD_{UT1}$, which is in turn a function a date timestamp expressed in UT1. 

\subsection{Solar and Sidereal Time}

A solar day is defined as the time required for the Sun to pass and return to an observer's meridian --- a line of constant longitude extending from the geographic south pole to the geographic north pole \cite{vallado4ed}. By contrast, a sidereal day is the time required for the stars to complete a revolution around an observer's meridian. Due to the Earth's orbit around the Sun, the sidereal day is about 4 minutes shorter than the solar day \cite{vallado4ed}. The Greenwich mean sidereal time (GMST) is computed in seconds via \cite{frueh2019notes}:

\begin{equation} \label{eq:date_to_gmst}
  \theta_{GMST} = 67310.54841
        + \left(3.15576 \cdot 10^9 + 8640184.812866 \right) T_{UT1}
        + 0.093104 T_{UT1}^2
        - 6.2 \cdot 10^{-6} T_{UT1}^3.
\end{equation}

Accounting for the variations in the inclination of the ecliptic $\epsilon$ and the the change in the equinox compared to the reference epoch $\Delta \Psi$ produces Greenwich apparent sidereal time (GAST) via \cite{frueh2019notes}:

\begin{equation} \label{eq:date_to_gast}
  \theta_{GAST} = \theta_{GMST} + \Delta \Psi \cos\epsilon.
\end{equation}

Both the inclination of the ecliptic and the difference in the equinox are computed with series expansions following the IAU 1980 theory of nutation \cite{vallado4ed}.

\section{Coordinate Systems}

A precise definition of coordinate systems is necessary for determining the position of the observer and the observed space object at a given time. The relevant conversions are summarized by a single question: how is a fixed position on the surface of the Earth transformed into a standardized inertial reference frame?

\subsection{Latitude, Longitude and Altitude}

Latitude, longitude, and altitude (LLA) are a spherical coordinates representation of position on or above the surface of the Earth. For the purposes of precise station positioning, the difference between the two types of longitude --- geocentric and geodetic --- is important. Geocentric latitude is the angle between the line from the center of mass of the Earth to the position of interest and the equatorial plane. Geodetic latitude instead measures the angle between the local ellipsoid surface normal and the equatorial plane. Geodetic latitude $\phi_{geod}$ is converted to geocentric $\phi_{geoc}$ latitude with \cite{frueh2019notes}:

\begin{equation} \label{eq:geod_to_geoc}
  \phi_{geoc} = \tan^{-1} \left((1 - f)^2 \tan\phi_{geod} \right).
\end{equation}

Additionally, the radius of the ellipsoid $r_E$ at a given geocentric latitude is necessary for later conversion, expressed by \cite{frueh2019notes}:

\begin{equation} \label{eq:rad_at_geoc}
  r_E = R_E - f \sin^2 \left( \phi_{geoc} \right).
\end{equation}

The altitude in an set of LLA coordinates needs a reference point. Different observers may be defined differently --- either relative to the approximate ellipsoidal shape of the Earth, the hypothetical mean sea level, or the surrounding terrain.

\subsubsection{Ellipsoid}

Due to Earth's equatorial bulge, it is common to model the rough shape of the Earth as an ellipsoid. In particular, the 1984 World Geodetic Survey (WGS-84) model is used throughout this work to define the shape of the Earth ellipsoid, with parameters listed in Table \ref{tb:wgs84} for use in Eqs \ref{eq:geod_to_geoc} and \ref{eq:rad_at_geoc}.

\begin{table}[ht]
  \centering
  \begin{tabular}{|l|l|}
  \hline
  \textbf{Parameter} & \textbf{Value}              \\ \hline
  Equatorial radius $R_E$             & $6378.137 \: [km]$ \\ \hline
  Flattening ratio $f$                & $1 / 298.257$      \\ \hline
  \end{tabular}
  \caption{WGS-84 ellipsoid model of the Earth \cite{vallado4ed}}
  \label{tb:wgs84}
\end{table}

These parameters are needed for the conversion from LLA to the International Terrestrial Reference Frame.

\subsubsection{Geoid}

The geoid accounts for the gravitational potential differences across the Earth's surface \cite{vallado4ed}. It is a surface of equal gravitational potential; the surface the ocean relaxes to without the influence of the wind and tides \cite{vallado4ed}. For this reason, the geoid is alternatively known as the mean sea level (MSL). The ellipsoid is a good approximation of the geoid, which deviates from the ellipsoid by less than $\approx 100$ meters at all latitudes and longitudes. The height of the geoid above the ellipsoid can be computed from a high-fidelity gravity model, but it is often more convenient to interpolate a pre-computed grid of geoid heights. Figure \ref{fig:geoid_shape} displays global geoid heights derived from the 1996 Earth Gravitational Model (EGM-96) relative to the ellipsoid.

\begin{figure}[ht]
  \centering
  \includegraphics[width=\figmed]{sphx_glr_geoid_heights_001_2_00x.png}
  \caption{EGM-96 geoid heights above the WGS-84 ellipsoid}
  \label{fig:geoid_shape}
\end{figure}

\subsubsection{Terrain}

Terrain elevation is usually the final component needed to fully define the altitude of a ground station, which is often defined relative to MSL. This work uses $30$-meter terrain tiles from the Shuttle Radar Topography Mission (SRTM). Figure \ref{fig:pogs_terrain} shows the local elevation around the Purdue Optical Ground Station using SRTM data.

\begin{figure}[ht]
  \centering
  \includegraphics[width=\figsmall]{sphx_glr_pogs_local_terrain_001.png}
  \caption{MSL elevations surrounding the Purdue Optical Ground Station}
  \label{fig:pogs_terrain}
\end{figure}

\subsubsection{Altitude Conversions}

Given an altitude relative to the terrain $a_{terrain}$, the elevation above the ellipsoid $a_{ellip}$ is given as a function of the terrain elevation above the geoid $h_{terrain}(\lambda, \phi)$ and the geoid elevation above the ellipsoid $h_{terrain}(\lambda, \phi)$ by:

\begin{equation} \label{eq:altitude_above_ellipsoid}
  a_{ellip} = a_{terrain} + h_{terrain}(\lambda, \phi_{geod}) + h_{geoid}(\lambda, \phi_{geod}).
\end{equation}

\subsection{International Terrestrial Reference Frame (ITRF)}

The cartesian form of LLA is known as the Earth-centered Earth-fixed (ECEF) reference frame. Throughout this work, ECEF and ITRF will be used interchangeably. This frame has its origin at the center of mass of the Earth and its axes fixed in the crust. The
fundamental plane of the frame is defined to be the equator ---  orienting the $z$-axis through Earth's
instantaneous spin axis, and the reference direction through the intersection of the prime meridian
and the equator ---  defining the $x$-axis. Completing the right-handed system with $\hat{y} = \hat{z} \times \hat{x}$ yields a
reference frame that remains fixed, neglecting effects like continental drift. The transformation from LLA $\left( \lambda, \phi_{geod}, a_{ellip} \right)$ to ITRF is given by Eq \ref{eq:lla_to_itrf}.

\begin{align*} \numberthis \label{eq:lla_to_itrf}
  e^2 &= 2f - f^2 \\
  N &= \frac{R_E}{\sqrt(1 - e^2 \sin(\phi_{geod})^2)} \\
  \rho &= (N + a_{ellip}) \cos(\phi_{geod}) \\
  x_{itrf} &= \rho \cos(\lambda) \\
  y_{itrf} &= \rho \sin(\lambda) \\
  z_{itrf} &= \left(N (1 - e^2) + a_{ellip} \right) \sin(\phi_{geod}) \\
\end{align*}

In Eq \ref{eq:lla_to_itrf}, $e^2$ is the squared eccentricity of the ellipsoid, $N$ is the radius of curvature in the meridian, and $\rho$ is the $x-y$ plane magnitude of the station's position \cite{vallado4ed}.

Many later transformations require the body axis rotation matrices $R_1$, $R_2$, and $R_3$ which are expressed in Eq \ref{eq:body_rotms}.

\begin{align*} \numberthis \label{eq:body_rotms}
  R_1(\theta) &= \begin{bmatrix}  1 & 0 & 0 \\ 0 & \cos\theta & \sin\theta \\ 0 & -\sin\theta & \cos\theta \end{bmatrix} \\
  R_2(\theta) &= \begin{bmatrix}  \cos\theta & 0 & -\sin\theta \\ 0 & 1 & 0 \\ \sin\theta & 0 & \cos\theta \end{bmatrix} \\
  R_3(\theta) &= \begin{bmatrix}  \cos\theta & \sin\theta & 0 \\ -\sin\theta & \cos\theta & 0 \\ 0 & 0 & 1 \end{bmatrix} \\
\end{align*}

\subsection{Topocentric Reference Frame (ENU)}

The remaining transformations in this chapter will only be defined in terms of their rotation matrices. It is often useful to express observations in a local reference frame. The East North Up (ENU) coordinate system is used throughout this work. This system has an origin at the observing station, with the first two basis vectors pointing towards the local East and North and the third pointing towards zenith. The transformation from ITRF to ENU is given by \cite{frueh2019notes}:

\begin{equation} \label{eq:itrf_to_enu}
  \vec{r}_{enu} = F_2 F_1 R_2(\phi_{geoc}) R_3(\lambda) \vec{r}_{itrf}.
\end{equation}

In Eq \ref{eq:itrf_to_enu}, $R_3$ is a rotation about the third body axis, $F_1$ swaps the second and third unit vectors, and $F_2$ swaps the first and third unit vectors. The orientation of the ENU reference frame at the Purdue Optical Ground Station is depicted in Figure \ref{fig:pogs_enu}.

\begin{figure}[ht]
  \centering
  \includegraphics[width=\figmed]{sphx_glr_az_el_parallel_001.png}
  \caption{ENU reference frame orientation at Purdue Optical Ground Station}
  \label{fig:pogs_enu}
\end{figure}

\subsection{International Celestial Reference Frame (ICRF/J2000)}

Transforming from ITRF to the a standardized intertial reference frame is an involved process due to the variety of nonlinear effects impacting the Earth's rotational motion. In total, this transformation must account for polar motion, the nutation and precession of the Earth's pole, and the mean sidereal time. These transformations are treated much more thoroughly in Vallado \cite{vallado4ed}. 

Accounting for polar motion --- the motion of the Earth's pole that cannot be explained through nutation theory --- transforms from ITRF to Greenwich True of Date (GTOD) via: 
\begin{equation} \label{eq:itrf_to_gtod}
  \vec{r}_{gtod} = R_1(y_p) R_2(x_p) \vec{r}_{itrf},
\end{equation}

where $x_p$ and $y_p$ are the angular components of the polar motion at the time of interest \cite{frueh2019notes}. Accounting for the sidereal rotation of the Earth about its pole transforms from GTOD to the True Equator, Mean Equinox (TEME) reference frame via \cite{frueh2019notes}:

\begin{equation} \label{eq:gtod_to_teme}
  \vec{r}_{teme} = R_3(-\theta_{GMST}) \vec{r}_{gtod}.
\end{equation}

Accounting for the difference between GMST and GAST at the date of interest transforms from TEME to the True of Date (TOD) reference frame via \cite{vallado4ed}:

\begin{equation} \label{eq:teme_to_tod}
  \vec{r}_{tod} = R_3(-\Delta \Psi \cos \epsilon) \vec{r}_{teme}.
\end{equation}

Accounting for the nutation of Earth's pole transforms from TOD to the Mean of Date (MOD) reference frame via:

\begin{equation} \label{eq:tod_to_mod}
  \vec{r}_{mod} = R_1(-\bar{\epsilon}) R_3(\Delta\Psi) R_1(\bar{\epsilon} + \Delta\epsilon) \vec{r}_{tod},
\end{equation}

where $\bar{\epsilon}$ is the mean inclination of the ecliptic at the time of interest, and $\epsilon$ is the true inclination of the ecliptic \cite{vallado4ed}.

Accounting for the secular precession of Earth's pole transforms from MOD to ICRF via:

\begin{equation} \label{eq:mod_to_icrf}
  \vec{r}_{icrd} = R_3(\zeta) R_2(\theta) R_3(z) \vec{r}_{mod},
\end{equation}

through the 3-2-3 Euler angle sequence $\left( z, \theta, \zeta \right)$, which are each a function of the date of the transformation \cite{frueh2019notes}.

The specific realization of ICRF used in this work is referenced to the position of the equator and equinox at the J2000.0 epoch (January 1, 2000 12:00:00.000 TT), leading to the common name for this reference frame, "J2000" \cite{vallado4ed}.

\subsection{Right Ascension and Declination}

Right ascension and declination, often shortened to RA/Dec, are useful angles from describing the angular position of an object on the celestial sphere from the perspective of an observer. Right ascension is defined as the angle
of the observation projected onto the inertial $x-y$ plane, measured counterclockwise from inertial
$\hat{x}$, represented by $\alpha$. Declination is the angle from the $x-y$ plane to the observation
with positive values above the $x-y$ plane (closer to inertial $z$) and negative values below.
Declination is represented by $\delta$. Given a unit vector direction $\hat{v} = \left[ x_{ITRF}, y_{ITRF}, z_{ITRF} \right]^T$ in J2000, RA/Dec is computed via \cite{frueh2019notes}:

\begin{equation} \label{eq:eci_to_ra_dec}
  \begin{bmatrix}
	\alpha \\
	\delta
  \end{bmatrix} = 
  \begin{bmatrix}
	\atantwo(y_{ITRF}, x_{ITRF}) \\
	\atantwo(z_{ITRF}, \sqrt{x_{ITRF}^2 + y_{ITRF}^2})
  \end{bmatrix}.
\end{equation}

\subsection{Azimuth and Elevation}

Azimuth and elevation, often shortened to Az/El, are similar angular quantities to right ascension and declination \cite{frueh2019notes}. Instead of being based on
the inertial sphere, they are referenced to an arbitrary reference frame. For a telescope making
observations of an object, the local topocentric ENU frame may be used. For a satellite star
tracker, star azimuth and elevation might be reported in the satellite body frame. In either case,
Eq \ref{eq:eci_to_ra_dec} can be repurposed in terms of Az/El, where $\hat{v} = \left[ x_{ENU}, y_{ENU}, z_{ENU}
\right]^T$ is expressed in the frame of interest \cite{frueh2019notes}.

\begin{equation} \label{eq:enu_to_az_el}
  \begin{bmatrix}
	Az \\
	El
  \end{bmatrix} = 
  \begin{bmatrix}
	\atantwo(y_{ENU}, x_{ENU}) \\
	\atantwo(z_{ENU}, \sqrt{x_{ENU}^2 + y_{ENU}^2})
  \end{bmatrix}
\end{equation}

Note that Eq \ref{eq:enu_to_az_el} references azimuth to the $x$-axis, proceeding in the
counterclockwise direction. Often, this reference axis and direction may be changed depending on the
reference frame being used. For example, ground station observations may be referenced to local
North ---  the second axis of the ENU system ---  proceeding clockwise. This would require the
substitution $Az' = \frac{\pi}{2} - Az$. Notice that this substitution leads to $Az'$ leaking
outside the domain of $[0, 2\pi)$. This is not an issue for later coordinate transformations, but
may be undesirable for plots. Wrapping the result back to the standard azimuth range via
$Az_{wrapped} = \textrm{mod}(Az, 2\pi)$ is a sufficient fix.

\subsection{Coordinate Transformations Summary}

With these transformations, any topocentric observation directions in RA/Dec or Az/El can be converted into a Cartesian representation and transformed into J2000. Simultaneously, any propagated space object orbits can be transformed similarly from an arbitrary propagation frame such that all future computations requiring the observation geometry take place in the same reference frame. 
\ProvidesFile{ch-attitude-reprs.tex}[Attitude]

\chapter{Attitude}

\section{Attitude Representations}

When we talk about the orientation ---  also known as attitude ---  of a rigid body in three dimensions, that orientation is always implicitly understood to be relative to some other reference frame. The orientation of a book might be expressed using a frame fixed in the table it sits on. If that same book was sitting in an empty void, we would have no way to talk ---  or even think ---  about its orientation. Orientation itself is a three-dimensional quantity. Consider a coordinate system fixed in a rigid object and a second reference frame in which we want to express the orientation of the object. For convenience, we will call the frame fixed in the object the body frame, and the second frame the world frame. Any effective attitude representation must let us express the directions of all three body axes in terms of the world frame basis vectors. This raises an important question: how many numbers do we need to express an object's attitude? We can express the direction of any unit vector with two numbers ---  the azimuth and elevation of that vector. Naïvely, we might extrapolate from this to conclude that we will need six numbers to express an orientation. Because the basis vectors form an orthonormal set $\left\{ \hat{b}_1, \hat{b}_2, \hat{b}_3\right\}$, we know we can express $\hat{b}_3 = \hat{b}_1 \times \hat{b}_2$, $\hat{b}_2 = \hat{b}_3 \times \hat{b}_1$, and $\hat{b}_1 = \hat{b}_2 \times \hat{b}_3$. Each of these equations constrains one further degree of freedom, indicating that only three quantities are necessary to express the relative orientation of two reference frames. The most obvious parameterization for attitude is the direction cosine matrix (DCM), a $3\times3$ symmetric matrix with determinant 1. We notate the DCM with two capital letters, the rightmost indicating the reference frame of the input vectors and the leftmost indicating the transformed frame. Alternatively, the DCM is sometimes expressed as $C$ when the frames involved are arbitrary or do not need to be denoted. For example, the DCM $\dcm{bn}$ takes vectors in the $\rf{n}$ frame to the $\rf{b}$ frame:

\begin{equation}
    \vrf{b}{r} = \dcm{bn} \vrf{n}{r}
\end{equation}

The orthogonal property of the DCM implies $\dcm{bn}^{-1} = \dcm{bn}^T$ such that $\dcm{bn}^T = \dcm{nb}$. 

Another core attitude representation is the Euler angle-axis, or principal rotation parameter, form. Euler's rotation theorem guarantees that any relative orientation can be expressed as a single rotation about an axis $\hat{\lambda} \in \mathbb{S}^2$ by an angle $\theta \in [0, 2\pi]$. The set $\left\{\hat{\lambda},\theta\right\}$ is known as a principal rotation parameter, abbreviated PRP hereafter. The DCM is mapped to the PRP representation via \ref{eq:dcm_to_prp} \cite{shuster1993}.

\begin{align*} \numberthis \labelAndRemember{eq:dcm_to_prp}
    {
    \theta &= \cos^{-1}\left(\frac{1}{2} \left[C_{1,1} + C_{2,2} + C_{3,3} - 1 \right] \right) \\
    \hat{\lambda} &= \frac{1}{2\sin{\theta}} 
    \begin{bmatrix} C_{2,3} - C_{3,2} \\ C_{3,1}-C_{1,3} \\ C_{1,2} - C_{2,1}\end{bmatrix}
    }
\end{align*}

Where $C_{i,j}$ refers to the $i$th row and $j$th column of $C$. The mapping from PRP to DCM is also relatively straightforward.

\begin{equation} \labelAndRemember{eq:prp2dcm}
    {C = I_3 + \sin\theta\matcp{\hat{\lambda}} + (1-\cos\theta)\matcp{\hat{\lambda}}^2}
\end{equation}

Where $\matcp{v}$ is the matrix cross product operator, defined on $\vctr{v} \in \mathbb{R}^3$ as:

\begin{equation}
    \matcp{\vctr{v}} = \begin{bmatrix}
        0 & -v_3 & v_2 \\
        v_3 & 0 & -v_1 \\
        -v_2 & v_1 & 0
    \end{bmatrix}
\end{equation}

This operator is useful as it rephrases the cross product as matrix multiplication, i.e. $\vctr{v} \times \vctr{u} = \matcp{\vctr{v}}\vctr{u}$. While the PRP $\{\theta, \hat{\lambda}\}$ is a four element set, there are only three degrees of freedom due to the unit norm constraint on $\hat{\lambda}$. 

The quaternion represents attitude with a point on the surface of the hypersphere \sthree. In terms of the PRP, the quaternion is given by Eq \ref{eq:prp2quat}.

\begin{equation} \labelAndRemember{eq:prp2quat}
    {
    \vctr{q} = \begin{bmatrix} \hat{\lambda} \sin\left( \theta \right) \\ \cos(\theta) \end{bmatrix}
    }
\end{equation}

The first three entries of the quaternion are often called the vector component, with the final entry being the scalar component. Some authors reorder the quaternion, placing the scalar term first. Often the entries of a single quaternion are referenced by index such that $\vctr{q} = \left[ q_1, q_2, q_3, q_4 \right]$. Similarly, we can reference the vector portion of the quaterion with $\vctr{q}_{1:3}$. The quaternion can be mapped back to the PRP via Eqs \ref{eq:quat2prp_theta} and \ref{eq:quat2prp_lambda}.

\begin{align*} \numberthis \labelAndRemember{eq:quat2prp_theta} 
    {
    \theta &= \cos^{-1}\left(q_4 \right) \\
    \lambda &= \frac{\vctr{q}_{1:3}}{\sin{\theta}}
    }
\end{align*}

The quaternion maps to the DCM via Eq \ref{eq:quat2dcm}

\begin{equation} \labelAndRemember{eq:quat2dcm}
    {
        C = \left[\begin{matrix}\ -\ q_2^2-\ q_3^2+q_1^2+q_4^2\ &\ 2\ q_1q_2+2\ q_3q_4&\ 2\ q_1q_3-2\ q_2q_4\\\ 2\ q_1q_2-2\ q_3q_4&\ -\ q_1^2-\ q_3^2+q_2^2+q_4^2\ &\ 2\ q_1q_4+2\ q_2q_3\\\ 2\ q_1q_3+2\ q_2q_4&\ 2\ q_2q_3-2\ q_1q_4&\ -q_1^2-\ q_2^2+q_3^2+q_4^2\\\end{matrix}\right]=\Xi\left(q\right)^T\Psi\left(q\right)
    }
\end{equation}

In Eq \ref{eq:quat2dcm}, $\Psi$ is defined to be

\begin{equation} \label{eq:quat_psi}
    \Psi = \left[\begin{matrix}q_4&q_3&-q_2\\{-q}_3&q_4&q_1\\q_2&-q_1&q_4\\-q_1&-q_2&-q_3\\\end{matrix}\right].
\end{equation}

Multiplying the Euler angle by the axis, we get an attitude representation similar to the PRP known as the rotation vector (RV), generally denoted $\vctr{p}$. 

\begin{equation} \labelAndRemember{eq:prp2rv}
    {\vctr{p} = \theta\hat{\lambda}}
\end{equation}

The RV is the first truly three dimensional representation we have come across so far. This is advantageous for visualizing sets of orientations, but there are multiple notable issues with any three dimensional embedding of $SO(3)$. Any representation embedded in \rthree loses some of the spherical qualities of \sthree, leading to singularities ---  regions where attitudes are not uniquely defined or are impossible to compute in the first place.

To summarize, we can transform to and from all attitude representations with relatively simple algebraic operations:

\begin{table}[]
\begin{tabular}{|l|l|l|l|l|}
\cline{1-5}
\textbf{} & DCM & PRP & RV & MRP \\ \cline{1-5}
DCM       & ---     &     &    &     \\ \cline{1-5}
PRP       &  \recalleq{prp2dcm}   &  ---    &  \recalleq{prp2rv}   &     \\ \cline{1-5}
RV        &     &     &  ---   &     \\ \cline{1-5}
MRP       &     &     &    &    ---  \\ \cline{1-5}
\end{tabular}
\end{table}

\section{Attitude Kinematics}

Because it is cheap to convert between attitude representations, we only need to discuss a single set of kinematic equations for propagating a rigid body attitude profile from an initial condition. We choose the quaternion kinematic differential equations as they have no singularity and produce very smooth dynamics that are comparably easy to integrate. Given the current orientation quaternion $\vctr{q}$ and angular velocity $\vctr{\omega}$ we can compute the quaternion derivative via Eq \ref{eq:quat_kde}

\begin{equation} \label{eq:quat_kde}
    \left[\begin{matrix}\dot{\epsilon_1}\\\dot{\epsilon_2}\\\dot{\epsilon_3}\\\dot{\epsilon_4}\\\end{matrix}\right]
    =
    \frac{1}{2}\left[\begin{matrix}\epsilon_4&-\epsilon_3&\epsilon_2&\epsilon_1\\\epsilon_3&\epsilon_4&-\epsilon_1&\epsilon_2\\-\epsilon_2&\epsilon_1&\epsilon_4&\epsilon_3\\-\epsilon_1&-\epsilon_2&-\epsilon_3&\epsilon_4\\\end{matrix}\right]
    \left[\begin{matrix}\omega_1\\\omega_2\\\omega_3\\0\\\end{matrix}\right]
\end{equation} 

\section{Attitude Dynamics}

Rigid body dynamics can be easily expressed in the body principal axes with an arbitrary torque $\vctr{M} = \left[M_1, M_2, M_3\right]^T$ in the same frame via Eq \ref{eq:rbtf_dynamics}

\begin{equation} \label{eq:rbtf_dynamics}
    \left[\begin{matrix}\dot{\omega_1}\\\dot{\omega_2}\\\dot{\omega_3}\\\end{matrix}\right]
    =
    \left[\begin{matrix}
        \left(M_1+I_2\omega_2\omega_3-I_3\omega_2\omega_3\right) / I_1 \\
        \left(M_2-I_1\omega_1\omega_3+I_3\omega_1\omega_3\right) / I_2 \\
        \left(M_3+I_1\omega_1\omega_2-I_2\omega_1\omega_2\right) / I_3 \\
    \end{matrix}\right]
\end{equation}
\ProvidesFile{ch-photometry.tex}[Photometry]
\graphicspath{{/Users/liamrobinson/Documents/PyLightCurves/docs/build/html/_images}}

\chapter{Photometry}

\section{Astronomical Spectra}

Four of the quantities needed for the background model vary with wavelength. These are the atmospheric
transmission, the sensor quantum efficiency, the irradiance of a 0th magnitude star, and the solar spectrum. Each spectrum is displayed in Figure \ref{fig:spectra}.

\begin{figure}[ht]
  \centering
  \includegraphics[width=\figmed]{sphx_glr_astro_spectra_001_2_00x.png}
  \caption{Astronomical Spectra from \cite{krag2003}}
  \label{fig:spectra}
\end{figure}

In practice, the quantum efficiency curve varies by sensor and the thermal conditions of the
observation. The curve adopted in this work is that used by Krag; modern sensors will often
perform better.

\section{Brightness Units}

In the context of photometry, "brightness" is a catch-all term for a variety of units. Let's explore the relationships between these units to make later conversions more clear.

\subsection{Irradiance}

Irradiance is the standard SI linear unit used to describe the total amount of energy incident on a
surface from a given source. An irradiance of $1 \: \left[ \frac{W}{m^2} \right]$ implies that a $10
\: [m]$ area would experience $10 \: [W]$ of incident power. The Sun's irradiance at a distance of $1$ AU is known as the solar constant and is approximately $1361 \: \left[ \frac{W}{m^2} \right]$. 

Visual magnitude ---  also known as apparent or relative magnitude --- is a reverse logarithmic scale
that originates in astronomy \cite{frueh2019notes}. Stellar sources span many orders of magnitude of brightness, making a
logarithmic scale a helpful middle ground for comparison. Note that apparent magnitude always
expresses brightness at the observer's location; absolute magnitude is a different quantity that
normalizes brightness from a distance of $10$ parsecs \cite{frueh2019notes}. Apparent magnitude $m$
is computed from irradiance via Eq \ref{eq:irradiance_to_mag}.

\begin{equation} \label{eq:irradiance_to_mag}
  m = -2.5 \log_{10}\left( \frac{I}{I_0} \right)
\end{equation}

In Eq \ref{eq:irradiance_to_mag}, $I$ is the irradiance of the source of interest and $I_0$ is
irradiance of the zero-point source. This makes sense; substituting $I = I_0$ returns
$m=0$. The star Vega is usually taken to be the zero-point with irradiance $I_0 = 2.518021002\cdot
10^{-8} \: \left[ \frac{W}{m^2} \right]$ \cite{frueh2019notes}.

We can rearrange Eq \ref{eq:irradiance_to_mag} to compute irradiance from a given apparent magnitude,
yielding Eq \ref{eq:mag_to_irradiance}.

\begin{equation} \label{eq:mag_to_irradiance}
  I = I_0 \cdot 10^{-\frac{m}{2.5}}
\end{equation}

\subsection{Normalized Irradiance}

The light curve simulation methods presented in this work heavily use normalized irradiance, 
the irradiance of a source observed from a distance of $1$ meter. This is a non-standard quantity in the literature, but proves useful for the same reasons
absolute magnitude is used by astronomers. Adjusting sources to be at a standard distance allows us
to simulate and invert light curves in a non-dimensionalized space. This simplifies simulation and
makes the shape inversion optimizations more robust. To make the conversion explicit, irradiance observed
at a distance $r$ in meters from an object is converted to normalized irradiance $\hat{I}$ in watts via Eq
\ref{eq:irradiance_to_norm_irradiance}.

\begin{equation} \label{eq:irradiance_to_norm_irradiance}
  \hat{I} = r^2 I
\end{equation}

\subsection{$S_{10}$}

While apparent magnitude and irradiance are effective for quantifying the flux of point sources, other units exist
to describe diffuse or extended sources where brightness is spread over an
area. $S_{10}$ is a unit of surface brightness representing the number of 10th magnitude stars per square degree that would produce the same flux as a given diffuse source.
Surface brightness in $S_{10}$ over a given solid angle $\Omega \: \left[ sr \right]$ can be converted to total irradiance $I \: \left[ \frac{W}{m^2} \right]$ via Eq \ref{eq:s10toirrad}.

\begin{equation} \label{eq:s10toirrad}
 \frac{I \left[ \frac{W}{m^2} \right]}{S_{10}} = 10^{-10/2.5} \left( \Omega \frac{180^2}{\pi^2} \right)
  \int_{10^{-8}}^{10^{-6}}{ \textrm{STRINT}(\lambda) \: d\lambda} = 8.26617 \Omega \cdot 10^{-9}
\end{equation}

In \ref{eq:s10toirrad}, $\textrm{STRINT}(\lambda) \: \left[ \frac{W}{m^2 \cdot m} \right]$ is the
representative spectrum of a 0th magnitude star, $\textrm{QE}(\lambda)$ is the quantum efficiency
spectrum of the observing sensor, $\textrm{ATM}(\lambda)$ is the atmospheric transmission spectrum, $\lambda \: [m]$ is wavelength, $h \: \left[
\frac{m^2 \cdot kg}{s} \right]$ is Plank's constant, and $c \: \left[ \frac{m}{s^2} \right]$ is the
speed of light in vacuum. Quantum efficiency has units of photoelectrons which conveys the fraction of incident photons which are (proportionally) converted to photoelectrons in the CCD sensor. Atmospheric transmission is a unitless quantity conveying the fraction of light that is not absorbed by the atmosphere. Example spectra for $\textrm{ATM}(\lambda)$ and $\textrm{QE}(\lambda)$ are displayed in Figure \ref{fig:spectra}, with underlying data provided in Appendices \ref{data:atm} and \ref{data:qe}.

\subsection{Magnitude per Square Arcsecond}

A second surface brightness unit is $\left[ \frac{mag}{arcsec^2} \right]$, also known as MPSAS (magnitude per square arcsecond). This quantity can be thought of as a generalized $S_{10}$, where instead of quantifying the number of stars of a certain
magnitude in a solid angle, the equivalent magnitude of a single point source is measured. A surface
brightness $B_{10}$ in $S_{10}$ can be converted into surface brightness $B_{mag}$ in 
$\left[ \frac{mag}{arcsec^2} \right]$ via Eq \ref{s10_to_mag_per_a2}.

\begin{equation} \label{s10_to_mag_per_a2}
	B_{mag} = -2.5 \log_{10}\left( \frac{B_{10} \cdot 10^{-4}}{12960000} \right)
\end{equation}

In Eq \ref{s10_to_mag_per_a2} $S_{10}$ is first converted to the total irradiance per square degree,
convert square degrees to square arcseconds, and transform the result back into apparent magnitude. MPSAS is converted to irradiance per steradian via Eq \ref{eq:mpsas_to_irrad_per_ster} using \ref{eq:mag_to_irradiance}.

\begin{equation} \label{eq:mpsas_to_irrad_per_ster}
  I = \left( \frac{180}{ 3600\pi} \right)^2 I_0 \cdot 10^{-\frac{MPSAS}{2.5}}
\end{equation}

\subsection{Candela} \label{sec:candela}

Some light pollution datasets are given in units that include candela. Candela is the SI base unit of luminous intensity defined by the International Committee for Weights and Measures as "Fixing the numerical value of the luminous efficacy of monochromatic radiation of frequency $540\cdot10^{12}$ Hz to be equal to exactly $683$" \cite{nist_units}. This means that an isotropic green light source with frequency $540\cdot10^{12}$ Hz ($\lambda = 555$ nm) has a luminous efficacy of $K_{cd} = 683 \: \left[ lm/W \right]$ where lm stands for lumens. Luminous efficacy itself determines how well a source produces visible light. For a given wavelength, candela $B_{cd}$ is converted to watts per steradian $B_{wsr}$ via Eq \ref{eq:cd_to_w_per_sr} \cite{nist_units}

\begin{equation} \label{eq:cd_to_w_per_sr}
  B_{wsr}(\lambda) = \frac{B_{cd}}{K_{cd}(\lambda)}.
\end{equation}

The luminous efficiency function $K_{cd}(\lambda)$ models the human eye's response to the visible spectrum \cite{sharpe2005}. Different fits of this function exist; the function proposed Sharpe et al. is adopted, displayed in Figure \ref{fig:luminous_efficiency} \cite{sharpe2005}.

\begin{figure}[ht]
  \centering
  \includegraphics[width=\figmed]{sphx_glr_luminous_efficiency_001_2_00x.png}
  \caption{Luminous effiency function from \cite{sharpe2005}}
  \label{fig:luminous_efficiency}
\end{figure}

Candela per unit area can be converted into MPSAS by combining Eq \ref{eq:cd_to_w_per_sr} with \ref{eq:irradiance_to_mag}, yielding Eq \ref{eq:cd_per_m2_to_mpsas}, which is still a function of the source's wavelength.

\begin{equation} \label{eq:cd_per_m2_to_mpsas}
  MPSAS(\lambda) = -2.5 \log_{10}\left( \frac{B_{cd}}{\left( \frac{180}{ 3600\pi} \right)^2 K_{cd}(\lambda) I_0} \right)
\end{equation}

\subsection{Photoelectron Counts}

Raw images taken by a CCD-equipped telescope have pixel values measured in photoelectron
counts, otherwise known as Analog-to-Digital Units (ADU) \cite{krag2003}. The count in a single
pixel obtained is directly proportional (via the CCD's gain) to the number of
photons incident on that pixel during the integration time. Higher order effects in the silicon of
the CCD makes this description incomplete, but for non-resolved imaging applications
concerned about, effects smaller than the sensor readout noise and dark current can be safely neglected
\cite{frueh2019notes}. Irradiance can be converted to ADU via the conversion factor $SINT$
in Eq \ref{eq:sint} \cite{krag2003}.

\begin{equation} \label{eq:sint}
 \textrm{SINT} = \frac{\pi D^2}{4}
	\int_{10^{-8}}^{10^{-6}}{ \left( \frac{\textrm{SUN}(\lambda)}{I_{sun}} \right) \cdot \textrm{QE}(\lambda) \cdot \textrm{ATM}(\lambda)
  \cdot \left( \frac{\lambda}{h c} \right) \: d\lambda}  
\end{equation}

In Eq \ref{eq:sint}, $\textrm{SUN}(\lambda)$ is the spectrum of solar irradiance in 
$\left[\frac{W}{m^2\cdot m} \right]$, $I_{sun}$ is the irradiance of the Sun (generally taken to be
the solar constant $1361 \: \left[ \frac{W}{m^2} \right]$. Read literally, the integral term as
units $\left[ \frac{1}{Ws} \right]$, giving the number of counts per incident Watt of solar
radiation and second of integration time. The aperture diameter factor outside the imtegral accounts
for the area of light incident on the CCD, giving $\textrm{SINT}$ units of $\left[ \frac{m^2}{Ws}
\right]$. The spectra in Eq \ref{eq:sint} are plotted in Figure \ref{fig:spectra} with data in Appendix \ref{data:spectra}. Multiplying by irradiance in $\left[ \frac{W}{m^2} \right]$ and an integration time $\Delta t$ 
in seconds will yield the count of photoelectrons $S$ in ADU as shown in Eq \ref{eq:irrad_to_count}.

\begin{equation} \label{eq:irrad_to_count}
  S = \textrm{SINT} \cdot I \cdot \Delta t
\end{equation}

For completeness, irradiance can be recovered from a signal in ADU and the integration time via Eq
\ref{eq:count_to_irrad}.

\begin{equation} \label{eq:count_to_irrad}
  I = \frac{S}{\textrm{SINT} \cdot \Delta t}
\end{equation}

\section{Telescope Filter Passbands}

TODO: add curves for JC UBVRI and discuss
\ProvidesFile{ch-background-signals.tex}[Background Model]
\graphicspath{{/Users/liamrobinson/Documents/PyLightCurves/docs/build/html/_images}}

\chapter{Method}
\section{CCD Performance Model} \label{sec:ccd_performance}

Whenever an optical telescope is observing an unresolved space object, the object's signal is necessarily superimposed on whatever signals exist in the background as the unresolved signal spreads much further than the object's actual geometric bounds. In this context, background does not only refer to sources physically further than the object --- as light can easily enter optical path through atmospheric scattering --- but all sources that impact the image apart from the object signal. Some of these sources even originate within the telescope optics and its sensor. To faithfully simulate a telescope observing an object, many position-based SDA tasks are able to ignore background effects while acquiring or tracking objects. For photometry-based SDA, the background is critical. The overall noise floor can be broken up into background signal sources and sensor effects.

\subsection{Background Signal Sources}

\subsubsection{Background Source Importance}

Some background signals are more impactful than others. For simulating realistic light curves, it is important to only model those background sources that have the possibility of being at or above the order of magnitude of the object signal, thereby seriously degrading the signal-to-noise ratio. As a baseline, the background terms modeled by Krag for the PROOF CCD performance model were implemented, namely scattered moonlight, airglow, zodiacal light, and itegrated starlight \cite{krag2003}. In addition, twilight and light pollution models were implemented to increase the time and station location flexibility of the simulation. Each background term is modeled at medium fidelity --- often using tabulated measurements or set of exponential distributions to govern the scattering physics. As a result, this background performance model does not capture local weather conditions or precisely simulate the scattering of each photon through the atmosphere, but still strives to faithfully models the physics of each background signal process. Table \ref{tb:signal_importance} ranks the approximate magnitudes in photoelectrons per pixel one can expect from a telescope similar to the Purdue Optical Ground Station.

\begin{table}[]
  \centering
  \begin{tabular}{|l|l|}
  \hline
  \textbf{Source} & \textbf{Magnitude} $\mathbf{\left[ e^- / \textbf{pix}\right]}$ \\ \hline
  Twilight               & $10^1 - 10^7$                              \\ \hline
  Scattered moonlight    & $0 - 10^5$                                 \\ \hline
  Airglow                & $10^3 - 10^4$                              \\ \hline
  Zodiacal light         & $10^2 - 10^4$                              \\ \hline
  Light pollution        & $10^2 - 10^3$                              \\ \hline
  Integrated starlight   & $10^1 - 10^2$                              \\ \hline
  \end{tabular}
  \caption{Background signal importance}
  \label{tb:signal_importance}
\end{table}

\subsubsection{Astronomical Spectra}

Four of the quantities needed for the background model vary with wavelength. These are the atmospheric transmission, the sensor quantum efficiency, the irradiance of a 0th magnitude star, and the solar spectrum. Each spectrum is displayed in Figure \ref{fig:spectra}.

\begin{figure}[ht]
  \centering
  \includegraphics[width=\figmed]{sphx_glr_astro_spectra_001_2_00x.png}
  \caption{Astronomical Spectra, atmospheric transmission and zero magnitude stellar spectrum from \cite{krag2003}}
  \label{fig:spectra}
\end{figure}

In practice, the quantum efficiency curve varies by sensor and the thermal conditions of the
observation. The curve adopted in this work is that used by Krag; modern sensors will often
perform better.

\subsubsection{Airglow}

Certain chemical reactions from 80-110 km altitude in the upper atmosphere release visible light
\cite{krag2003}. This effect is known as
airglow. Since these reactions are assumed to be isotropic ---  equally intense when integrated along any
vertical line extending upwards from the surface. The airglow signal $\textrm{AINT}$ is modeled in a
similar fashion to integrated starlight. Given the airglow spectra $\textrm{GLINT}(\lambda) \:
\left[ \frac{W}{m^2\cdot m \cdot sr} \right]$, the airglow signal is computed via Eq \ref{eq:aint} \cite{krag2003}.

\begin{equation} \label{eq:aint}
 \textrm{AINT} = A_{aperture}
  \int_{10^{-8}}^{10^{-6}}{ \textrm{GLINT}(\lambda) \cdot \textrm{QE}(\lambda) \cdot \textrm{ATM}(\lambda)
  \cdot \left( \frac{\lambda}{h c} \right) \: d\lambda}  
\end{equation}

The quantity $\textrm{AINT}$ has units $\left[ \frac{1}{s\cdot sr} \right]$, meaning that the
mean airglow signal in ADU per pixel is simply given by Eq \ref{eq:airglow_adu}

\begin{equation} \label{eq:airglow_adu}
  \bar{S}_{airglow} = AINT \cdot \textrm{AM}(\theta_z) \cdot \Delta t \cdot \left( \frac{\pi s_{pix}}{648000} \right)^2
\end{equation}

In Eq \ref{eq:airglow_adu}, $\textrm{AM}(\theta_z)$ is the relative airmass function which accounts for the accumulation of air along the optical path at different zenith angles \cite{frueh2019notes}. This airmass is termed \textit{relative} as it relates the ratio of absolute airmass at a zenith angle to the absolute airmass at zenith. Often, this function is approximated by the Van-Rhijn factor $\textrm{AM}(\theta_z) = \sec{\theta_z}$ which remains accurate up to $\theta_z \approx 70^\circ$ before diverging to infinity. Instead, a function proposed by Pickering is used \cite{pickering2002}.

\begin{equation} \label{eq:pickering_airmass}
  \textrm{AM}(\theta_z) = \frac{1}{\sin\left((90 - \theta_z) +  \frac{244}{165 + 47 * \left(90 - \theta_z \right)^{1.1}}\right)}
\end{equation}

Using Eq \ref{eq:pickering_airmass} instead of the Van-Rhijn factor is important for computing background signals near the horizon. Figure \ref{fig:airmass_fcns} displays this comparison in action.

\begin{figure}[ht]
  \centering
  \includegraphics[width=\figmed]{sphx_glr_vr_factor_001.png}
  \caption{Airmass function comparison. The Van-Rhijn factor diverges to $+\infty$ while Pickering's function reaches the correct maximum of $\textrm{AM}(\theta_z) \approx 40$.}
  \label{fig:airmass_fcns}
\end{figure}

\begin{figure}[ht]
  \centering
  \includegraphics[width=\figmed]{sphx_glr_background_signals_005.png}
  \caption{Mean airglow signal on the local observer hemisphere. The observer is in New Mexico, USA at
  \pogslla}
  \label{fig:airglowhemi}
\end{figure}

\subsubsection{Light Pollution}

Another source of background noise light pollution. On a cloudless night with low levels of atmospheric aerosols, 
the zenith surface brightness is approximately $22 \: \left[ \frac{mag}{arcsec^2}
\right]$ (MPSAS) \cite{krag2003}. As light pollution increases, this zenith brightness may dip down to
$14-15 \: \left[ \frac{mag}{arcsec^2} \right]$. To get accurate localized zenith brightness values,
we use the 2015 World Atlas of Sky Brightness dataset \cite{falchi2016_data}. The data is reported in $\left[
	\frac{mcd}{cm^2} \right]$ on a 30-arcsecond grid, requiring conversion to a more useful unit. A subset of the global dataset is displayed in \ref{fig:pollution_data} This conversion is listed in Eq \ref{eq:cd_per_m2_to_mpsas}, using a monochromatic $\lambda = 474$ nm to fit the conversions of Falchi et al. \cite{falchi2016}.  

\begin{figure}[ht]
  \centering
  \includegraphics[width=\figmed]{sphx_glr_nightlights_001_2_00x.png}
  \caption{Zenith light pollution in the eastern USA, data from \cite{falchi2016_data}}
  \label{fig:pollution_data}
\end{figure}

The mean light pollution CCD signal in ADU per pixel is formulated similarly to airglow. The station's zenith surface brightness $B_{poll,z}$ in MPSAS, linearly interpolated from the World Atlas dataset, is converted to irradiance per steradian via \ref{eq:mpsas_to_irrad_per_ster} and to ADU per pixel via \ref{eq:pollution_adu}. Note that Krag does not implement a specific light pollution model, but instead takes the dark sky site zenith brightness of $22$ MPSAS as input to an atmospherically scattered light model. This is simply an adaptation of Krag's model with a variable zenith brightness.
 
\begin{equation} \label{eq:pollution_adu}
  \bar{S}_{pollution} = B_{poll,z} \cdot SINT \cdot \textrm{AM}(\theta_z) \cdot \Delta t \cdot \left( \frac{\pi s_{pix}}{648000} \right)^2
\end{equation}

\begin{figure}[ht]
  \centering
  \includegraphics[width=\figmed]{sphx_glr_background_signals_003.png}
  \caption{Mean light pollution signal on the local observer hemisphere. The observer is in New Mexico, USA at
  \pogslla}
  \label{fig:pollution_hemi}
\end{figure}

\subsubsection{Twilight}

Even after the Sun sets, scattered sunlight in the upper atmosphere creates a signal on our CCD. The twilight model implemented for this work is due to Patat et al. and was developed for the European Southern Observatory at Paranal in Chile \cite{patat2006}. This model implements the zenith brightness as a function of the solar zenith angle $\gamma$ --- the angle from zenith to the Sun's apparent centroid. The model of Patat et al. fits a second-degree polynomial in $\gamma$ to approximately 2000 observations in varying atmospheric conditions, yielding separate curves for each of the UBVRI passbands. For example, for the V band, the twilight zenith brightness in MPSAS is given by \ref{eq:b_zenith_twilight} \cite{patat2006}.

\begin{equation} \label{eq:b_zenith_twilight}
  B_{twi,z} = 11.84 + 1.518(\gamma - 95^\circ) - 0.057 (\gamma -  95^\circ)^2
\end{equation}

Eq \ref{eq:b_zenith_twilight} is valid from $95^\circ \leq \gamma \leq 105^\circ$. While $\gamma \le 95^\circ$, the zenith brightness is taken to be constant and equal to the brightness at $\gamma = 95^\circ$. This is not accurate, as it predicts daylight to be the brightness of twilight, but is sufficiently bright to correctly forbid daytime observations by lowering the SNR drastically. After $\gamma = 105^\circ$ the zenith surface brightness is set to $B_{twi,z} = 22$ MPSAS to match the optimal observation condition of the light pollution model \cite{krag2003}. Zenith twilight brightness is plotted as a function of $\gamma$ in Figure \ref{fig:twilight_model}.

\begin{figure}[ht]
  \centering
  \includegraphics[width=\figmed]{sphx_glr_twilight_model_001_2_00x.png}
  \caption{Twilight model surface brightness at zenith as a function of solar zenith angle}
  \label{fig:twilight_model}
\end{figure}

Computing the mean CCD signal in ADU per pixel due to the twilight brightness proceeds identically to the light pollution formulation. 

\begin{equation} \label{eq:twilight_adu}
  \bar{S}_{twilight} = B_{twi,z} \cdot SINT \cdot \textrm{AM}(\theta_z) \cdot \Delta t \cdot \left( \frac{\pi s_{pix}}{648000} \right)^2
\end{equation}

\begin{figure}[ht]
  \centering
  \includegraphics[width=\figmed]{sphx_glr_background_signals_006.png}
  \caption{Mean twilight signal on the local observer hemisphere. The observer is in New Mexico, USA at
  \pogslla}
  \label{fig:pollution_hemi}
\end{figure}

\subsubsection{Integrated Starlight}

Stars are almost always present in optical images of space objects. The brightest stars streaking across the field of view in Figure \ref{fig:pogs_observation_example} have high SNRs and stand out clearly against the dark background. This raises a question: if the telescope observes a full $1^\circ \times 1^\circ$ area of the sky, where are the rest of the stars? The Milky Way alone contains approximately $1\cdot10^{11}$ stars. The answer is clear: many more stars are present in the image, most of them falling into the background. This residual faint starlight is called "integrated" starlight. 

\begin{figure}[ht]
  \centering
  \includegraphics[width=\figmed]{static_images/static_pogs_annotated.png}
  \caption{Raw image of three GEO objects with stars streaking through the background. As expected the star signals have a variety of signal-to-noise ratios. Taken by the Purdue Optical Ground station at \pogslla by Nathan Houtz.}
  \label{fig:pogs_observation_example}
\end{figure}

Krag \cite{krag2003} modeled this signal by building a $1^\circ \times 1^\circ$ grid of surface
brightness values for the full inertial sphere, parameterized by RA/Dec. Krag used the
Guide Star catalog, which contains 15 million stars down to apparent magnitude 16. Exponential extrapolation
was used to predict star counts in each bin for higher magnitudes \cite{krag2003}. Twenty years later, larger star catalogs exist that are nearly complete to much higher apparent magnitudes. The integrated
starlight catalog used in this work was built from the GAIA catalog with approximately 1.5 billion
stars down to magnitude 21-22 \cite{gaia_dr3}. The same $1^\circ \times 1^\circ$ grid was computed
using GAIA \cite{astroquery_gaia}, resulting in Figure
\ref{fig:gaiapatched} which shows the computed brightness map in units of $S_{10}$. 

\begin{figure}[ht]
  \centering
  \includegraphics[width=\figbig]{sphx_glr_gaia_patched_catalog_001_2_00x.png}
  \caption{Integrated starlight brightness map}
  \label{fig:gaiapatched}
\end{figure}

With this map of exoatmospheric mean brightness of the night sky due to integrated
starlight, the corresponding signal mean in the telescope CCD is computed, adopting Krag's formulation \cite{krag2003}.

\begin{equation} \label{eq:bint}
 \textrm{BINT} = A_{aperture}
  \int_{10^{-8}}^{10^{-6}}{ \textrm{STRINT}(\lambda) \cdot \textrm{QE}(\lambda) \cdot \textrm{ATM}(\lambda)
  \cdot \left( \frac{\lambda}{h c} \right) \: d\lambda}  
\end{equation}

In Eq \ref{eq:bint}, $D$ is the telescope aperture diameter in meters, $h$ is Plank's constant in
$\left[ \frac{m^2 kg}{s} \right]$, and $c$
is the speed of light in vacuum in $\left[ \frac{m}{s} \right]$. The resulting quantity
$\textrm{BINT}$ has units of $\left[ \frac{1}{s} \right]$, representing the mean total photons passing
through the telescope aperture due to integrated starlight. 

\begin{equation} \label{eq:starlightmean}
  \bar{S}_{star} = 10^{-4} \cdot BINT \cdot \left( \frac{s_{pix}}{3600} \right)^2 \cdot \Delta t \cdot
  b_{is}
\end{equation}

In Eq \ref{eq:starlightmean}, $b_{is}$ is the integrated starlight brightness in $\left[ S_{10}
\right]$ computed by linearly interpolating the dataset in Figure \ref{fig:gaiapatched}, $s_{pix}$ is the telescope pixel scale in $\left[ \frac{arcsecond}{pix} \right]$, and $\Delta t$ is the integration time in seconds. Note the addition of the $10^{-4}$ factor to reconcile catalog surface brightness in terms of 10th magnitude stars, and the 0th magnitude source in $\textrm{BINT}$. This yields $\bar{S}_{star}$ with units $\left[ \frac{e^-}{pix^2} \right]$; photoelectron counts (ADU) per pixel. Figure \ref{fig:starlight_hemi} shows the background signal mean due to integrated starlight.

\begin{figure}[ht]
  \centering
  \includegraphics[width=\figmed]{sphx_glr_background_signals_002.png}
  \caption{Integrated starlight signal on the local observer hemisphere. The observer is in New Mexico, USA at
  \pogslla}
  \label{fig:starlight_hemi}
\end{figure}

\subsubsection{Scattered Moonlight}

Moonlight scattering through the atmosphere significant increases background brightness \cite{krag2003}. This scattering effect can be decomposed into Rayleigh (isotropically distributed) and Mie (exponentially distributed) scattering modes. The Rayleigh scattered component is computed with Table 4 published by Daniels parameterized by the angle from the observation to zenith $z_{obs}$, the angle from the Moon to zenith $z_{moon}$, and the angle between the observation and the Moon on the horizon $\Delta Az$ \cite{daniels1977}. Interpolating this table yields the intensity of the Rayleigh scattering $F_{rs}$ in $10^{-10}$ $W/(cm^2 \cdot \mu m \cdot sr)$ \cite{krag2003}. The Mie scattered component is formulated with Eq \ref{eq:mie_scattering_moon}.

\begin{equation} \label{eq:mie_scattering_moon}
  F_{ms}(\lambda) = a_1 \left[ e^{-\left(\frac{\Psi}{\Psi_1}\right)} + a_2 e^{-\left(\frac{\pi - \Psi}{\Psi_2}\right)} \right] F_{rs}(\lambda)
\end{equation}

Daniels recommends $a_1 \in [50, 100]$, $a_2 \in [0.01, 0.02]$, $\Psi_1 \in [10^\circ, 20^\circ]$, and $\Psi_2 \approx 50$ \cite{daniels1977}. Prior to any station-specific fitting, the middle of these intervals are chosen, yielding $a_1 = 75$, $a_2 = 0.015$, $\Psi_1 = 15^\circ$, and $\Psi_2 = 50^\circ$. $a_1$ and $a_2$ are dimensionless, such that $F_{ms}$ also has units of $10^{-10}$ $W/(cm^2 \cdot \mu m \cdot sr)$. The total intensity of the scattered moonlight $F_{mt}$ via Eq \ref{eq:total_scattered_moonlight} following Krag's formulation \cite{krag2003}.

\begin{equation} \label{eq:total_scattered_moonlight}
  F_{mt} = f(\theta) \left[ F_{rs}(\lambda) + F_{ms}(\lambda) \right]
\end{equation}

in Eq \ref{eq:total_scattered_moonlight}, $f(\theta)$ is the lunar phase function which describes the fraction of the full Moon brightness is reflected at an observer when the Sun-Moon-observer angle is $\theta$. This function is linearly interpolated within Table 3 in \cite{daniels1977}. Finally, Krag introduces a correction factor $f_{corr}$ to account for the difference between the Sun's irradiance spectrum and the spectrum of scattered moonlight, defined in Eq \ref{eq:krag_f_corr}.

\begin{equation} \label{eq:krag_f_corr}
  f_{corr} = \frac{I_0}{SUN(550 \: \left[\textrm{nm}\right])}
\end{equation}

With all these pieces, the mean scattered moonlight signal in ADU per pixel is computed in Eq \ref{eq:moonlight_adu}.

\begin{equation} \label{eq:moonlight_adu}
  \bar{S}_{moon} = F_{mt}(550 \: \left[\textrm{nm}\right]) \cdot SINT \cdot \left( \frac{s_{pix}}{3600} \right)^2 \cdot \Delta t \cdot f_{corr}
\end{equation}

\begin{figure}[ht]
  \centering
  \includegraphics[width=\figmed]{sphx_glr_background_signals_001.png}
  \caption{Mean scattered moonlight signal on the local observer hemisphere. The observer is in New Mexico, USA at
  \pogslla}
  \label{fig:moonlight_hemi}
\end{figure}

\subsubsection{Zodiacal Light}

Zodiacal light is an effect created by sunlight reflecting off of dust in the ecliptic plane \cite{krag2003}. Zodiacal light is strongest around the Sun --- an exclusion zone for most optical telescopes --- but also reaches a peak directly away from the Sun due to the opposition effect. This peak is known as the Gegenschein, meaning "opposing light". The zodiacal light brightness is linearly interpolated within Table 1 of \cite{roach1972} which is listed for convenience in Appendix \ref{data:roach_zod}. This reports the surface brightness of the zodiacal light in $S_{10}$, which is used without conversion to find the mean CCD signal in ADU per pixel via Eq \ref{eq:zodiacal_adu}.

\begin{equation} \label{eq:zodiacal_adu}
  \bar{S}_{zod} = BINT \cdot \left( \frac{s_{pix}}{3600} \right)^2 \cdot \Delta t \cdot ZOD \cdot 10^{-4}
\end{equation}

As in the integrated starlight signal, the $10^{-4}$ factor reconciles the $S_{10}$ surface brightness with the 0th magnitude source in $\textrm{BINT}$. 

\begin{figure}[ht]
  \centering
  \includegraphics[width=\figmed]{sphx_glr_background_signals_004.png}
  \caption{Mean zodiacal light signal on the local observer hemisphere. The observer is in New Mexico, USA at
  \pogslla}
  \label{fig:zod_hemi}
\end{figure}

\subsubsection{Background Sampling}

The background signals are only defined in terms of their means, as each signal models the expected amount of radiation without accounting for the quantized nature of light \cite{krag2003}. Since light is transmitted in individual photons, their incidence on a given pixel will follow a statistical distribution. Assuming that each photon does not interact with others, the incidence of a photon on a pixel is well-modeled as a Poisson process for each background term \cite{frueh2019notes}. This distribution models the number of independent and identically distributed events that occur during a time period. For CCD astronomy, this translates to the event of a photon hitting the sensor. A Poisson distribution is defined on the positive integers by a single parameter $\lambda$ which is both the mean and variance of the distribution. The probability density function (PDF) for the Poisson distribution takes the form of Eq \ref{eq:poisson_pdf} \cite{frueh2019notes}.

\begin{equation} \label{eq:poisson_pdf}
  P_\lambda(x=k) = \frac{\lambda^k e^{-\lambda}}{k!}
\end{equation}

This distribution has a useful property that $P_{\lambda_1 + \lambda_2}(x=k) = P_{\lambda_1}(x=k) + P_{\lambda_2}(x=k)$ so long as the distributions described by $\lambda_1$ and $\lambda_2$ are independent. Our background sources are reasonably assumed to be independent as they each originate from distinct physical processes.

\begin{equation} \label{eq:background_poisson}
  \lambda_{background} = \bar{S}_{airglow} + \bar{S}_{pollution} + \bar{S}_{twilight} + \bar{S}_{star} + \bar{S}_{moon} + \bar{S}_{zod}
\end{equation}

Drawing samples from the Poisson distribution defined by $\lambda_{background}$ computes the background of the CCD image. 

\subsection{Sensor Effects}

\subsubsection{Dark Noise}

The dark noise, also called the dark current or dark count, captures the temperature-dependent accumulation of electrons in the CCD pixel wells \cite{krag2003}. This noise source is modeled as a Poisson process with parameter $\lambda_{dark}$ \cite{frueh2019notes} and is assumed to be independent from the other sensor effects. This source accumulates with the integration time, giving it units of counts per second \cite{krag2003}.

\subsubsection{Readout Noise}

When the CCD is read out, the charge contained in each pixel well must be digitized. This process introduces noise in the final signal due to electronic effects within the CCD circuitry and its surrounding environment \cite{krag2003}. The readout noise is modeled as a zero-mean Gaussian distribution with variance $\sigma_{read}^2$ and is also assumed to be independent from other sensor effects \cite{frueh2019notes}.

\subsubsection{Truncation Noise}

Truncation noise in a CCD stems from the fact that the charge in each pixel is digitized into an integer factor of the gain \cite{frueh2019notes}. This is modeled using a uniform distribution on $\left[ -g/2, g/2 \right]$, yielding a variance $N^2_{trunc} = \frac{g^2}{24}$ \cite{frueh2019notes}.

The particular noise variances for the Purdue Optical Ground Station are listed in \ref{tb:pogs_parameters}.

%%% METHOD
\ProvidesFile{ch-light-curve-simulation.tex}[Light Curve Simulation]

\chapter{Light Curve Simulation}

\section{Simulating Convex Objects}

Light curve simulation for convex geometry can be solved semi-analytically as each facet's contribution 
to the measured irradiance can be computed individually \cite{kaasalainen2001}. 
Determining whether a face is illuminated requires two horizon checks to determine visibility 
from the Sun and to the observer. For a facet $i$ at timestep $j$ these horizon checks are 
expressed by the shadowing condition $\mu_{ij}$. 

\begin{equation} \label{eq:cvx_shadow_cond}
  \mu_{ij} = \begin{cases}
    1 \text{ if } \left( \hat{O}_j \cdot \hat{n}_i \right) > 0 \text{ and } \left( \hat{S}_j \cdot \hat{n}_i \right) > 0 
	  \text{ and } \delta_{ij,\text{ss}} = 0 \text{ and } \delta_{ij,\text{os}} = 0\\
    0 \text{ otherwise } \\
  \end{cases}
\end{equation}

The unit vectors $\hat{O}$ and $\hat{S}$ point from the  center of mass of the object to the observer and Sun, respectively. 
We choose the outward-pointing facet normal unit vector $\hat{n}$ by convention for all mesh operations. 
The self-shadowing and observer-shadowing conditions, $\delta_{ij,\text{ss}}$ and $\delta_{ij,\text{os}}$, 
are always zero for convex polyhedra but are crucial for accurately simulating non-convex geometry. 
For objects with concavities, self-shadowing refers to shadows cast by an object onto itself and observer-shadowing 
refers to otherwise visible faces blocked by other portions of the geometry.

The irradiance $I$ received by the observer at timestep $j$ is the sum of the received irradiance from all facets, 
composed of specular and diffuse contributions. We express each contribution as the product of the
normalized irradiance $\hat{I}$. This can be scaled to adjust for the distance from the observer to
the object to yield the noiseless received irradiance.

\section{Simulating Non-Convex Objects}

Many existing light curve simulation methods for non-convex objects rely on ray tracing schemes like Möller and Trumbore's ray-triangle intersection algorithm \cite{moller2005,fan2020thesis}. This computation is necessarily complex as there may be significant self-shadowing at large phase angles. As a result, we cannot assume $\delta_{ij,\text{ss}} = 0$ and $\delta_{ij,\text{os}} = 0$ \cite{frueh2014,fan2020thesis}. In the absense of a bounding volume hierarchy or other techniques to reduce the number of rays cast, ray traced shadows generally require $\mathcal{O}(n^2)$ ray-triangle intersections per timestep for $n$ facets. For this reason, ray traced shadows quickly become infeasible for complex reference geometries without GPU parallelization. The limitations of ray-triangle intersections for light curve simulation is discussed at length by Frueh et al. \cite{frueh2014}.

For faster and more accurate simulated light curves, we use shadow mapping computed on the GPU. Shadow mapping is a well understood, if dated, technique in computer graphics \cite{kolivand2013}. Although modern ray traced shadowing may be more computationally efficient, shadow mapping was selected for its ease of implementation, once the inherent aliasing or `shadow acne' was addressed using standard remedies \cite{kolivand2013}. Because shadow mapping shades individual pixel fragments instead of entire facets, it offers increasing shadow quality over facetwise ray tracing as the number of mesh faces falls.

\graphicspath{{/Users/liamrobinson/Documents/msthesis/static_images/aas_2022_figs}}
\begin{figure}[!htb]
  \centering
  \includegraphics[width=350px]{hst_shadow_mapping/composite_hst_raytraced.png}
  \caption{Hubble Space Telescope ray traced shadow categorization and shading. Models from \cite{nasa_models}}
  \label{hst_shadows_ray}
\end{figure}

\subsection{Importance of Self-Shadowing for Human-Made Objects}

To motivate the importance of accurate shadowing computation for human-made space objects, consider the error introduced by neglecting shadows for different types of space objects. Kaasalainen and Torppa's work on asteroids reasonably assumed that shadowing was a negligible contribution to the measured light curve. Human-made objects do not afford the same luxury. Figure \ref{fig:hst_bennu_shadows} displays light curves for the asteroid Bennu and the Hubble Space Telescope with and without accurate shadows under a single-axis attitude profile with inertially fixed Sun and observer vectors. Without accurate shadowing, the light curve's magnitude and time derivatives are both dramatically effected.

\begin{figure}[!htb]
  \centering
  \includegraphics[width=350px]{convex_vs_nconv_lcs.png}
  \caption{Brightness errors introduced by neglecting shadows for Bennu and HST. Models from \cite{nasa_models}}
  \label{fig:hst_bennu_shadows}
\end{figure}

\subsection{Shadow Mapping for Light Curve Simulation}

\begin{figure}[!htb]
  \centering
  \includegraphics[width=200px]{hst_shadow_mapping/hst_shadow_mapping.png}
  \caption{Hubble Space Telescope shadow mapping with self (red) and horizon (blue) shadows rendered. Models from \cite{nasa_models}}
  \label{fig:hst_shadows_map}
\end{figure}

Given an observer and Sun vector in the body frame of the object, shadow mapping proceeds in a four step process. In step one, a camera is positioned along the Sun vector and a perpendicular depth texture is computed. In the second step, depth values in Sun camera space are transformed to observer camera space, forming a second depth texture. This second texture is used to find the closest fragment along each ray to the Sun, determining the self-shadowing condition \cite{brabec2002}. Self-shadowed fragments are classified as those further from the Sun than the closest fragment along the same ray, indicated in red in Figure \ref{fig:hst_shadows_map}. Fragments that do not pass the convex shadowing condition are horizon shadowed, indicated in blue in Figure \ref{fig:hst_shadows_map}. All remaining fragments are shaded with using the same Lambertian reflection model in \ref{lc_func_diffuse}. Computing the light curve function for the final rendered image requires summing all pixel values and dimensionalizing the result by the area of the observer camera's field of view. The light curve simulation environment used in this work was implemented in C and OpenGL using raylib \cite{raylib}.

\ProvidesFile{ch-light-curve-inversion.tex}[Light Curve Shape Inversion]
\graphicspath{{/Users/liamrobinson/Documents/msthesis/static_images/aas_2022_figs}}
\section{Light Curve Shape Inversion}

\subsection{Direct Convex Shape Inversion}

Traditionally, direct light curve inversion involves two distinct optimization problems: a linear least squares problem to fit an EGI to the measured light curve, and a second optimization to produce accurate vertex positions and face adjacency information \cite{fan2020thesis}. The first problem is data-driven and linear, using the observations to estimate a plausible EGI. The second problem is highly nonlinear but convex and requires significant tuning for robust convergence \cite{fan2019}.

\subsubsection{The Extended Gaussian Image} \label{sec:egi_definition}

The discrete EGI $\vec{E} \in \mathbb{R}^{m \times 3}$ is composed of $m$ unit vectors $\hat{n}$ each scaled a nonnegative scalar $a \in \mathbb{R}, \: a_i \geq 0$ \cite{little1983}.

\begin{equation}
  \vec{E}_i = a_i \hat{n}_i
\end{equation}

In the context of shape inversion, the $m$ vectors $\hat{n}$ should be a relatively uniform tessellation of the unit sphere. A convex polytope can be uniquely represented by an EGI of facet normal vectors scaled by each facet's area. The set of normal vectors in an EGI is denoted $\mathcal{N}$ with the set of areas denoted $\mathcal{A}$. The vector of facet areas is denoted $\vec{a} \in \mathbb{R}^{m \times 1}$. The norm of the EGI is nottated $\| \vec{E} \| = \vec{a}$ with the `size' of the EGI $\|\vec{E}\| = m$.

The solution to the Minkowski problem proves the existence and uniqueness of a convex polytope for any EGI satisfying the closure condition in Eq \ref{eq:egi_closure} \cite{minkowski1909}. Equivalently, an EGI uniquely represents a closed, convex polyhedron --- a polytope --- with no open boundaries, up to a translation.

\begin{equation} \label{eq:egi_closure}
  \sum_{i=1}^m a_i \hat{n}_i = [0, 0, 0]
\end{equation}

While a given EGI uniquely represents a polytope, that same EGI could also be interpreted to be an infinite number of nonconvex and open geometries. An example of this extended family is depicted in Figure \ref{fig:egi_family}.

\begin{figure}[!htb]
  \centering
  \includegraphics[width=\figmed]{convex_non_open_egis.png}
  \caption{Simplified convex, nonconvex, and open EGI nonuniqueness. Larger circles indicate greater relative areas assigned to a given normal vector.}
  \label{fig:egi_family}
\end{figure}

\subsubsection{EGI Optimization}

The EGI fulfills two important criteria for the shape inversion problem: it can be estimated directly from the light curve, attitude profile, and material properties, and uniquely represents a convex object \cite{kaasalainen2001}. Further, the EGI can be transformed into a unique convex object and vice versa through the dual transform and Minkowski problem \cite{little1985, minkowski1909}. 

Given a light curve, direct shape inversion schemes sample $m$ candidate normal vectors $\hat{n}$ on the unit sphere to fit an EGI to the observed light curve $\vec{L}_\textrm{ref} \in \mathbb{R}^{n \times 1}$ \cite{friedman2020, fan2020thesis}. This is accomplished by solving an optimization problem to distribute the area vector $\vec{a}$ across the sampled normals to minimize the residual between the observed and modeled light curves. In practice, this is a constrained nonnegative least squares (NNLS) problem and can be solved efficiently for large numbers of normal vectors and light curve data points:

\begin{equation} \label{eq:area_opt_convex}
  \min_{a}{\|\vec{L}_{\textrm{ref}} - G \vec{a}\|_2} \:\:\: \textrm{ subject to } \vec{a}_i \geq 0.
\end{equation}

NNLS problems are efficiently solved via Lawson and Hanson's original algorithm \cite{lawson1976}, or the more recent Fast NNLS (FNNLS) algorithm due to Bro and De Jong \cite{bro1996}. It is important to note that the area estimated with Eq \ref{eq:area_opt_convex} is the \textit{albedo-area} due to the reflectivity coefficients in Eq \ref{eq:lc_func_normalized} or \ref{eq:lc_normalized_engine}. If the value of $C_d$ and $C_s$ are uniform with a known ratio $C_d / C_s$, the recovered geometry will incorrectly scaled without impacting the face adjacency or relative feature sizes.

The convex reflection matrix $G \in \mathbb{R}^{n \times m}$ with $ij$th entries $G_{ij}$ defined at time $i$ for each facet $j$ is defined as the normalized received facet irradiance per unit facet area:

\begin{align*} \label{ref_cond_matrix}
  G_{ij} &= \frac{I_{ij}}{I_0 a_j} \\
  &= f_r\left(L \rightarrow O \right) \left(L_j \cdot N_i\right) \left(O_j \cdot N_i\right).
\end{align*}

This relationship between the recieved normalized irradiance and the areas of the object faces reveals a more compact form of Eq \ref{eq:lc_func_normalized} for convex objects for any BRDF:

\begin{equation} \label{convex_lc_with_g}
  \hat{I}_{convex} = G \vec{a}.
\end{equation}

The optimization in Eq. \ref{eq:area_opt_convex} produces a coarse approximation of the true EGI as $m$ is finite. Increasing $m$ necessarily improves the quality and sparsity of the estimated EGI, but at the cost of computational resources. The estimation was performed using a synthetic light curve input from $n=500$ Sun and observer vectors uniformly sampled on the sphere in the body frame, producing a full rank $G$ matrix. $m = 500$ candidate normal vectors were sampled using a spherical Fibbonaci mapping described by Keinert et al. in \cite{keinert2015}. Results are visualized for an icosahedron in the body frame in Figure \ref{fig:initial_egi_sampling}. Reconstructing the object at this stage is difficult due to the quantity of faces present in the estimated EGI. 

\graphicspath{{/Users/liamrobinson/Documents/PyLightCurves/docs/build/html/_images}}
\begin{figure}[!htb]
  \centering
  \includegraphics[width=\figmed]{sphx_glr_egi_figs_aas22_002.png}
  \caption{Initial EGI optimization example for a cube, using 500 candidate normal vectors, the Phong BRDF with $C_d=0.5$, $C_s=0.5$, $n=10$. Relative face area denoted by higher sphere opacity, reference shape shown in grey.}
  \label{fig:initial_egi_sampling}
\end{figure}

\subsubsection{EGI Resampling}

A normal vector resampling step is presented in this work to promote a more accurate and sparse EGI. The normal vectors used in Fig \ref{fig:initial_egi_sampling} are generally correct, with each group clustering around a normal vector of the truth geometry. This clustering behavior occurs when none of the candidate normal vectors are sufficiently close to the truth. Resampling in a cone centered on each initial EGI normal vector provides more accurate candidates for EGI estimation. This process mimics a single optimization step with a much larger $m$, where the coarse EGI is used to exclude areas on the sphere with little or no normal area.

Uniformly sampling a cone of half-angle $\phi$ is accomplished by strategically sampling points on the unit sphere. 

\begin{equation} \label{cone_sample_n_pole}
  \hat{n}_{cone} = \begin{bmatrix}
    \sqrt{1-z^2}\cos{\theta} \\
    \sqrt{1-z^2}\sin{\theta} \\
    z \\
  \end{bmatrix}
\end{equation}

In Eq. \ref{cone_sample_n_pole} two coordinates are chosen $z \in [\cos{\phi}, 1]$ and $\theta \in [0, 2\pi)$, yielding a point uniformly distributed on a cone of half-angle $\phi$ about the central axis $[0, 0, 1]^T$ \cite{cone_sampling_wolfram}. These points are then rotated using a direction cosine matrix to center the cone on an axis of interest. The axis of rotation for this transformation is the cross product of the original central axis $[0, 0, 1]^T$ with the final axis $\hat{n}_{cone}$ with the rotation angle $\theta$ being the angle between the same two vectors. The principal rotation parameter form of this transformation is given in Eq \ref{eq:cone_sampling_rv} which can be converted into the DCM using Eq \ref{eq:quat2dcm} and \ref{eq:prp2quat}.

\begin{align*} \label{eq:cone_sampling_rv} \numberthis
  \theta &= \cos^{-1}\left( \hat{n}_{cone,z} \right) \\
  \lambda &= \hat{n}_{cone} \times [0, 0, 1]^T
\end{align*}

The number of new candidates sampled per initial solution vector and the cone half-angle should be adjusted on a case-by-case basis depending on the compute power available and light curve data quality. Multiple iterative methods exist for solving nonnegatively constrained least squares (NNLS) problems. The classical NNLS algorithm was published by Lawson and Hanson and improved later by Bro and De Jong in their Fast NNLS (FNNLS) approach \cite{lawson1976, bro1996}.

Existing EGI optimization schemes like those of Fan \cite{fan2020thesis}, Friedman \cite{friedman2020}, and Cabrera \cite{cabrera2021} are limited by a single normal vector sampling step, leading to a lack of sparsity in the optimized EGI. High-density normal vector sampling in regions known to contain non-zero area leads to EGI solutions that are generally more sparse and cluster more tightly about true normal vectors. An example of this process is shown in Figure \ref{fig:resampled_egi}.

\begin{figure}[!htb]
  \centering
  \includegraphics[width=\figmed]{sphx_glr_egi_figs_aas22_003.png}
  \caption{Resampled EGI for the cube data using $\phi = \frac{\pi}{20}$, $100$ candidate vectors per resampled cone. Relative face area denoted by higher sphere opacity, reference shape shown in grey.}
  \label{fig:resampled_egi}
\end{figure}

\subsubsection{EGI Merging}

After resampling and reoptimizing with Eq. \ref{eq:area_opt_convex}, the reestimated EGI is merged by computing all groups $\mathcal{G}$ of EGI vectors within an angular offset $\alpha$:

\begin{equation}
  \mathcal{G}_k = \left\{ \vec{E}_i \in \vec{E} \:\| \cos^{-1}\left( \frac{\hat{E}_i \cdot \hat{E}_k}{\|\vec{E}_i \| \| \vec{E}_k \|}\right) < \alpha \right\}.
\end{equation}

In practice, the choice of $\alpha$ is dependent on the user's tolerance for discretization, as merging will approximate smooth geometry by discrete faces with normal vectors offset by $2\alpha$. Groups are merged by summing all group members, yielding a single EGI vector $\vec{E}_m$ without loss of closure. 

\begin{equation} \label{eq:fixing_egi}
  \vec{E}_m = \sum_{\vec{E} \in \mathcal{G}_k}{\vec{E}}
\end{equation}

Merging the resampled EGI shown in Figure \ref{fig:resampled_egi} produces a final sparse EGI fit for object reconstruction, shown in Figure \ref{fig:merged_egi}. 

\begin{figure}[!htb]
  \centering
  \includegraphics[width=\figmed]{sphx_glr_egi_figs_aas22_004.png}
  \caption{Merged EGI for the cube data using $\alpha = \frac{\pi}{10}$. Relative face area denoted by higher sphere opacity, reference shape shown in grey.}
  \label{fig:merged_egi}
\end{figure}
\graphicspath{{/Users/liamrobinson/Documents/msthesis/static_images/aas_2022_figs}}

\subsubsection{Geometry Recovery from the EGI}

At this stage, the resampled and merged EGI encodes a convex approximation of the underlying object with no guarantee of the closure of this EGI. The EGI closure constraint Eq \ref{eq:egi_closure} motivates a simple procedure to correct an invalid EGI by adding the mean closure error to each entry:

\begin{equation} \label{eq:egi_validation}
  \vec{E}_{\textrm{closed}} = \vec{E}_{\textrm{open}} - \sum_{i=1}^m a_i \hat{n}_i .
\end{equation}

The concept of a closure step is not a novel contribution. Fan's method solved an problem to adjust the EGI towards closure \cite{fan2020thesis}. This process is inproved with a simpler analytical correction. In practice, this process should be performed before each reconstruction to accelerate convergence. Failing to correct non-closed EGIs will cause convergence to a nonzero minimum in the reconstruction objective function as there is no corresponding convex object with the given EGI.

The unique convex object encoded by each closed EGI is reconstructed by solving for the polytope's set of vertices $\mathcal{V}$ and faces $\mathcal{F}$ encoding the adjacency relations between vertices. This is accomplished following the procedure introduced by Little through the dual transformation \cite{little1983}. The dual set $\mathcal{D}$ are vertices in $(A, B, C) \in \mathbb{R}^3$ that satisfy the following plane condition for a point $(x, y, z)$ on each face of the object:

\begin{equation} \label{dual_abc_form}
  Ax + By + Cz + 1 = 0
\end{equation}

If $(x, y, z)$ are chosen to be the closest points in the object's planes to the origin, the dual set $\mathcal{D}$ can be expressed in terms of the EGI and a support vector $\vec{h} \in \mathbb{R}^{\|F\| \times 1}$, as expressed in Eq \ref{eq:dual_egi_form}. The support vector is the perpendicular distance of each face defining the object to the origin.

\begin{equation} \label{eq:dual_egi_form}
  \mathcal{D} = \frac{\vec{E}}{ \| \vec{E} \| \vec{h}}
\end{equation}

The object's vertices $\vec{v}_{ref}$ are found by solving a linear system of equations for each face on the convex hull of dual set vertices. Triplets of vertices on the resulting faces are used to find a single real vertex by intersecting the three planes defining the dual set vertices.

\begin{equation} \label{eq:vertex_recovery}
  \begin{bmatrix}
    v_{ref,x} \\
    v_{ref,y} \\
    v_{ref,z} \\
  \end{bmatrix} = \begin{bmatrix}
    v_{i,x} & v_{j,x} & v_{k,x} \\
    v_{i,y} & v_{j,y} & v_{k,y} \\
    v_{i,z} & v_{j,z} & v_{k,z}
  \end{bmatrix}^{-1} \begin{bmatrix}
    1 \\ 1 \\ 1
  \end{bmatrix}
\end{equation}

Convex face adjacency information is found by triangulating the convex hull of all reference vertices. The accuracy of the recovered geometry is entirely dependent on the correctness of the support vector $\vec{h}$ used to produce the dual set. Finding the true support vector is the challenge of the final optimization in convex shape inversion.

\subsubsection{Support Vector Optimization}

This work adopts the same objective function proposed by Little for the support vector optimization as used by Fan \cite{little1983, fan2020thesis}. Little's objective function leverages work of Minkowski, who proved that any closed EGI is a unique representation of a convex object, up to a translation \cite{little1983}. In the literature, these convex shapes in $\mathbb{R}^3$ are known as polytopes. In particular, that unique polytope is merely a scaling and translation from the polytope with volume $1$ that minimizes $\vec{h} \cdot \| \vec{E} \|$ \cite{little1985}. This formulates the support vector optimization problem:

\begin{align*} \numberthis \label{eq:little_problem}
  & \min_{\vec{h}}\left\{ \vec{h} \cdot \| \vec{E} \| \right\} \\
  & \textrm{such that } \frac{1}{3}\left(\vec{h} \cdot \vec{a} \right) = 1
\end{align*} 

Notice that there is a difference between $ \| \vec{E} \| $, the areas of the faces on the desired polytope, and $\vec{a}$, the areas of the polytope recovered by constructing a convex object with the dual transform using Eqs \ref{eq:dual_egi_form} and \ref{eq:vertex_recovery}. This means that a convex object must be fully constructed at each step of the optimization to compute the objective function value. This optimization is highly nonlinear, as adjusting the support of one face may dramatically change the both the areas of surrounding faces as well as the volume of the resulting polytope. Due to the convexity of the problem, an initial guess of $\vec{h} = \mathbf{1}$ is sufficient.

Depending on the optimizer, additional function evaluations will be needed to evaluate support gradient as well as the Jacobian of the constraints. Many optimizers struggle with this optimization problem, either due to an inability to meet the constraint or instability that leads to unbounded supports. Little implemented a constrained gradient descent scheme from scratch, showing good results for reconstructing simple objects \cite{little1983}. Gardner solved an equivalent optimization problem using an unspecified solver in MATLAB \cite{gardner2003}. In this work, trust-region methods \cite{conn2000} were found to be robust at solving this optimization.

\subsection{Direct Nonconvex Shape Inversion}

\subsubsection{Nonconvex Feature Detection and Location}

Many human-made space objects are, as highlighted in Figure \ref{hst_bennu_shadows}, highly nonconvex. As a result, their shape inversion is plagued by the fact that the Minkowski problem-driven reconstruction methods of Eq \ref{little_problem} cannot recover nonconvex features. Instead of beginning from the ground up, the convex shape guess can be leveraged to detect and locate concavities.

Information about large, unilateral object concavities is obtained during EGI estimation in Eq. \ref{eq:area_opt_convex} by relaxing the EGI closure constraint. This unconstrained form is also generally functional for most convex objects and can be used without loss of detail in the final reconstruction as long as closure correction in Eq. \ref{eq:fixing_egi} is still employed.

The mean axis of prominent concave features is determined by measuring the divergence of the optimized EGI from a closed object with the magnitude of the closure error $\vec{e}_{EGI}$.

\begin{equation} \label{eq:closure_error}
  \vec{e}_{EGI} = -\sum_{i=1}^{m} a_i \hat{n}_i.
\end{equation}

The EGI closure error vector in Eq \ref{eq:closure_error} represents the missing area on each body axis that could be added to close the object. The addition of the minus sign transforms the vector from expressing the presence of excess area to the absence of missing area. The closure error will be negligable if there are no concavities present. The closure error may also be negligable if there is no self-shadowing is present over the sampled attitude profile, therefore the closure error merely quantifies the self-shadowing that is occuring, not whether there may be self-shadowing in other orientations.

Under the strong assumption that the concavities present are major and unilateral, this EGI error vector points along the mean axis of the concavity. This is because self-shadowing in that region causes brightness measurements to be smaller than expected for the true area present, leading to an optimized EGI that attributes less area to that section of the object. As a result, summing the EGI entries produces a vector that points away from the concavity, requiring the negation seen in Eq \ref{eq:closure_error} to point towards it.

\subsubsection{Concavity Creation}

Given the direction of a supposed concavity in the body frame of the object, the next challenge is introducing that concavity. This work approximates unknown concave features as valleys with sharp floors, adopting the interior angle $\psi$ between the valley walls as the measure of the feature depth. An interior angle of $\psi=180^\circ$ implies no concavity, whereas $\psi \rightarrow 0^\circ$ will push the concavity through the object entirely.

The proposed process for creating an accurate concavity in the reconstructed convex guess proceeds in four major steps. The model is first subdivided to add more faces and vertices. Subdivided vertices are then classified by their proximity to the EGI error vector, indicating whether their positions should be updated. Boundary vertices are identified, and vertex positions are updated based on an internal angle. The light curve error of that nonconvex guess is computed, and the internal angle is iterated until a suitable minimum error case is identified.

\subsubsection{Model Subdivision}

Subdividing the initial convex object guess is essential for retaining object detail during concavity creation. A combination of linear subdivision, Loop subdivision, and remeshing algorithms are used to accomplish this. Linear subdivision is advantageous when object faces are equally sized and boundary edges must be maintained. Loop subdivision is preferable when there are numerous vertices so that subdivisions do not drastically diverge from the initial boundary surface. Loop subdivision softens sharp edges as it relies on B-splines to interpolate new vertex positions \cite{loop1987}. The specific type and resolution of subdivision used depends on the level of detail the user needs to maintain in the introduced concavity, although linear subdivision followed by Loop subdivision is a useful baseline. Varying combinations of subdivision are shown in Figure \ref{fig:subd_grid} to illustrate the available configurations.

\begin{figure}[!htb]
  \centering
  \includegraphics[width=350px]{subd_grid.png}
  \caption{Subdivided object (black) with reference (red) with various levels of subdivision}
  \label{fig:subd_grid}
\end{figure}

\subsubsection{Vertex Classification}

When introducing a concavity, it is important to classify which vertices are part of the concave feature --- and therefore need to be updated --- and which vertices should remain unaffected. This is accomplished by measuring the angle from each face normal to the EGI error vector, where faces with normal vectors within an angle of $\pi/2$ to the error vector must be updated. In reality, all face normals and areas are impacted by the presence of the concavity in the area optimization Eq. \ref{eq:area_opt_convex} and EGI correction step Eq \ref{eq:egi_validation}. Selecting the angle defect $\pi/2$ updates all faces above the horizon from the EGI error vector. This bound tends to produce visually accurate concavities. Faces requiring an update are termed \textit{free} faces, with all others termed \textit{root} faces.

Vertices on free faces are further classified as being \textit{root-adjacent} or \textit{free}. Root-adjacent vertices are part of at least one root face, whereas free vertices belong to only free faces. Classifying vertices in this way results in a border of root-adjacent vertices around the interior free vertices, visualized in Figure \ref{fig:root_and_free}.

\begin{figure}[!htb]
  \centering
  \includegraphics[width=250px]{rootadj_and_free_verts_try2.png}
  \caption{Root-adjacent and free vertices}
  \label{fig:root_and_free}
\end{figure}

\subsubsection{Vertex Displacement}

Given the estimated internal angle $\psi_{est}$ and the error vector $\hat{e}_{EGI}$, each $i$th free vertex is displaced to introduce a geometrically accurate concavity by moving each a distance $d_i$ in the direction of $-\hat{e}_{EGI}$:

\begin{equation} \label{eq:flip_depth}
  d_i = p_i \sqrt{\csc^2 \frac{\psi_{est}}{2} - 1},
\end{equation}

where $p_i$ is the distance from each $i$th free vertex to the nearest root-adjacent vertex.

\subsubsection{Internal Angle Iteration}

Prior work by the author indicated an analytical relationship between this interior angle and the norm of the EGI error vector, summarized in Figure \ref{fig:misleading_egi_error} \cite{robinson2022}.

\begin{figure}[!htb]
  \centering
  \includegraphics[width=\figsmall]{error_mag_study/combined_error_mag.png}
  \caption{EGI error relationship to internal angle for the collapsed cylinder and collapsed house from \cite{robinson2022}}
  \label{fig:misleading_egi_error}
\end{figure}

The objects in Figure \ref{fig:misleading_egi_error} were illuminated with a simple Lambertian BRDF and the brightness measurements used to optimize the EGI were produced by randomly sampled Sun and observer vectors. Once specular effects and non-random observation conditions are accounted for, the linear relationship between $\psi$ and $\sqrt{\frac{\|\vec{e}_{EGI}\|}{\|\vec{E}\|}}$ no longer exists. 

In practice, a line search is sufficient to find the interior angle that minimizes the light curve error of the reconstructed object, summarized in Algorithm \ref{alg:concavity_iter}.

\begin{algorithm}
  \caption{Concavity sizing algorithm}\label{alg:concavity_iter}
  \begin{algorithmic}
    \State $f_{cvx},v_{cvx}$ \Comment{Faces and vertices of the convex guess}
    \State $f_{subd}, \:v_{subd} \gets \mathrm{Subdivide}(f_{cvx},v_{cvx})$ \Comment{Subdivided convex guess}
    \State $\psi = 180^\circ$ \Comment{Initial guess, no concavity}
    \State $i = 0$ \Comment{Iteration number}
    \While {$i < i_{max}$}
      \State $f_{disp}, \:v_{disp} = \mathrm{DisplaceVertices}(f_{subd},v_{subd})$
      \State $\hat{I}_{i} = \mathrm{LightCurve}(f_{disp}, \:v_{disp})$
      \State $e_i = \| \hat{I} - \hat{I}_{i} \|$
      \State{$\psi \gets \psi + \Delta \psi$}
      \State{$i \gets i + 1$}
    \EndWhile

    $\psi_{opt} = \min_i{e_i}$
  \end{algorithmic}
\end{algorithm}

\subsection{Shape Inversion With Noisy Measurements}

\subsubsection{Shape Interpolation With Signed Distance Functions}

The signed distance field (SDF) implicitly represents a shape by associating each point in $\mathbb{R}^3$ with the distance to the closest point on the surface of the object \cite{baerentzen2002}. For a given SDF $f(x, y, z)$, the surface of the object is given by:

\begin{equation} \label{eq:sdf_zero_level_set}
  \begin{bmatrix} x \\ y \\ z \end{bmatrix} : \: \: f(x, y, z) = 0.
\end{equation}

Computing the SDF of a triangulated mesh breaks down into computing distances from the points, line segments, and planes making up the mesh to the queried point \cite{baerentzen2002}. A slice of the SDF of a test model is displayed in Figure \ref{fig:sdf_slice}.

\graphicspath{{/Users/liamrobinson/Documents/PyLightCurves/docs/build/html/_images}}
\begin{figure}[!htb]
  \centering
  \includegraphics[width=\figmed]{sphx_glr_sfds_001_2_00x.png}
  \caption{SDF slice of the Stanford bunny model}
  \label{fig:sdf_slice}
\end{figure}

Interpolating three-dimensional meshes using signed distance fields is not a novel concept of this work. Cohen-Or et al. used distance fields with anchor points to find warping functions between two geometries \cite{cohen_or1998}. A simpler, less robust interpolation strategy between two shapes can be accomplished through a weighted average of the respective SDFs. If both objects are weighted at $50\%$, the weighted sum of their SDFs produces a surface that lies halfway in between the two original shapes, measured perpendicular to the input objects' surfaces. This interpolated zero level set surface can be extracted through any three dimensional isocontouring algorithm such as marching cubes \cite{lorensen1987} or flying edges \cite{schroeder2015}. For example, interpolating between a torus and an icosahedron using this method yields Figure \ref{fig:interpolating_torus_ico}. 

\begin{figure}[!htb]
  \centering
  \includegraphics[width=\figmed]{sphx_glr_shape_interpolation_001.png}
  \caption{SDF interpolation between a torus and an icosahedron}
  \label{fig:interpolating_torus_ico}
\end{figure}

The proposed algorithm for shape interpolation for SDF interpolation of two shapes $M_1, M_2$ with convex weights $w_1, w_2$ is:

\begin{algorithm}
  \caption{SDF interpolation}\label{alg:sdf_interp}
  \begin{algorithmic}
  \Require $w_1 + w_2 = 1$
  \State $\mathrm{SDF}_{\textrm{interp}} \gets w_1 \cdot \mathrm{SDF}(M_1) + w_2 \cdot \mathrm{SDF}(M_2)$
  \State $M_{\textrm{interp}} = \mathrm{Isocontour}(\mathrm{SDF}_{\textrm{interp}}, 0)$.
  \end{algorithmic}
\end{algorithm}

Algorithm \ref{alg:sdf_interp} can operate on arbitrary convex or non-convex meshes, making it well-suited to interpolating between light curve inversion results.

TODO: write

\chapter{Results}

\section{Realistic Light Curves}

\begin{figure}[!htb]
  \centering
  \includegraphics[width=\figmed]{sphx_glr_lc_uncertainty_002_2_00x.png}
  \caption{Light curves for a $1$-meter diffuse cube with $C_d = 0.5$ observed from the Purdue Optical Ground Station at \pogslla. Object is in the orbit of GOES 15 under torque-free rigid body motion with an inertia tensor $I = \mathrm{diag}(1.0, 2.0, 3.0) \: \left[kg \cdot m^2\right]$, $q_0 = \left[0, 0, 0, 1\right]^T$, $\omega_0 = \left[ 0.01, 0.02, 0.01 \right] \: \left[rad/s\right]$}
  \label{fig:interpolating_torus_ico}
\end{figure}

\section{Convex Shape Inversion Without Noise}

\subsection{Object Reconstruction Improvements}

The proposed resampling and merging steps in the object reconstruction process produce sparser results, in less computation time, with fewer convergence issues. Figure \ref{fig:egi_reconstructions} shows object reconstructions with each available EGI. The initial and resampled EGIs do not converge well due to the density of faces with nearby normal vectors, causing large linearization errors in the optimizer steps. It is clear that the final merged EGI produces the most accurate reconstruction of the truth object.

\graphicspath{{/Users/liamrobinson/Documents/PyLightCurves/docs/build/html/_images}}
\begin{figure}[!htb]
  \centering
  \includegraphics[width=\figmed]{sphx_glr_egi_figs_aas22_006.png}
  \caption{Object reconstruction with the EGIs of Figure \ref{fig:initial_egi_sampling, fig:resampled_egi, fig:merged_egi}, wireframe truth object in red}
  \label{fig:egi_reconstructions}
\end{figure}

\section{Nonconvex Shape Inversion Without Noise}

Displacing free vertices in the EGI error vector direction by $d_i$ yields accurate concavities for objects whose concave boundaries lie in a plane. The result of applying this process to a set of representative convex objects is shown in Figure \ref{fig:non_convex_recon_of_non_convex} using the same attitude profiles and as in Figure \ref{convex_grid}. 

\graphicspath{{/Users/liamrobinson/Documents/msthesis/static_images/aas_2022_figs}}
\begin{figure}[!htb]
  \centering
  \includegraphics[width=400px]{rec_non_convex_objs/non_convex_grid_of_nonconvex_try2.png}
  \caption{Collapsed house, cube, icosahedron, and box-wing satellite reconstructions using vertex displacement}
  \label{fig:non_convex_recon_of_non_convex}
\end{figure}

The collapsed cube and icosahedron in Figure \ref{fig:non_convex_recon_of_non_convex} are recovered effectively, but the collapsed house and box-wing satellite expose two limitations of the vertex displacement technique. In the case of the house where the concavity boundary is not constrained to a plane, the edges of the created concave feature are incorrect. The box-wing satellite's shadowing geometry leads the convex guess to be a poor approximation of the geometry outside of the concavity while also inheriting the same problem as the house.

This vertex displacement scheme will negligibly impact the convex guess if the truth object is also convex. A convex truth object will produce a small $\|\vec{e}_{EGI}\|$, causing the vertex update depth $d_{i}$ to trend towards zero as the estimated internal angle approaches $\psi = 180^\circ$. This is illustrated in Figure \ref{fig:non_convex_recon_of_convex} using the same input convex objects and attitude profiles as in Figure \ref{convex_grid}.

\graphicspath{{/Users/liamrobinson/Documents/msthesis/static_images/aas_2022_figs}}
\begin{figure}[!htb]
  \centering
  \includegraphics[width=400px]{rec_non_convex_objs/non_convex_grid_of_convex_try2.png}
  \caption{Convex objects under vertex displacement procedure}
  \label{fig:non_convex_recon_of_convex}
\end{figure}

Figure \ref{fig:non_convex_recon_of_convex} clearly displays the compatibility of vertex displacement with truly convex objects. All objects are reconstructed faithfully in both their convex and nonconvex inversions, with the same caveats noted in the discussion following Figure \ref{convex_grid}. Some truly sharp edges are rounded during mesh subdivision as seen in the gem or rectangular prism. That said, others like the cylinder become more accurate as subdivision reintroduces continuity lost to discretization in EGI merging.

\section{Convex Shape Inversion With Noise}

TODO: this

\section{Nonconvex Shape Inversion With Noise}

TODO: this



%%% RESULTS

%%% CONCLUSION
\ProvidesFile{ch-recommendations.tex}[Recommendations]
\chapter{Recommendations}
\ProvidesFile{ch-future-work.tex}[Future Work]
\chapter{Future Work}
\ProvidesFile{ch-appendices.tex}[Appendices]
\chapter{Appendices}

\section{Astronomical Spectra Data}

\subsubsection{Atmospheric Extinction}
\begin{listing}[H]
\inputminted[breaklines=true, breakanywhere=true, breaksymbol=\hspace{0pt}, fontsize=\footnotesize]{json}{/Users/liamrobinson/Documents/PyLightCurves/pyspaceaware/resources/data/atmos_extinction.json}
\end{listing}

\subsubsection{Quantum Efficiency}
\begin{listing}[H]
\inputminted[breaklines=true, breakanywhere=true, breaksymbol=\hspace{0pt}, fontsize=\footnotesize]{json}{/Users/liamrobinson/Documents/PyLightCurves/pyspaceaware/resources/data/quantum_efficiency.json}
\end{listing}

\subsection{Background Source Data}

\subsubsection{Lunar Phase Factor}
\begin{listing}[H]
\inputminted[breaklines=true, breakanywhere=true, breaksymbol=\hspace{0pt}, fontsize=\footnotesize]{json}{/Users/liamrobinson/Documents/PyLightCurves/pyspaceaware/resources/data/lunar_phase.json}
\end{listing}

\subsubsection{Scattered Moonlight}
% \begin{ZZlisting} ... \end{ZZlisting}
\begin{listing}[H]
\inputminted[breaklines=true, breakanywhere=true, breaksymbol=\hspace{0pt}, fontsize=\footnotesize]{json}{/Users/liamrobinson/Documents/PyLightCurves/pyspaceaware/resources/data/moonlight.json}
\end{listing}

\subsubsection{Zodiacal Light}
\begin{listing}[H]
\inputminted[breaklines=true, breakanywhere=true, breaksymbol=\hspace{0pt}, fontsize=\footnotesize]{json}{/Users/liamrobinson/Documents/PyLightCurves/pyspaceaware/resources/data/zodiacal.json}
\end{listing}


%
% This is only done if you are using BibLaTeX.
%
\makeatletter  % commented out on 2022-01-26
  \defbibenvironment{bibliography}
    {%
      \list
        {%
          \printtext[labelnumberwidth]%
          {%
            \printfield{prefixnumber}%
            \printfield{labelnumber}%
          }%
        }%
        {%
          \setlength{\bibhang}{1in} %%%%% was 0pt
          \setlength{\itemindent}{1in}%  -\leftmargin} %%%%% was 0pt
          \setlength{\itemsep}{\bibitemsep}%
          \setlength{\leftmargin}{0pt}%  .22in} % 0.42in}
          \setlength{\parsep}{\bibparsep}%
           \setlength{\rightmargin}{0.33in}%
        }%
    }
    {\endlist}
    {\item}
\makeatother  % commented out on 2022-01-26

% \immediate\setlength{\labelnumberwidth}{1.5in} %%%%% was commented out
\setlength{\labelwidth}{1.5in}
\def\sllnsez{[999] }

{%
  % Make _ in URLs visible.
  % \def\t{\char'137}%
  \catcode`*=\active
  \def*{\char'137}%  \char'137 is _
  \PrintBibliography
}

% LaTeX won't read after the \end{document} command.
% You can put notes to yourself or LaTeX input not
% ready for use after "\end{document}" if you'd like.
\end{document}
