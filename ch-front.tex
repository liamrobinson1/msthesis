\ProvidesFile{ch-front.tex}[2022-10-05 front matter chapter]
%
%  This is the ``front matter'' for the thesis.
%
%  REFERENCES
%
%    TCMOS17
%      The Chicago Manual of Style Online, 17th edition.
%      https://www.chicagomanualofstyle.org/home.html
%      retrieved on 2020-02-29
%
%    TEMPL
%      Thesis and Disertation Office Templates.
%      https://www.purdue.edu/gradschool/research/thesis/templates.html
%      retrieved on 2020-02-29
%
%    WNNCD
%    Webster's Ninth New Collegiate Dictionary.
%

%
%   Only Purdue University uses this page
%
%   Comment out \begin{statement} through \end{statement}
%   if you are not at Purdue University.
%
% Statement of Thesis/Dissertation Approval Page
% This page is REQUIRED.  The page should be numbered "2"
% and should NOT be listed in your TABLE OF CONTENTS.
\begin{statement}
  % Delete or add \entry commands as needed for all committe members.
  \entry{Dr.~Carolin Frueh}{School of Aeronautics and Astronautics}
  \entry{Dr.~Kenshiro Oguri}{School of Aeronautics and Astronautics}
  \entry{Dr.~Keith LeGrand}{School of Aeronautics and Astronautics}
  % There should be one \approvedby command containing the
  % "FORM 9 THESIS FORM HEAD NAME HERE" (from TEMPL, retrieved on 2020-03-01).
  \approvedby{Dr.~Gregory Blaisdell}
\end{statement}

% Dedication page is optional.
% A name and often a message in tribute to a person or cause.
% References: WEB9 332.

% \begin{dedication}
%   To graduate students
% \end{dedication}

% Acknowledgements page is optional but most theses include
% a brief statement of appreciation or recognition of special
% assistance.

\begin{acknowledgments}
  I would like to thank my advisor, Dr.~Carolin Frueh, for her guidance and mentorship. Her feedback has guided me through undergraduate research, publishing a conference paper, applying for fellowships, and now writing this thesis. For that, I am forever grateful.

  I must thank my committee, Dr.~Oguri and Dr.~LeGrand, for their feedback and attendance at my defense. Their comments have made this thesis what it is today.

  Finally,  my friends and family have been an amazing resource throughout my research. They've helped me workshop ideas, find bugs in code, and proofread proposals. Most importantly, they've always been willing to look at my unusual plots.

  This work was supported by Katalyst Space Technologies grant number FA6451-22-P-0019, Boeing work number SSOW-BRT-Z0722-5045, and the National Defense Science and Engineering Graduate Fellowship through the Air Force Office of Scientific Research under grant number FA9550-18-1-0154.
\end{acknowledgments}

% The preface is optional.
% References: TCMOS17 1.49, WEB9 927.

% \begin{preface}
%   This is the preface.
% \end{preface}

% The Table of Contents is required.
% The Table of Contents will be automatically created for you
% using information you supply in
%     \chapter
%     \section
%     \subsection
%     \subsubsection
%     commands.
\pdfbookmark{TABLE OF CONTENTS}{Contents}
\tableofcontents

% If your thesis has tables, a list of tables is required.
% The List of Tables will be automatically created for you using
% information you supply in
%     \begin{table} ... \end{table}
% environments.
\listoftables

% If your thesis has figures, a list of figures is required.
% The List of Figures will be automatically created for you using
% information you supply in
%     \begin{figure} ... \end{figure}
% environments.
\listoffigures

% If your thesis has listings, a list of listings is required.
% The List of Listings will be automatically created for you using
% information you supply in
%     \begin{ZZlisting} ... \end{ZZlisting}
% environments.

% \ZZlistoflistings

% List of Symbols is optional.
\begin{symbols}
  $I$& irradiance in $\left[ \frac{W}{m^2} \right]$\cr
  $\hat{I}$& normalized irradiance in $\left[ W \right]$\cr
  $I_0$ & Irradiance of Vega $\left[ \frac{W}{m^2} \right]$ \cr
  $m$ & Apparent magnitude $[nondim]$ \cr
  $JD$ & Julian date \cr
  $T$ & Julian centuries \cr
  $\theta_{GMST}$ & Greenwich mean sidereal time \cr
  $\theta_{r}$ & Angular offset of the first zero of the Airy disk diffraction pattern \cr
  $C_{Airy}(\theta)$ & CCD signal amplitude due to an Airy disk diffraction pattern $[ADU]$\cr
  $k$ & Wavenumber \cr
  $r_d$ & Telescope aperture radius $[m]$ \cr
  $r_o$ & Telescope central obstruction radius $[m]$ \cr
  $d$ & Telescope aperture diameter $[m]$ \cr
  $A_\mathrm{aperture}$ & Telescope aperture area $[m^2]$ \cr
  $A_\mathrm{eff}$ & Telescope effective aperture area $[m^2]$ \cr
  $f$ & Telescope focal length $[m]$ \cr
  $\lambda$ & Wavelength $[m]$ \cr
  $FWHM$ & Full width at half maximum \cr
  $C_{Gauss}(\theta)$ & CCD signal amplitude due to a Gaussian approximation of the Airy disk $[ADU]$ \cr
  $\textrm{STRINT}(\lambda)$ & Representative zero apparent magnitude star irradiance spectrum $\left[ \frac{W}{m^2 \cdot m} \right]$ \cr
  $\textrm{QE}(\lambda)$ & Quantum efficiency spectrum $\left[ \frac{ADU}{m} \right]$ \cr
  $\textrm{ATM}(\lambda)$ & Atmospheric transmission spectrum $\left[ \frac{1}{m} \right]$ \cr
  $K_{cd}(\lambda)$ & Luminous efficacy spectrum $\left[ \frac{lm}{W} \right]$ \cr
  $\mathcal{S}_\mathrm{int}$ & CCD ADU conversion factor $\left[ \frac{ADU}{W \cdot m^{-2} \cdot s} \right]$ \cr
  $\textrm{SUN}(\lambda)$ & Solar irradiance spectrum $\left[ \frac{W}{m^2 \cdot m} \right]$ \cr
  $\mathcal{A}(\lambda)$ & Airglow radiance spectrum $\left[ \frac{W}{m^2 \cdot m \cdot sr} \right]$ \cr
  $\mathcal{A}_\mathrm{int}$ & Intermediate airglow signal $\left[ \frac{1}{s \cdot sr} \right]$ \cr
  $\theta_z$ & Zenith angle $[rad]$ \cr
  $\textrm{AM}(\theta_z)$ & Relative airmass function $[nondim]$ \cr
  $s_\mathrm{pix}$ & Telescope pixel scale $\left[ \frac{arcsec}{pix} \right]$ \cr
  $\Delta t$ & CCD integration time $[s]$ \cr
  $B_{\mathrm{poll},z}$ & Zenith light pollution brightness in magnitudes per square arcsecond \cr
  $\bar{S}_{airglow}$ & Mean airglow signal $[ADU]$ \cr
  $\gamma$ & Solar zenith angle $[deg]$ \cr
  $B_{twi,z}$ & Zenith twilight brightness in magnitudes per square arcsecond \cr
  $\bar{S}_{twilight}$ & Mean twilight signal $[ADU]$ \cr
  $\mathcal{Z}$ & Zero magnitude starlight signal $[\frac{ADU}{s}]$ \cr
  $\bar{S}_{star}$ & Mean integrated starlight signal $[ADU]$ \cr
  $F_{rs}$ & Moonlight Rayleigh scattering radiance spectrum $\left[ \frac{W}{m^2 \cdot m \cdot sr} \right]$ \cr
  $F_{ms}$ & Moonlight Mie scattering radiance spectrum $\left[ \frac{W}{m^2 \cdot m \cdot sr} \right]$ \cr
  $F_{mt}$ & Total scattered moonlight radiance spectrum $\left[ \frac{W}{m^2 \cdot m \cdot sr} \right]$ \cr
  $f(\theta)$ & Lunar brightness phase function $[nondim]$ \cr
  $\bar{S}_\mathrm{Moon}$ & Mean scattered moonlight signal $[ADU]$ \cr
  $\bar{S}_{zod}$ & Mean zodiacal light signal $[ADU]$ \cr
  $\lambda_\mathrm{background}$ & Mean of background signal Poisson distribution $[ADU]$ \cr
  $\hat{n}$ & Face outward pointing unit normal vector \cr
  $\left( v_1, v_2, v_3 \right)$ & First, second, and third vertices $v_i \in \mathbb{R}^3$ on a given triangular face \cr
  $h_i$ & Support of the $i$th face \cr
  $\vec{E}$ & Extended Gaussian Image \cr
  $f_r$ & Bidirectional Reflectance Distribution Function \cr
  $L$ & Illumination direction unit vector \cr
  $O$ & Observation direction unit vector \cr
  $H$ & Halfway unit vector \cr
  $C_d$ & Coefficient of diffuse reflection \cr
  $C_s$ & Coefficient of specular reflection \cr
  $C_a$ & Coefficient of absorption \cr
  $n$ & Specular exponent \cr


\end{symbols}

% Abstract is required.
% Note that the information for the first paragraph of the output
% doesn't need to be input here...it is put in automatically from
% information you supplied earlier using \title, \author, \degree,
% and \majorprof.
% Reference: PU 17.
\begin{abstract}%
  Characterizing unknown space objects is an important component of robust space situational awareness. Estimating the shape of an object allows analysts to perform more accurate orbit propagation, probability of collision, and inference analysis about the object's origin. Due to the sheer distance from the camera combined with diffraction and atmospheric effects, most resident space objects of interest are unresolved when observed from the ground with electro-optical sensors. State of the art techniques for object characterization often rely on light curves --- the time history of the object's observed brightness. The brightness of the object is a function of the object's shape, material properties, attitude profile, as well as the observation geometry. The process of measuring real light curves is complex, involving the physics of the object, the sensor, and the background environment. The process of recovering shape information from brightness measurements is known as the light curve shape inversion problem. This problem is ill-posed without further assumptions: modern direct shape inversion methods require that the attitude profile and material properties of the object is known, or at least can be hypothesized. This work describes improvements to light curve simulation that faithfully model the environmental and sensor effects present in true light curves, yielding synthetic measurements with more accurate noise characteristics. Having access to more accurate light curves is important for developing and validating light curve inversion methods. This work also presents new methods for direct shape inversion for convex and nonconvex objects with realistic measurement noise. In particular, this work finds that improvements to the convex shape inversion process produce more accurate, sparser geometry in less time. The proposed nonconvex shape inversion method is effective at resolving singular large concave feature.
\end{abstract}
