\ProvidesFile{ch-front.tex}[2022-10-05 front matter chapter]
%
%  This is the ``front matter'' for the thesis.
%
%  REFERENCES
%
%    TCMOS17
%      The Chicago Manual of Style Online, 17th edition.
%      https://www.chicagomanualofstyle.org/home.html
%      retrieved on 2020-02-29
%
%    TEMPL
%      Thesis and Disertation Office Templates.
%      https://www.purdue.edu/gradschool/research/thesis/templates.html
%      retrieved on 2020-02-29
%
%    WNNCD
%    Webster's Ninth New Collegiate Dictionary.
%

%
%   Only Purdue University uses this page
%
%   Comment out \begin{statement} through \end{statement}
%   if you are not at Purdue University.
%
% Statement of Thesis/Dissertation Approval Page
% This page is REQUIRED.  The page should be numbered "2"
% and should NOT be listed in your TABLE OF CONTENTS.
\begin{statement}
  % Delete or add \entry commands as needed for all committe members.
  \entry{Dr.~Carolin Frueh}{School of Aeronautics and Astronautics}
  \entry{Dr.~Kenshiro Oguri}{School of Aeronautics and Astronautics}
  \entry{Dr.~Keith LeGrand}{School of Aeronautics and Astronautics}
  % There should be one \approvedby command containing the
  % "FORM 9 THESIS FORM HEAD NAME HERE" (from TEMPL, retrieved on 2020-03-01).
  \approvedby{Dr.~Gregory Blaisdell}
\end{statement}

% Dedication page is optional.
% A name and often a message in tribute to a person or cause.
% References: WEB9 332.

% \begin{dedication}
%   To graduate students
% \end{dedication}

% Acknowledgements page is optional but most theses include
% a brief statement of appreciation or recognition of special
% assistance.

\begin{acknowledgments}
  This work was supported by Katalyst grant number FA6451-22-P-0019, Boeing work number SSOW-BRT-Z0722-5045, and the National Defense Science and Engineering Graduate Fellowship through the Air Force Office of Scientific Research under grant number FA9550-18-1-0154.
\end{acknowledgments}

% The preface is optional.
% References: TCMOS17 1.49, WEB9 927.

% \begin{preface}
%   This is the preface.
% \end{preface}

% The Table of Contents is required.
% The Table of Contents will be automatically created for you
% using information you supply in
%     \chapter
%     \section
%     \subsection
%     \subsubsection
%     commands.
\pdfbookmark{TABLE OF CONTENTS}{Contents}
\tableofcontents

% If your thesis has tables, a list of tables is required.
% The List of Tables will be automatically created for you using
% information you supply in
%     \begin{table} ... \end{table}
% environments.
\listoftables

% If your thesis has figures, a list of figures is required.
% The List of Figures will be automatically created for you using
% information you supply in
%     \begin{figure} ... \end{figure}
% environments.
\listoffigures

% If your thesis has listings, a list of listings is required.
% The List of Listings will be automatically created for you using
% information you supply in
%     \begin{ZZlisting} ... \end{ZZlisting}
% environments.

% \ZZlistoflistings

% List of Symbols is optional.
\begin{symbols}
  $I$& irradiance in $\left[ \frac{W}{m^2} \right]$\cr
  $\hat{I}$& normalized irradiance in $\left[ W \right]$\cr
\end{symbols}

% Abstract is required.
% Note that the information for the first paragraph of the output
% doesn't need to be input here...it is put in automatically from
% information you supplied earlier using \title, \author, \degree,
% and \majorprof.
% Reference: PU 17.
\begin{abstract}%
  Characterizing unknown space objects is an important component of robust space situational awareness. Estimating the shape of an object allows analysts to perform more accurate orbit propagation, probability of collision, and inference analysis about the object's origin. Because most resident space objects of interest are unresolved when observed from the ground, state of the art techniques for object characterization often rely on the light curve -- the time history of the object's observed brightness. This problem is ill-posed without further assumptions; modern direct shape inversion methods require that the attitude profile of the object is known, or at least can be hypothesized. This work describes improvements to light curve simulation and direct shape inversion for non-convex and actively-controlled objects. In particular, this work finds that the proposed non-convex shape inversion method is effective at resolving large concave features, while a parametric approach is effective at fitting the geometry of controlled box wing satellites.
\end{abstract}
