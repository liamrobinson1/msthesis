\ProvidesFile{ch-meshes.tex}[Mesh chapter]

\chapter{MESHES}

\section{Implicit and Explicit Shape Representations}

A computer can represent 3D objects implicitly or explicitly. An implicit representation might be the solution to an algebraic equation, i.e., $x^2 + y^2 + z^2 = 1$ defines a sphere of radius $1$ centered at the origin. Often, signed distance functions (SDFs) may be used to implicitly define ray traced shapes. An SDF takes in a point in \rthree and outputs the distance from the object, returning negative distance if the queried point is inside the object. The object is then easily rendered via ray marching. A ray is cast for each pixel of the screen, each of which performs distance queries along its length eminating from the camera until it intersects the object or diverges. 

By contrast, an explicit shape representation creates complex 3D geometry from simple 2D building blocks. In the simplest case, object faces --- equivalently called facets --- are defined by triangles. This means that at the scale of the individual facets, the shape is always composed of flat surfaces that meet at sharp angles. While this can add complexity to many fields of shape analysis and geometry processing, triangulated surfaces are perfect for our application. Human-made space objects like most satellites are mostly composed of flat faces, with the exception of parabolic antennas and cylindrical rocket bodies.

\section{The Wavefront OBJ File Format}

One common text file format for 3D model files is \texttt{.obj}, developed by Wavefront Technologies in the early 1990s \cite{obj_format}. Each OBJ file consists of a list of vertex positions and facet definitions, with optional vertex normals and tangents. To illustrate this, the \texttt{.obj} listing for a cube is included for reference in the appendix \ref{sec:obj_listing}. Once the model file is loaded, we can compute a few properties that will prove useful for both light curve simulation and shape inversion. For each triangular facet of the model defined by vertices $\left(v_1, v_2, v_3\right)$, we compute the outward-pointing facet normal with

\begin{equation} \label{eq:facet_normal}
    \hat{n} = \frac{\left( v_2 - v_1 \right) \times \left( v_3 - v_1 \right)}{\| \left( v_2 - v_1 \right) \times \left( v_3 - v_1 \right) \|_2}.
\end{equation}

The face area is computed with

\begin{equation} \label{eq:facet_areas}
    a = \frac{\| \left( v_2 - v_1 \right) \times \left( v_3 - v_1 \right)\|_2}{2}.
\end{equation}

The support of each face --- the perpendicular distance from the origin to the plane defining the facet --- is computed with

\begin{equation} \label{eq:facet_support}
    h = v_1 \cdot \hat{n}.
\end{equation}

The volume of the entire object is compute with

\begin{equation} \label{eq:object_volume}
    \frac{1}{3} \sum_{i=0}^{ \lvert F \rvert}\vec{h}_i \cdot \vec{a}_i.
\end{equation}

In Eq \ref{eq:object_volume}, $\lvert F \rvert$ is the number of facets defining the object. $\vec{h}$ and $\vec{a}$ are column vectors collecting all facet supports and areas. The Extended Gaussian Image, a quantity defined in \ref{sec:egi_definition}, is computed row-wise for the $i$th facet with

\begin{equation} \label{eq:egi_definition}
    \vec{E}_i = \vec{a}_i \vec{n}_i.
\end{equation}

\section{Selected Space Object Model Files}

Most of the analysis in this work used one of the 3D model files shown in Figure \ref{fig:satellite_lineup}.

\begin{figure}[ht]
    \centering
    \includegraphics[width=\figbig]{sphx_glr_satellite_lineup_001.png}
    \caption{Selected space objects with soccer field for size reference. In order, the objects are TESS, Starlink V1, TDRS, Landsat 8, Hispasat 30W-6, Saturn V SII, TELSTAR 19V, HYLAS 4, and simplified ASTRA.
    }
    \label{fig:satellite_lineup}
\end{figure}

Figure \ref{fig:satellite_lineup} highlights the size of the GEO communications satellites (TELSTAR, HYLAS, Hispasat, and ASTRA). In contrast, the LEO satellites (Starlink and Landsat) are dwarfed at the left end of the lineup.