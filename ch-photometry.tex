\ProvidesFile{ch-photometry.tex}[Photometry]
\graphicspath{{/Users/liamrobinson/Documents/PyLightCurves/docs/build/html/_images}}

\chapter{Photometry}

\section{Diffraction}

\subsection{Rayleigh Criterion}

Many objects of interest are far past low-Earth orbit, making optical observations diffraction limited. Diffraction is always occuring when observing an object at any distance through any optics, but it begins to dominate when the object's scale is equal or smaller than the Rayleigh criterion. The Rayleigh criterion states that light of wavelength $\lambda$ will spread into a diffraction pattern with the first minimum of the distribution at an angular radius $\theta_R$ when passing through a circular aperture of diameter $d$ such that:

\begin{equation} \label{eq:rayleigh_criterion}
  \sin\theta_R = 1.22 \frac{\lambda}{d}.
\end{equation}

For a $1$-meter aperture optical telescope observing a $10$-meter diameter object in GEO -- giving an angular radius of $\approx 10^{-7}$ radians -- Eq \ref{eq:rayleigh_criterion} predicts that the diffraction pattern will be $5$ times wider than the object. As a result, GEO objects cannot be resolved from the ground, independent of atmospheric effects. 

\subsection{The Airy Disk}

The far-field diffraction pattern produced by a point source is known as an Airy pattern \cite{frueh2019notes}. The Airy disk is expressed in terms of an amplitude $C$ at an angular distance $\theta$ from the center $C(\theta)$ as:

\begin{equation} \label{eq:airy_disk}
  C_{Airy}(\theta) = C_0 \left( \frac{2 J_1(k \cdot r_d \sin\theta)}{k \cdot r_d \sin\theta} \right).
\end{equation}

In Eq \ref{eq:airy_disk}, $C_0$ is the amplitude of the center of the Airy disk, $r_d$ is the radius of the aperture, $k = \frac{2\pi}{\lambda}$ is the wavenumber, and $J_1$ is the first order Bessel function of the first kind. The central magnitude $C_0$ is expressed:

\begin{equation} \label{eq:airy_center}
  C_0 = \frac{\bar{C}_{all}^2 D^2}{2 f^4}.
\end{equation}

In Eq \ref{eq:airy_center}, $\bar{C}_{all}$ is the total mean signal incident on the CCD due to the source, $D$ is the aperture area, and $f$ is the focal length of the optics \cite{frueh2019notes}. The pattern produced by Eq \ref{eq:airy_disk} is depicted in Figure \ref{fig:airy_disk_magnitude}.

\begin{figure}[ht]
  \centering
  \includegraphics[width=\figmed]{sphx_glr_airy_disk_diffraction_001_2_00x.png}
  \caption{Airy disk diffraction pattern}
  \label{fig:airy_disk_magnitude}
\end{figure}

The Rayleigh criterion expresses the angular size of the first zero of the Airy disk, after which the amplitude of the Airy disk drops off exponentially. It is often useful to approximate the Airy disk with a 2D Gaussian. We can fit this Gaussian with a single parameter -- the full width at half maximum (FWHM). The FWHM expresses the diameter at which the signal drops to half the magnitude of its central maximum \cite{frueh2019notes}. The FWHM of the Airy disk is expressed:

\begin{equation} \label{eq:fwhm_airy}
  FWHM_{airy} = \frac{1.028 \lambda}{2 r_d}.
\end{equation}

The diffraction pattern is not the only effect that spreads the unresolved signal over the pixel grid. Atmospheric turbulence contributes to further spreading and speckling of the signal \cite{frueh2019notes}. This effect -- known as the \textit{seeing} -- is encapsulated in $FWHM_{seeing}$ and is generally between $1$ and $3$ arcseconds \cite{frueh2019notes}. While the seeing and diffraction pattern are additive, it is sufficient to take the larger value for simulation purposes \cite{frueh2019notes}. The standard deviation of the Gaussian approximation of the Airy disk is given by:

\begin{equation} \label{eq:airy_variance}
  \sigma = \frac{FWHM}{2 \sqrt{2 \ln{2}}}.
\end{equation}

The full Gaussian approximation at an angular distance $\theta$ from the source is given by:

\begin{equation} \label{eq:airy_gaussian}
  C_{Gauss}(\theta) = \frac{0.838 \bar{C}_{all}}{2 \pi \sigma^2} \exp\left( - \frac{\theta}{2 \sigma^2} \right)
\end{equation}

In pratice, computing the Airy disk or its Gaussian approximation on rectangular pixel grid amounts to integrating the amplitude function $C(\theta)$ over the pixel area:

\begin{equation}
  C_{pix}(x, y) = \int_{x}^{x + \Delta x} \int_{y}^{y + \Delta y}{C(\theta)} \: dy \: dy.
\end{equation}



\section{Astronomical Spectra}

Four of the quantities needed for the background model vary with wavelength. These are the atmospheric transmission, the sensor quantum efficiency, the irradiance of a 0th magnitude star, and the solar spectrum. Each spectrum is displayed in Figure \ref{fig:spectra}.

\begin{figure}[ht]
  \centering
  \includegraphics[width=\figmed]{sphx_glr_astro_spectra_001_2_00x.png}
  \caption{Astronomical Spectra from \cite{krag2003}}
  \label{fig:spectra}
\end{figure}

In practice, the quantum efficiency curve varies by sensor and the thermal conditions of the
observation. The curve adopted in this work is that used by Krag; modern sensors will often
perform better.

\section{Brightness Units}

In the context of photometry, "brightness" is a catch-all term for a variety of units. Let's explore the relationships between these units to make later conversions more clear.

\subsection{Irradiance}

Irradiance is the standard SI linear unit used to describe the total amount of energy incident on a
surface from a given source. An irradiance of $1 \: \left[ \frac{W}{m^2} \right]$ implies that a $10
\: [m]$ area would experience $10 \: [W]$ of incident power. The Sun's irradiance at a distance of $1$ AU is known as the solar constant and is approximately $1361 \: \left[ \frac{W}{m^2} \right]$. 

Visual magnitude ---  also known as apparent or relative magnitude --- is a reverse logarithmic scale
that originates in astronomy \cite{frueh2019notes}. Stellar sources span many orders of magnitude of brightness, making a
logarithmic scale a helpful middle ground for comparison. Note that apparent magnitude always
expresses brightness at the observer's location; absolute magnitude is a different quantity that
normalizes brightness from a distance of $10$ parsecs \cite{frueh2019notes}. Apparent magnitude $m$
is computed from irradiance via Eq \ref{eq:irradiance_to_mag}.

\begin{equation} \label{eq:irradiance_to_mag}
  m = -2.5 \log_{10}\left( \frac{I}{I_0} \right)
\end{equation}

In Eq \ref{eq:irradiance_to_mag}, $I$ is the irradiance of the source of interest and $I_0$ is
irradiance of the zero-point source. This makes sense; substituting $I = I_0$ returns
$m=0$. The star Vega is usually taken to be the zero-point with irradiance $I_0 = 2.518021002\cdot
10^{-8} \: \left[ \frac{W}{m^2} \right]$ \cite{frueh2019notes}.

We can rearrange Eq \ref{eq:irradiance_to_mag} to compute irradiance from a given apparent magnitude,
yielding Eq \ref{eq:mag_to_irradiance}.

\begin{equation} \label{eq:mag_to_irradiance}
  I = I_0 \cdot 10^{-\frac{m}{2.5}}
\end{equation}

\subsection{Normalized Irradiance}

The light curve simulation methods presented in this work heavily use normalized irradiance, 
the irradiance of a source observed from a distance of $1$ meter. This is a non-standard quantity in the literature, but proves useful for the same reasons
absolute magnitude is used by astronomers. Adjusting sources to be at a standard distance allows us
to simulate and invert light curves in a non-dimensionalized space. This simplifies simulation and
makes the shape inversion optimizations more robust. To make the conversion explicit, irradiance observed
at a distance $r$ in meters from an object is converted to normalized irradiance $\hat{I}$ in watts via Eq
\ref{eq:irradiance_to_norm_irradiance}.

\begin{equation} \label{eq:irradiance_to_norm_irradiance}
  \hat{I} = r^2 I
\end{equation}

\subsection{$S_{10}$}

While apparent magnitude and irradiance are effective for quantifying the flux of point sources, other units exist
to describe diffuse or extended sources where brightness is spread over an
area. $S_{10}$ is a unit of surface brightness representing the number of 10th magnitude stars per square degree that would produce the same flux as a given diffuse source.
Surface brightness in $S_{10}$ over a given solid angle $\Omega \: \left[ sr \right]$ can be converted to total irradiance $I \: \left[ \frac{W}{m^2} \right]$ via Eq \ref{eq:s10toirrad}.

\begin{equation} \label{eq:s10toirrad}
 \frac{I \left[ \frac{W}{m^2} \right]}{S_{10}} = 10^{-10/2.5} \left( \Omega \frac{180^2}{\pi^2} \right)
  \int_{10^{-8}}^{10^{-6}}{ \textrm{STRINT}(\lambda) \: d\lambda} = 8.26617 \Omega \cdot 10^{-9}
\end{equation}

In \ref{eq:s10toirrad}, $\textrm{STRINT}(\lambda) \: \left[ \frac{W}{m^2 \cdot m} \right]$ is the
representative spectrum of a 0th magnitude star, $\textrm{QE}(\lambda)$ is the quantum efficiency
spectrum of the observing sensor, $\textrm{ATM}(\lambda)$ is the atmospheric transmission spectrum, $\lambda \: [m]$ is wavelength, $h \: \left[
\frac{m^2 \cdot kg}{s} \right]$ is Plank's constant, and $c \: \left[ \frac{m}{s^2} \right]$ is the
speed of light in vacuum. Quantum efficiency has units of photoelectrons which conveys the fraction of incident photons which are (proportionally) converted to photoelectrons in the CCD sensor. Atmospheric transmission is a unitless quantity conveying the fraction of light that is not absorbed by the atmosphere. Example spectra for $\textrm{ATM}(\lambda)$ and $\textrm{QE}(\lambda)$ are displayed in Figure \ref{fig:spectra}, with underlying data provided in Appendices \ref{data:atm} and \ref{data:qe}.

\subsection{Magnitude per Square Arcsecond}

A second surface brightness unit is $\left[ \frac{mag}{arcsec^2} \right]$, also known as MPSAS (magnitude per square arcsecond). This quantity can be thought of as a generalized $S_{10}$, where instead of quantifying the number of stars of a certain
magnitude in a solid angle, the equivalent magnitude of a single point source is measured. A surface
brightness $B_{10}$ in $S_{10}$ can be converted into surface brightness $B_{mag}$ in 
$\left[ \frac{mag}{arcsec^2} \right]$ via Eq \ref{s10_to_mag_per_a2}.

\begin{equation} \label{s10_to_mag_per_a2}
	B_{mag} = -2.5 \log_{10}\left( \frac{B_{10} \cdot 10^{-4}}{12960000} \right)
\end{equation}

In Eq \ref{s10_to_mag_per_a2} $S_{10}$ is first converted to the total irradiance per square degree,
convert square degrees to square arcseconds, and transform the result back into apparent magnitude. MPSAS is converted to irradiance per steradian via Eq \ref{eq:mpsas_to_irrad_per_ster} using \ref{eq:mag_to_irradiance}.

\begin{equation} \label{eq:mpsas_to_irrad_per_ster}
  I = \left( \frac{180}{ 3600\pi} \right)^2 I_0 \cdot 10^{-\frac{MPSAS}{2.5}}
\end{equation}

\subsection{Candela} \label{sec:candela}

Some light pollution datasets are given in units that include candela. Candela is the SI base unit of luminous intensity defined by the International Committee for Weights and Measures as "Fixing the numerical value of the luminous efficacy of monochromatic radiation of frequency $540\cdot10^{12}$ Hz to be equal to exactly $683$" \cite{nist_units}. This means that an isotropic green light source with frequency $540\cdot10^{12}$ Hz ($\lambda = 555$ nm) has a luminous efficacy of $K_{cd} = 683 \: \left[ lm/W \right]$ where lm stands for lumens. Luminous efficacy itself determines how well a source produces visible light. For a given wavelength, candela $B_{cd}$ is converted to watts per steradian $B_{wsr}$ via Eq \ref{eq:cd_to_w_per_sr} \cite{nist_units}

\begin{equation} \label{eq:cd_to_w_per_sr}
  B_{wsr}(\lambda) = \frac{B_{cd}}{K_{cd}(\lambda)}.
\end{equation}

The luminous efficiency function $K_{cd}(\lambda)$ models the human eye's response to the visible spectrum \cite{sharpe2005}. Different fits of this function exist; the function proposed Sharpe et al. is adopted, displayed in Figure \ref{fig:luminous_efficiency} \cite{sharpe2005}.

\begin{figure}[ht]
  \centering
  \includegraphics[width=\figmed]{sphx_glr_luminous_efficiency_001_2_00x.png}
  \caption{Luminous effiency function from \cite{sharpe2005}}
  \label{fig:luminous_efficiency}
\end{figure}

Candela per unit area can be converted into MPSAS by combining Eq \ref{eq:cd_to_w_per_sr} with \ref{eq:irradiance_to_mag}, yielding Eq \ref{eq:cd_per_m2_to_mpsas}, which is still a function of the source's wavelength.

\begin{equation} \label{eq:cd_per_m2_to_mpsas}
  MPSAS(\lambda) = -2.5 \log_{10}\left( \frac{B_{cd}}{\left( \frac{180}{ 3600\pi} \right)^2 K_{cd}(\lambda) I_0} \right)
\end{equation}

\subsection{Photoelectron Counts}

Raw images taken by a CCD-equipped telescope have pixel values measured in photoelectron
counts, otherwise known as Analog-to-Digital Units (ADU) \cite{krag2003}. The count in a single
pixel obtained is directly proportional (via the CCD's gain) to the number of
photons incident on that pixel during the integration time. Higher order effects in the silicon of
the CCD makes this description incomplete, but for non-resolved imaging applications
concerned about, effects smaller than the sensor readout noise and dark current can be safely neglected
\cite{frueh2019notes}. Irradiance can be converted to ADU via the conversion factor $SINT$
in Eq \ref{eq:sint} \cite{krag2003}.

\begin{equation} \label{eq:sint}
 \textrm{SINT} = \frac{\pi D^2}{4}
	\int_{10^{-8}}^{10^{-6}}{ \left( \frac{\textrm{SUN}(\lambda)}{I_{sun}} \right) \cdot \textrm{QE}(\lambda) \cdot \textrm{ATM}(\lambda)
  \cdot \left( \frac{\lambda}{h c} \right) \: d\lambda}  
\end{equation}

In Eq \ref{eq:sint}, $\textrm{SUN}(\lambda)$ is the spectrum of solar irradiance in 
$\left[\frac{W}{m^2\cdot m} \right]$, $I_{sun}$ is the irradiance of the Sun (generally taken to be
the solar constant $1361 \: \left[ \frac{W}{m^2} \right]$. Read literally, the integral term as
units $\left[ \frac{1}{Ws} \right]$, giving the number of counts per incident Watt of solar
radiation and second of integration time. The aperture diameter factor outside the imtegral accounts
for the area of light incident on the CCD, giving $\textrm{SINT}$ units of $\left[ \frac{m^2}{Ws}
\right]$. The spectra in Eq \ref{eq:sint} are plotted in Figure \ref{fig:spectra} with data in Appendix \ref{data:spectra}. Multiplying by irradiance in $\left[ \frac{W}{m^2} \right]$ and an integration time $\Delta t$ 
in seconds will yield the count of photoelectrons $S$ in ADU as shown in Eq \ref{eq:irrad_to_count}.

\begin{equation} \label{eq:irrad_to_count}
  S = \textrm{SINT} \cdot I \cdot \Delta t
\end{equation}

For completeness, irradiance can be recovered from a signal in ADU and the integration time via Eq
\ref{eq:count_to_irrad}.

\begin{equation} \label{eq:count_to_irrad}
  I = \frac{S}{\textrm{SINT} \cdot \Delta t}
\end{equation}