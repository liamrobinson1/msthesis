\ProvidesFile{ch-results.tex}[Results]
\graphicspath{{/Users/liamrobinson/Documents/PyLightCurves/docs/build/html/_images}}

\section{Shape Inversion Results}

% \begin{figure}[!htb]
%   \centering
%   \includegraphics[width=\textwidth]{sphx_glr_lc_uncertainty_002_2_00x.png}
%   \caption{Light curves for a $1$-meter diffuse cube with $C_d = 0.5$ observed from the Purdue Optical Ground Station at \pogslla. Object is in the orbit of GOES 15 under torque-free rigid body motion with an inertia tensor $J = \mathrm{diag}(1.0, 2.0, 3.0) \: \left[kg \cdot m^2\right]$, $q_0 = \left[0, 0, 0, 1\right]^T$, $\omega_0 = \left[ 0.01, 0.02, 0.01 \right] \: \left[rad/s\right]$}
%   \label{fig:cube_lcs}
% \end{figure}


\subsection{Object Reconstruction Improvements}

The proposed resampling and merging steps in the object reconstruction process produce sparser results, in less computation time, with fewer convergence issues. Figure \ref{fig:egi_reconstructions} shows object reconstructions with each available EGI. The resampled EGIs do not converge in the maximum number of iterations evaluations ($100$) due to the density of faces with nearby normal vectors, causing large linearization errors the gradient estimation that lead to inefficient optimizer steps. It is clear that the final merged EGI produces the most accurate reconstruction of the truth object.

\begin{figure}[!htb]
  \centering
  \includegraphics[width=\textwidth]{sphx_glr_egi_figs_aas22_006.png}
  \caption{Shape inversions from the initial, resampled, and merged EGIs}
  \label{fig:egi_reconstructions}
\end{figure}


\subsection{Convex Inversion Noise Dependence}

The effect of the noise model on the inversion process is most clearly seen when the object signal is lowered while holding the noise distribution constant for the same observation scenario. This can be accomplished by placing the object in a more distant orbit or making the object less reflective. Figure \ref{fig:noisy_convex_inversions} displays the effect of shrinking a cube with a Phong BRDF ($C_d=0.5$, $C_s=0.5$, $n=10$) in a tumbling attitude profile in the orbit of the GOES 15 satellite over a three hour light curve captured at $10$ second intervals beginning at 2023-03-26 10:00:00 UTC. These inversions were performed with the full convex shape estimation process: beginning with generating the light curve using the same process as Section \ref{sec:case2} and inverting the shape using the same BRDF with Eq \ref{eq:area_opt_convex} with the proposed resampling and merging steps defined by Eqs \ref{eq:cone_sample_n_pole}, \ref{eq:cone_sampling_rv}, and \ref{eq:egi_merge} before optimizing the supports with Eq \ref{eq:little_problem} and recovering the final geometry with the dual transform in Eq \ref{eq:dual_egi_form} and Eq \ref{eq:vertex_recovery} for the final vertices. Taking the convex hull of the vertices yields the face adjacency information, completing the convex inversion process.

\begin{figure}[!htb]
  \centering
  \includegraphics[width=\textwidth]{sphx_glr_noisy_inversion_pdf_001.png}
  \caption{Ten samples of convex shape inversions of a cube from the same set of noisy measurements for each width, with full cube side width and mean observation SNR noted}
  \label{fig:noisy_convex_inversions}
\end{figure}

Figure \ref{fig:noisy_convex_inversions} illustrates the degradation in the convex shape estimate as the SNR falls. A few important conclusions can be drawn from this result. First, the errors in the estimates with high SNRs cluster in the corners of the geometry; the first sign of noise adversely effecting inversion results is the existence of small spurious faces between the true faces. This further motivates the existence of the resampling and merging steps of the inversion as they encourage more accurate normal vector locations than a simple coarse initial sampling. Despite these new stages, a low enough SNR will still degrade the shape estimate. Second, it is interesting to notice that the boundary between the mostly-correct shape estimates for a width of $0.05$ meters and the highly noise-corrupted estimates at a width of $0.04$ meters coincides with SNRs slightly above the detection limit of the CCD at $SNR\approx 4$. This makes intuitive sense, as detecting the object in the image and attributing a photon count to its signal are similar processes --- failure to detect the object implies a failure to quantify its signal.

\clearpage
\section{Inversion of Convex Shapes Without Noise}

Four representative shapes were selected for the convex inversion results presented in this work. The cube and icosahedron are regular shapes with a low and medium level of detail respectively, testing the inversion's ability to recover accurate flat facet areas and normal vectors. The cylinder tests how the inversion process responds to smooth surfaces, while the gem is both asymmetric about one of its body axes and has a high level of detail. Performing single-shot shape inversions of these geometries using the conditions in Table \ref{tb:cvx_props} yields Figure \ref{fig:res_convex_no_noise}. These inversions were performed with a merge angle $\theta_\mathrm{merge} = \pi / 8$ and $2000$ normal vector candidates for the initial EGI optimization. Nonconvex inversions were performed on top of the convex guess with the line search detailed in Algorithm \ref{alg:concavity_iter} with $3$ iterations of linear subdivision and the vertex selection tolerance $\varepsilon = 0.5 \: [rad]$ as needed by Eq \ref{eq:root_free}. The nonconvex guess with the lowest light curve error is plotted.

\begin{table}[]
  \centering
  \begin{tabular}{|l|l|}
  \hline
  \textbf{Parameter} & \textbf{Value} \\ \hline
  SATNUM & $36411$ \\ \hline
  $q_0$ & $\left[ 0, 0, 0, 1 \right]^T$ \\ \hline
  $w_0$ & $\left[ 0.04082483, 0.08164966, 0.04082483 \right]^T$ $[rad/s]$ \\ \hline
  $J$ & $\mathrm{diag}\left( 1.0, 2.0, 3.0 \right)$ $\left[ kg \cdot m^2 \right]$ \\ \hline
  BRDF & Phong \\ \hline
  $C_d$ & $0.5$ \\ \hline
  $C_s$ & $0.5$ \\ \hline
  $n$ & $10$ \\ \hline
  Initial date & 2023-03-26 05:00:00 UTC \\ \hline
  $\Delta t$ & $10$ $[s]$ \\ \hline
  $n_{obs}$ & $1080$ $[s]$ \\ \hline
  Telescope & Table \ref{tb:pogs_parameters} \\ \hline
  $\phi_{geod}$ & $32.900^\circ$ \\ \hline
  $\lambda$ & $-105.533^\circ$ \\ \hline
  $SNR_{min}$ & $3$ \\ \hline
  $\theta_{Moon,ex}$ & $30^\circ$ \\ \hline
  $\theta_{elev,min}$ & $15^\circ$ \\ \hline
  \end{tabular}
  \caption{Object and station simulation parameters for shape inversion results}
  \label{tb:cvx_props}
\end{table}

\begin{figure}[!htb]
  \centering
  \includegraphics[width=\textwidth]{sphx_glr_aas_non_convex_inversion_002.png}
  \caption{Shape inversion results for convex objects without noise in the light curve, the most accurate shape estimate for each truth object in terms of the residual light curve error is shaded green}
  \label{fig:res_convex_no_noise}
\end{figure}

Figure \ref{fig:res_convex_no_noise} displays convex and nonconvex inversion results for convex truth objects without noise in the light curve. All shapes are reconstructed faithfully, and the concavities introduced in the nonconvex estimates are small, indicating that the concavity optimization correctly identified that there is no self-shadowing taking place in the observations. The normal vector angle merge tolerance $\theta_\mathrm{merge}$ causes the faces of the cylinder to be truncated, although the resulting geometry is still recognizable as a cylinder. Smooth surfaces can also be reintroduced into the final geometry by adding iterations of Loop subdivision in place of an equal number of linear subdivision passes. This discretization loss is also apparent in the upper facets of the gem where some detail is lost. 

\clearpage
\section{Inversion of Nonconvex Shapes Without Noise}

Figure \ref{fig:res_nonconvex_no_noise} displays shape inversions for nonconvex objects with a variety of concavity geometries under the same attitude profile, BRDF, and observation geometry of listed in Table \ref{tb:cvx_props}.

\begin{figure}[!htb]
  \centering
  \includegraphics[width=\textwidth]{sphx_glr_aas_non_convex_inversion_003.png}
  \caption{Shape inversion results for nonconvex objects without noise in the light curve, the most accurate shape estimate for each truth object in terms of the residual light curve error is shaded green}
  \label{fig:res_nonconvex_no_noise}
\end{figure}

The reconstructed convex objects in Figure \ref{fig:res_nonconvex_no_noise} display the behavior of the standard inversion pipeline when self-shadowing occurs in the light curve. The optimized EGIs for these objects are quite similar to the true EGI of the nonconvex object, but each face area above the horizon of the concavity is too small in the shape guess. This is because the constrained least squares solution attempts to minimize the error between all measurements involving a face, including those where that face is self-shadowed. As a result, the optimal face area that minimizes the residual brightness is necessarily smaller than the true area as NNLS cannot simulate the self-shadowing analytically. 

The nonconvex guesses for each object introduce a small concavity in the direction of the EGI error vector --- the approximate direction of the concavity in the body frame --- but the magnitude of these concavities is too small. The converged line search detailed in Algorithm \ref{alg:concavity_iter} finds that the concavity angle $\psi_{opt}$ that minimizes the light curve error is quite close to the convex guess. This is another artifact of the NNLS EGI optimization. The optimal EGI is effected globally by the presence of self-shadowing in one area, leading to a shape which is in some sense at a local minimum of brightness error with respect to any small part of its geometry --- no small tweaks to its geometry are likely to improve the error. As a result, introducing a large concavity which is closer to the truth may reduce the error in some measurements where self-shadowing is occuring, but ruins other measurements where it is not. As a result, the convex guesses have lower light curve error in all cases, even though a small concavity was introduced in the correct direction.

\clearpage
\section{Inversion of Convex Shapes With Noise}

\begin{figure}[!htb]
  \centering
  \includegraphics[width=\textwidth]{sphx_glr_aas_non_convex_inversion_004.png}
  \caption{Shape inversion results for convex objects with noise in the light curve, the most accurate shape estimate for each truth object in terms of the residual light curve error is shaded green}
  \label{fig:res_convex_with_noise}
\end{figure}

With the addition of noise, 

\clearpage
\section{Inversion of Nonconvex Shapes With Noise}

\begin{figure}[!htb]
  \centering
  \includegraphics[width=\textwidth]{sphx_glr_aas_non_convex_inversion_005.png}
  \caption{Shape inversion results for nonconvex objects with noise in the light curve, the most accurate shape estimate for each truth object in terms of the residual light curve error is shaded green}
  \label{fig:res_nonconvex_with_noise}
\end{figure}

TODO: this

\section{Multiple Hypothesis Inversion}

\subsection{Case 1: Convex Truth}

Figure \ref{fig:multi_hyp_candidates_cvx} shows 16 inversion candidates for a convex cylinder produced with independent noisy light curves. Some of the individual estimates are clearly quite far from the true geometry shown in black wireframe. These individual estimates are then merged into a single final shape guess using the procedure laid out in Section \ref{sec:mh_inversion}, producing the shape in Figure \ref{fig:multi_hyp_final_cvx}.

\begin{figure}[!htb]
    \centering
    \includegraphics[width=\figbig]{sphx_glr_multi_hypothesis_inversion_001.png}
    \caption{Multiple hypothesis candidates from 16 nights of noisy observations of a $2$-meter cube in the orbit of GOES 15. The object is simulated with inverted with a Phong BRDF using $C_d=0.5$, $C_s=0.5$, $n=5$. Nightly light curves are each $3$ hours long with data collected every $10$ seconds, beginning January 1, 2023, 05:00 UTC.}
    \label{fig:multi_hyp_candidates_cvx}
\end{figure}

\begin{figure}[!htb]
  \centering
  \includegraphics[width=\figmed]{sphx_glr_multi_hypothesis_inversion_002.png}
  \caption{Final shape guess merged from candidates in Figure \ref{fig:multi_hyp_candidates_cvx}}
  \label{fig:multi_hyp_final_cvx}
\end{figure}

The final estimated geometry in Figure \ref{fig:multi_hyp_final_cvx} is closer to the truth when compared to some of the candidate objects, but is still imperfect. Notably, the sharp weight boundaries introduced by the spherical nearest neighbor interpolation scheme introduced in Section \ref{sec:nearest_sphere} produce small, sharp boundaries in the final geometry. This could be avoided by applying a smoothing factor to the interpolation at the boundaries. Additionally, the final result in Figure \ref{fig:multi_hyp_final_cvx} correctly rejects hypotheses with large geometric errors, weighting the more accurate objects more highly when constructing the final estimate.

\subsection{Case 2: Non-Convex Truth}

\begin{figure}[!htb]
  \centering
  \includegraphics[width=\figbig]{sphx_glr_multi_hypothesis_inversion_003.png}
  \caption{Multiple hypothesis candidates from 16 nights of noisy observations of a $2$-meter collapsed cube in the orbit of GOES 15. The object is simulated with inverted with a Phong BRDF using $C_d=0.5$, $C_s=0.5$, $n=5$. Nightly light curves are each $3$ hours long with data collected every $10$ seconds, beginning January 1, 2023, 05:00 UTC.}
  \label{fig:multi_hyp_candidates_ncvx}
\end{figure}

\begin{figure}[!htb]
\centering
\includegraphics[width=\figmed]{sphx_glr_multi_hypothesis_inversion_004.png}
\caption{Final shape guess merged from candidates in Figure \ref{fig:multi_hyp_candidates_cvx}}
\label{fig:multi_hyp_final_ncvx}
\end{figure}
