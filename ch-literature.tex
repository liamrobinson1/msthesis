\ProvidesFile{ch-literature.tex}[Literature]

\chapter{Literature}

Light curve simulation methods differ between approaches and the object class under study. Kaasalainen and Torppa employ a Lambertian model for convex objects with a facetwise ray tracing scheme for non-convex objects \cite{kaasalainen2001}. Fan, Friedman, Kobayashi, and Frueh \cite{fan2016, fan2020thesis,friedman2020,kobayashi2020,frueh2014} use a nearly identical scheme for human-made objects. Allworth et al. developed a ray traced simulator for light curves in Blender, accounting for photorealistic shadowing and motion blur \cite{allworth2020, allworth2021}. Many deep learning approaches including Furfaro et al. \cite{furfaro2019} and Cabrera and Bradley \cite{cabrera2021,bradley2014} use a simple Lambertian model with no self-shadowing. Linares and Crassidis \cite{linares2018space} apply a more specialized approach with a non-Lambertian Bidirectional Reflectance Distribution Function (BRDF) for lighting. McNally et al. \cite{mcnally2021} use a Phong BRDF without shadowing shadowing, citing computational intensity. Blacketer \cite{blacketer2022} implemented a Cook-Torrance BRDF for lighting with a plane stacking method for self-shadowing.

Methods for shape inversion fall into three major categories: Extended Gaussian Image (EGI), statistical estimation, and deep learning based methods, each approaching the problem from a different perspective.

Direct light curve inversion with the EGI uses a series of optimization problems to fit a convex shape to measurements. These methods were pioneered by Kaasalainen and Torppa for asteroids in \cite{kaasalainen2001} with simultaneous attitude inversion in \cite{kaasalainen2001}. While natural space objects like asteroids are largely convex, nearly all human-made space objects are highly non-convex, highlighting the need for a robust inversion scheme for both convex and non-convex space objects. The work of Kaasalainen et al. on asteroids was extended by Chng et al. \cite{chng2022} to find globally optimal spin pole and area vector solutions. Calef et al. \cite{calef2006photometric} were early adopters of Kaasalainen and Torppa's EGI methods for human-made objects, focusing on multispectrum measurements. Bradley and Axelrad \cite{bradley2014} applied EGI methods to recover convex approximations of representative GEO objects. Fan and Frueh \cite{fan2019, fan2020thesis, fan2021} used the EGI with a multi-hypothesis scheme to recover human-made object shapes with measurement noise. Friedman \cite{friedman2020, friedman2022} quantified the observability of EGI inversion to inform sensor tasking schemes. Cabrera et al. \cite{cabrera2021} studied the effects of area regularization on Fan and Friedman's methods to achieve more accurate reconstructions.

A second approach leverages statistical estimation to retrieve shape information. Linares et al. \cite{linares2012} applied an unscented Kalman filter to estimate attitude and convex shape simultaneously, representing shape with vertex displacement on a sphere. Linares et al. \cite{linares2014space} used a Multiple-Model Adaptive Estimation (MMAE) algorithm to predict the truth geometry and attitude by comparing observations with a bank of reference objects. Linares and Crassidis \cite{linares2018space} used an an Adaptive Hamiltonian Markov Chain Monte Carlo scheme to estimate shape and other characteristics simultaneously. 

A third approach relies on deep learning. Linares and Furfaro \cite{linares2016} used a deep convolutional neural network to classify novel light curves as rocket bodies, payloads, or debris. Furfaro et al. \cite{furfaro2019} used similar methods classify novel light curves into four truth object classes. Kerr et al. \cite{kerr2021} adapted the architecture developed by Furfaro et al. to classify object shape and size in an extended training set. McNally et al. \cite{mcnally2021} use AI and differential approaches to identify satellites from simulated and real light curves. Allworth et al. \cite{allworth2021} applied transfer learning to simulated and real measurements to classify object type.

Various other unique methods have been applied to the light curve shape inversion problem. Hall et al. \cite{hall2007} investigated methods for independently solving shape parameters in isolation without attitude information. Fulcoly et al. \cite{fulcoly2012} used measurements from different sensor locations to determine shape under various attitude profiles. Yanagisawa and Kurosaki \cite{yanagisawa2012} fit an analytical light curve model for a tri-axial ellipsoid to derive the shape and attitude profile of a Cosmos rocket body. Kobayashi applied compressed sensing to recover shape information from light curves by taking advantage of shadowing geometry \cite{kobayashi2020,kobayashi2021}.

Shape inversion for non-convex objects --- mainly asteroids --- has been studied by others in the past. Durech and Kaasalainen \cite{durech2003} determined a relationship between concavity size and the minimum solar phase angle where self-shadowing impacts the light curve. Viikinkoski et al. \cite{viikinkoski2017} investigated recovering large concavities from adaptive optics imagery, noting the fundamental non-uniqueness of any solution. They discuss how a single large concavity may produce identical scattering behavior to multiple smaller concave features \cite{viikinkoski2017}. Cabrera et al. \cite{cabrera2021} studied convex solutions for non-convex objects, concluding that the convex fit diverges from the true shape as the relative concavity size increases. 

We approach the shape inversion problem with the foundational EGI optimization and object reconstruction methods of \cite{kaasalainen2001,fan2020thesis}. The EGI optimization processes of \cite{fan2020thesis,cabrera2021,kaasalainen2001} are improved using novel resampling and merging steps. These improvements circumvent the need for the regularization terms explored by Cabrera et al. \cite{cabrera2021}. We also address the reconstruction scaling issues present in Fan's work \cite{fan2020thesis} with an objective function proposed by Ikeuchi et al. \cite{ikeuchi1981} in place of Little's \cite{little1985}. The support optimization procedure is accelerated and strengthened with a preconditioning term proposed by Nicolet et al. \cite{nicolet2021}, enabling the rapid reconstruction of more detailed convex objects than previously feasible.

Our approach has a number of general advantages. We do not require any \textit{a priori} information about the truth geometry. Thus, unlike MMAE methods \cite{linares2014space}, we do not require a bank of reference models to recover shape information. Unlike deep learning methods, our method does not rely on the diversity of a training set to achieve realistic results \cite{furfaro2019,kerr2021}. Our light curve simulation method improves on the facetwise ray traced shadows of \cite{kaasalainen2001,fan2020thesis,frueh2014} with shadow mapping, increasing shadow fidelity per unit computation time.
